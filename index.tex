% Options for packages loaded elsewhere
\PassOptionsToPackage{unicode}{hyperref}
\PassOptionsToPackage{hyphens}{url}
\PassOptionsToPackage{dvipsnames,svgnames,x11names}{xcolor}
%
\documentclass[
  letterpaper,
  DIV=11,
  numbers=noendperiod]{scrreprt}

\usepackage{amsmath,amssymb}
\usepackage{iftex}
\ifPDFTeX
  \usepackage[T1]{fontenc}
  \usepackage[utf8]{inputenc}
  \usepackage{textcomp} % provide euro and other symbols
\else % if luatex or xetex
  \usepackage{unicode-math}
  \defaultfontfeatures{Scale=MatchLowercase}
  \defaultfontfeatures[\rmfamily]{Ligatures=TeX,Scale=1}
\fi
\usepackage{lmodern}
\ifPDFTeX\else  
    % xetex/luatex font selection
\fi
% Use upquote if available, for straight quotes in verbatim environments
\IfFileExists{upquote.sty}{\usepackage{upquote}}{}
\IfFileExists{microtype.sty}{% use microtype if available
  \usepackage[]{microtype}
  \UseMicrotypeSet[protrusion]{basicmath} % disable protrusion for tt fonts
}{}
\makeatletter
\@ifundefined{KOMAClassName}{% if non-KOMA class
  \IfFileExists{parskip.sty}{%
    \usepackage{parskip}
  }{% else
    \setlength{\parindent}{0pt}
    \setlength{\parskip}{6pt plus 2pt minus 1pt}}
}{% if KOMA class
  \KOMAoptions{parskip=half}}
\makeatother
\usepackage{xcolor}
\setlength{\emergencystretch}{3em} % prevent overfull lines
\setcounter{secnumdepth}{5}
% Make \paragraph and \subparagraph free-standing
\ifx\paragraph\undefined\else
  \let\oldparagraph\paragraph
  \renewcommand{\paragraph}[1]{\oldparagraph{#1}\mbox{}}
\fi
\ifx\subparagraph\undefined\else
  \let\oldsubparagraph\subparagraph
  \renewcommand{\subparagraph}[1]{\oldsubparagraph{#1}\mbox{}}
\fi


\providecommand{\tightlist}{%
  \setlength{\itemsep}{0pt}\setlength{\parskip}{0pt}}\usepackage{longtable,booktabs,array}
\usepackage{calc} % for calculating minipage widths
% Correct order of tables after \paragraph or \subparagraph
\usepackage{etoolbox}
\makeatletter
\patchcmd\longtable{\par}{\if@noskipsec\mbox{}\fi\par}{}{}
\makeatother
% Allow footnotes in longtable head/foot
\IfFileExists{footnotehyper.sty}{\usepackage{footnotehyper}}{\usepackage{footnote}}
\makesavenoteenv{longtable}
\usepackage{graphicx}
\makeatletter
\def\maxwidth{\ifdim\Gin@nat@width>\linewidth\linewidth\else\Gin@nat@width\fi}
\def\maxheight{\ifdim\Gin@nat@height>\textheight\textheight\else\Gin@nat@height\fi}
\makeatother
% Scale images if necessary, so that they will not overflow the page
% margins by default, and it is still possible to overwrite the defaults
% using explicit options in \includegraphics[width, height, ...]{}
\setkeys{Gin}{width=\maxwidth,height=\maxheight,keepaspectratio}
% Set default figure placement to htbp
\makeatletter
\def\fps@figure{htbp}
\makeatother
% definitions for citeproc citations
\NewDocumentCommand\citeproctext{}{}
\NewDocumentCommand\citeproc{mm}{%
  \begingroup\def\citeproctext{#2}\cite{#1}\endgroup}
\makeatletter
 % allow citations to break across lines
 \let\@cite@ofmt\@firstofone
 % avoid brackets around text for \cite:
 \def\@biblabel#1{}
 \def\@cite#1#2{{#1\if@tempswa , #2\fi}}
\makeatother
\newlength{\cslhangindent}
\setlength{\cslhangindent}{1.5em}
\newlength{\csllabelwidth}
\setlength{\csllabelwidth}{3em}
\newenvironment{CSLReferences}[2] % #1 hanging-indent, #2 entry-spacing
 {\begin{list}{}{%
  \setlength{\itemindent}{0pt}
  \setlength{\leftmargin}{0pt}
  \setlength{\parsep}{0pt}
  % turn on hanging indent if param 1 is 1
  \ifodd #1
   \setlength{\leftmargin}{\cslhangindent}
   \setlength{\itemindent}{-1\cslhangindent}
  \fi
  % set entry spacing
  \setlength{\itemsep}{#2\baselineskip}}}
 {\end{list}}
\usepackage{calc}
\newcommand{\CSLBlock}[1]{\hfill\break\parbox[t]{\linewidth}{\strut\ignorespaces#1\strut}}
\newcommand{\CSLLeftMargin}[1]{\parbox[t]{\csllabelwidth}{\strut#1\strut}}
\newcommand{\CSLRightInline}[1]{\parbox[t]{\linewidth - \csllabelwidth}{\strut#1\strut}}
\newcommand{\CSLIndent}[1]{\hspace{\cslhangindent}#1}

\KOMAoption{captions}{tableheading}
\makeatletter
\@ifpackageloaded{tcolorbox}{}{\usepackage[skins,breakable]{tcolorbox}}
\@ifpackageloaded{fontawesome5}{}{\usepackage{fontawesome5}}
\definecolor{quarto-callout-color}{HTML}{909090}
\definecolor{quarto-callout-note-color}{HTML}{0758E5}
\definecolor{quarto-callout-important-color}{HTML}{CC1914}
\definecolor{quarto-callout-warning-color}{HTML}{EB9113}
\definecolor{quarto-callout-tip-color}{HTML}{00A047}
\definecolor{quarto-callout-caution-color}{HTML}{FC5300}
\definecolor{quarto-callout-color-frame}{HTML}{acacac}
\definecolor{quarto-callout-note-color-frame}{HTML}{4582ec}
\definecolor{quarto-callout-important-color-frame}{HTML}{d9534f}
\definecolor{quarto-callout-warning-color-frame}{HTML}{f0ad4e}
\definecolor{quarto-callout-tip-color-frame}{HTML}{02b875}
\definecolor{quarto-callout-caution-color-frame}{HTML}{fd7e14}
\makeatother
\makeatletter
\@ifpackageloaded{bookmark}{}{\usepackage{bookmark}}
\makeatother
\makeatletter
\@ifpackageloaded{caption}{}{\usepackage{caption}}
\AtBeginDocument{%
\ifdefined\contentsname
  \renewcommand*\contentsname{Table of contents}
\else
  \newcommand\contentsname{Table of contents}
\fi
\ifdefined\listfigurename
  \renewcommand*\listfigurename{List of Figures}
\else
  \newcommand\listfigurename{List of Figures}
\fi
\ifdefined\listtablename
  \renewcommand*\listtablename{List of Tables}
\else
  \newcommand\listtablename{List of Tables}
\fi
\ifdefined\figurename
  \renewcommand*\figurename{Figure}
\else
  \newcommand\figurename{Figure}
\fi
\ifdefined\tablename
  \renewcommand*\tablename{Table}
\else
  \newcommand\tablename{Table}
\fi
}
\@ifpackageloaded{float}{}{\usepackage{float}}
\floatstyle{ruled}
\@ifundefined{c@chapter}{\newfloat{codelisting}{h}{lop}}{\newfloat{codelisting}{h}{lop}[chapter]}
\floatname{codelisting}{Listing}
\newcommand*\listoflistings{\listof{codelisting}{List of Listings}}
\makeatother
\makeatletter
\makeatother
\makeatletter
\@ifpackageloaded{caption}{}{\usepackage{caption}}
\@ifpackageloaded{subcaption}{}{\usepackage{subcaption}}
\makeatother
\ifLuaTeX
  \usepackage{selnolig}  % disable illegal ligatures
\fi
\usepackage{bookmark}

\IfFileExists{xurl.sty}{\usepackage{xurl}}{} % add URL line breaks if available
\urlstyle{same} % disable monospaced font for URLs
\hypersetup{
  pdftitle={Libraries in the Digital Age: A Guide to Information Systems for Librarians and Information Professionals},
  pdfauthor={Princess Anne D. Balajadia; Jan Harvey B. Bonzon; Anne Frances B. Calceta; Jericho C. Diano; Frances Riscel S. Padin; Reyna M. Parman; Denzel G. Sarmiento; Angela Nicole C. Yap},
  colorlinks=true,
  linkcolor={blue},
  filecolor={Maroon},
  citecolor={Blue},
  urlcolor={Blue},
  pdfcreator={LaTeX via pandoc}}

\title{Libraries in the Digital Age: A Guide to Information Systems for
Librarians and Information Professionals}
\author{Princess Anne D. Balajadia \and Jan Harvey B. Bonzon \and Anne
Frances B. Calceta \and Jericho C. Diano \and Frances Riscel S.
Padin \and Reyna M. Parman \and Denzel G. Sarmiento \and Angela Nicole
C. Yap}
\date{2023-01-19}

\begin{document}
\maketitle

\renewcommand*\contentsname{Table of contents}
{
\hypersetup{linkcolor=}
\setcounter{tocdepth}{2}
\tableofcontents
}
\bookmarksetup{startatroot}

\chapter*{Preface}\label{preface}
\addcontentsline{toc}{chapter}{Preface}

\markboth{Preface}{Preface}

This is the class project of LIS 260.

In the rapidly evolving landscape of the digital age, information
systems play a pivotal role in shaping the way we live, work, and
interact. As we stand on the threshold of an era defined by
technological advancements, it is essential to comprehend the
significance of information systems and their transformative impact on
our society. This book delves into the intricate web of interconnected
technologies, processes, and people that form the backbone of
information systems, unraveling the mysteries and unveiling the
potential that lies within.

This book embarks on a journey, charting the evolution of information
systems and their profound impact on every facet of society. We delve
into the intricate dance between hardware and software, the pulse of
algorithms governing our digital experiences, and the ethical
considerations that arise as lines blur between physical and virtual
worlds. From the groundbreaking innovations that sparked the information
age to the cutting-edge advancements shaping our future, each chapter
becomes a window into the intricate workings of these systems that
invisibly govern our interactions.

We encounter the visionaries who dared to dream of a hyperconnected
world, the architects who translated those dreams into tangible code,
and the everyday users whose lives are constantly transformed by the
systems they utilize. In their struggles and triumphs, we discover the
transformative power of information, its capacity to empower, connect,
and even challenge established norms.

This book is not merely a technical manual or a celebratory ode to
technological progress. It is a critical examination of the profound
influence information systems wield in shaping our present and future.
We grapple with the ethical dilemmas posed by algorithmic
decision-making, the widening digital divide, and the ever-present
threat of misuse of powerful tools. Through these critical lenses, we
aim to foster a deeper understanding of the systems that govern our
lives, empowering us to become not just users, but informed navigators
of the information age.

So, join us on this journey as we delve into the captivating world of
information systems. Prepare to be surprised, challenged, and
ultimately, empowered by the knowledge that within these interconnected
networks lies the potential to shape a future driven by informed
choices, meaningful connections, and a deeper understanding of the world
around us.

\bookmarksetup{startatroot}

\chapter{History of Information
Systems}\label{history-of-information-systems}

\section{Learning objectives}\label{learning-objectives}

\emph{At the end of this chapter, you should be able to:}

\begin{itemize}
\tightlist
\item
  Define information systems;
\item
  Identify and understand the origins and evolution of information
  systems;
\item
  Explore the emergence of key technologies;
\item
  Analyze the significant historical events that shaped the trajectory
  of information systems;
\item
  Identify how historical legacies impact the present landscape of
  information systems.
\end{itemize}

\section{Introduction}\label{introduction}

In today's digital age, information systems play a crucial role in our
daily lives. From the moment we wake up and check our smartphones to the
time we go to bed and browse through social media, we are constantly
interacting with various forms of information systems. But have you ever
wondered about the origins of these systems and how they have evolved
over time?

In this chapter, we will delve into the fascinating history of
information systems, tracing their roots back to ancient civilizations
and exploring their transformation into the sophisticated technologies
we rely on today. By understanding the history of information systems,
we can gain valuable insights into their development, impact, and
potential future advancements.

From early manual record-keeping methods employed by ancient, to the
revolutionary inventions like punch cards and early mainframe computers
in the 20th century, each milestone in the history of information
systems has shaped our modern world in significant ways.

We will explore key developments such as the birth of databases, the
rise of personal computers, and the advent of networking technologies
that paved the way for global connectivity. Additionally, we will
examine how advancements in software development led to powerful
applications that transformed industries ranging from finance to
healthcare.

By studying this rich history, we can appreciate not only how far
information systems have come but also anticipate their future
trajectory. We will uncover the past and present state of information
systems---a foundation upon which our digital society is built.

\textbf{\emph{What is an information system?}}

Migrating from a traditional to a digital world, information systems are
considered the backbone of modern organizations, enabling them to
effectively manage and utilize information for decision-making and
operational processes. In simple terms, an information system (IS) is a
sociotechnical system designed to collect, store, process, and
distribute data in a structured manner (Watson, 2007). It is an
intricate blend of technology, people, and processes that work together
to transform raw data into meaningful information, ultimately supporting
decision-making and fulfilling various organizational objectives.

At its core, we can think of an information system as a well-oiled
machine with the following components (What is an information system?,
2022):

\begin{itemize}
\tightlist
\item
  Hardware: Computers, servers, network devices, and other physical
  equipment.
\item
  Software: Operating systems, applications, and database programs.
\item
  Data: Raw facts and figures collected from various sources.
\item
  People: Users, analysts, managers, and anyone who interacts with the
  system.
\item
  Processes: Defined procedures for handling data, generating
  information, and completing tasks.
\end{itemize}

Information systems play a pivotal part in organizations by facilitating
efficient communication between different departments or individuals
within an organization. They enhance productivity by automating routine
tasks and providing access to accurate and timely information.
Understanding the concept of information systems is vital for
individuals seeking to navigate the increasingly digital landscape of
today's world.

\textbf{\emph{Why are information systems essential?}}

In today's fast-paced world, where data is generated at an unprecedented
rate, information systems play a crucial role in every imaginable
domain, significantly impacting individuals, organizations, and society
as a whole. It revolutionizes the way businesses operate and enhances
their overall efficiency and productivity.

In the realm of business, information systems are vital for streamlining
operations and facilitating efficient communication, thus enhancing
decision-making processes and boosting efficiency and productivity of
businesses (Denning, 2023). From inventory management to customer
relationship management, these systems provide real-time data that
empowers organizations to make informed choices and gain a competitive
edge in the market. Companies using information systems may also improve
their customer service by setting up online customer portals, chatbots,
and personalized recommendations to foster improved customer
satisfaction and loyalty (Jenneboer et al., 2022).

Education has also greatly benefited from information systems. With
digital learning platforms and online resources, students can access a
wealth of knowledge from anywhere in the world. Compared to the
traditional way of learning, information systems opened access to vast
collections of digital resources to provide students with diverse and
engaging learning experiences. Through virtual classrooms, learning
management systems (LMS) and other communication tools, information
systems support collaboration and communication between learners and
teachers, regardless of their current location. Information systems
facilitate personalized learning experiences and enable educators to
track student progress effectively.

In the field of healthcare, information systems enable seamless
communication between healthcare providers, improving patient care
coordination and reducing medical errors. Electronic health records
(EHR) systems, telehealth platforms and other diagnostic tools have
transformed how patient data is stored and accessed, allowing for better
decision-making and personalized treatment plans, as well as better
communication between healthcare professionals and patients. Data
analysis tools and simulations also helped accelerate medical research
and development, leading to quicker advancements in treatments and
disease prevention.

In government sectors, information systems play a pivotal role in
enhancing public service. From citizen registration to tax collection,
online e-government platforms and public information systems provide
convenient access to government services and operations in order to
ensure efficiency and transparency in governance processes, and in turn,
improve citizen engagement and satisfaction.

Information systems also have a significant impact on environmental
sustainability. From vigilant sensors guarding air, water, and forests
to platforms uniting conservation champions, information systems are
weaving a digital tapestry of environmental protection. They monitor and
analyze environmental factors, optimize resource use, educate
stakeholders, and connect individuals and organizations working towards
a greener future, transforming our planet into a seamlessly protected
and sustainably nourished sanctuary. These kinds of platforms can
connect individuals and organizations working on environmental
initiatives, fostering collaboration and sharing best practices.

The importance of information systems extends beyond these domains
mentioned above; it permeates nearly every industry imaginable. In an
interconnected world driven by data-driven decision-making processes,
organizations that leverage robust information system infrastructures
gain a competitive edge.

\textbf{\emph{Before bytes : when counting came alive}}

The seeds of information systems were sown long before the first
computer whirred to life. Let us delve into the ingenious ways our
ancestors managed information in the centuries preceding the
technological revolution.

Imagine a world where the whispers of calculation were not the whirring
of fans or the hum of circuits, but the gentle clatter of wooden beads
against a frame. This was the world of the abacus, a Mesopotamian
invention dating back to roughly 2700 BC that transformed numerical
manipulation with its elegant simplicity. Through practiced movements,
users could add, subtract, multiply, and even divide with remarkable
speed and accuracy. The brilliance of the abacus lays in its tangible
representation of numbers. Unlike abstract symbols, the beads provided a
physical connection to mathematical concepts, making them accessible to
a wider audience in societies where literacy was rare. This power tool
found favor not just with merchants and traders, but also with
astronomers charting the heavens and architects envisioning monumental
structures.

\textbf{\emph{The legacy of early writing technologies}}

In Mesopotamia and the Indus Valley, clay tablets rose to prominence.
Molded from wet earth, these sturdy pages bore the intricate scars of
cuneiform. Wedge-shaped symbols, pressed by a stylus, etched tales of
gods and kings, administrative records, and celestial observations onto
these durable tomes.

Across the sands of time, Egypt unveiled its secrets on papyrus, a
supple canvas woven from the papyrus plant. With brushes dipped in a
concoction of soot and gum Arabic, scribes painted hieroglyphs, a
symphony of pictograms and ideograms that adorned temple walls, papyrus
scrolls, and sarcophagi. From poetic verses to medical treatises, these
vibrant symbols sang of a civilization in thrall to both the earthly and
the divine. Moreover, stone and metal became the chosen canvases for
pronouncements. Hieroglyphs, carved with meticulous precision, adorned
monuments and temple walls, whispering tales of power and eternity.

Further west, the Greco-Roman world embraced the practicality of animal
skins and parchment. Vellum, crafted from treated hides, provided a
durable and portable surface for official documents, maps, and the
burgeoning libraries of knowledge. These supple skins served as the
cradle for epic poems, historical accounts, and the philosophical
musings of ancient minds.

In the verdant tapestry of East Asia, wood and bamboo whispered wisdom
onto slender slips. Bamboo and wooden tablets, inscribed with delicate
brushstrokes and bound with silken cords, gave birth to elegant scrolls
teeming with intricate characters. From philosophical treatises to
meticulously documented court proceedings, these slender pages
chronicled the rise and fall of dynasties, the whispers of poets, and
the wisdom of scholars. Around 105 AD, paper, crafted from humble plant
fibers, was a seismic shift in the history of information. Lightweight
and portable, it surpassed the limitations of its predecessors -- the
bulky clay tablets of Mesopotamia, the heavy papyrus scrolls of Egypt.
Paper offered a canvas for words to dance, diagrams to unfurl, and ideas
to take flight.

The ingenious minds of the Incas crafted a unique way to weave
information into knots. Quipus, intricate braids of llama wool, held
within their colorful fibers a complex language of knots, colors, and
spacing. This seemingly unassuming tool served as a sophisticated system
for recording everything from taxes and inventories to historical
narratives and military campaigns.

The ancient Greeks and Romans found a whisper of efficiency in wax
tablets. These reusable slates, coated with a thin layer of beeswax,
allowed for quick notation and messages, erased and renewed with a
simple stylus. These tablets echoed whispers of daily life in bustling
empires.

Before towering libraries of glass and steel became beacons of
information, nestled amidst ancient citadels and temples lay the humble
yet awe-inspiring seeds of knowledge preservation. As early as the 3rd
millennium BC, the rich minds of Mesopotamia and Egypt gave birth to
libraries, sanctuaries for scrolls and clay tablets teeming with
knowledge and history. These were not just mere collections of dusty
archives; they were vibrant centers of learning, research, and
intellectual exchange, humming with the whispers of scholars and scribes
navigating the labyrinthine paths of knowledge.

In Mesopotamia, the city of Nippur housed the famed library of Ekur, its
clay shelves bearing cuneiform-etched tablets on everything from
astronomy and mathematics to epic poems and legal codes. In Egypt, the
Library of Alexandria, a monument to intellectual grandeur, boasted
scrolls on philosophy, medicine, and history, whispering secrets across
civilizations and fostering the cross-pollination of ideas. These
libraries were not just static repositories; they were living organisms,
constantly replenished with newly inscribed tablets and scrolls,
ensuring the continuity of knowledge through generations.

Beyond the confines of physical libraries, intricate record-keeping
systems blossomed across the ancient world. The Egyptians, with their
obsession with order and control, crafted sophisticated bookkeeping
systems based on papyrus scrolls and hieroglyphics, meticulously
tracking everything from grain harvests to land ownership. In imperial
China, bureaucrats developed detailed census records where they logged
populations, resources, and tax levies, ensuring the smooth functioning
of their vast empire. These pre-modern data management practices laid
the foundation for the systematic organization and utilization of
information that shapes our world today.

The early writing and recordkeeping technologies transcended the mere
preservation of facts and figures. They fostered a culture of learning
and inquiry, nurturing scholars and scientists who pushed the boundaries
of knowledge. They served as bridges across time and space, transmitting
the wisdom of past generations to future minds, ensuring the continuity
of civilizations and the evolution of human understanding.

\textbf{\emph{When machines did the math}}

The humble beginnings of computing started before pocket calculators and
flashing screens, where ingenious minds wrested mathematical prowess
from gears and punched-paper melodies. This period witnessed the dawn of
machines specifically designed to automate and simplify arithmetic
calculations.

In 1642, 19-year-old Blaise Pascal, with his youthful mind teeming with
mathematical fervor, birthed the pascaline. This mechanical tool, a
symphony of cogs and wheels, tackled addition, subtraction,
multiplication, and even division with an accuracy and speed that
astounded the contemporary world. Its influence resonated for decades,
inspiring a lineage of calculating machines that yearned to liberate
calculations from the tedium of human hands.

Gottfried Wilhelm Leibniz, another genius of mathematical thought,
invented in 1672 his Stepped Reckoner. This advanced calculator boasted
the ingenious ``Leibniz wheel,'' a stepped drum that transformed
multiplication and division into a series of repeated additions. With
its movable carriage and cursors, it foreshadowed the elegant form of
future calculators like the pocket-sized Curta.

As the 19th century dawned, Charles Thomas, a skilled engineer, unveiled
the arithmometer. This robust machine, honed for reliability and
functionality, transcended the limitations of its predecessors. It
became the ruler of calculations for businesses, scientists, and
engineers, its rhythmic clicking a familiar soundtrack in the offices of
progress.

But computing was not just confined to mere crunching of numbers. In
1801, Joseph Marie Jacquard's Jacquard loom shattered the boundaries of
textile creation. This mechanical tool wove intricate designs not by the
whims of human hands, but by the precise dictations of punched cards,
each hole a coded instruction in the fabric of a pattern. The concept of
stored programs, instructions dictating a machine's actions, had taken
root, and its echoes would vibrate through future computing giants.

Charles Babbage, often hailed as the ``father of computing,'' envisioned
a machine that transcended mere calculations. In 1837, he conceived the
Analytical Engine. Armed with a programmable memory, an arithmetic logic
unit, and even an output device, it boasted a resemblance to the
computers that now hum in our pockets and grace our desks. Its influence
resonated far and wide, inspiring geniuses like Ada Lovelace, often
credited as the world's first programmer, and laying the groundwork for
the computational revolutions that were to come.

These early inventions, though humble in their capabilities and often
confined to the dusty shelves of history, were the fertile seeds from
which modern computing bloomed. They showcased the human mind's
insatiable hunger for automating processes, for weaving logic into gears
and instructions into fabrics.

\textbf{\emph{The rise of electronic computation}}

The 20th century witnessed the birth and rapid evolution of electronic
computers, transforming humanity's relationship with information. In
1936, a brilliant mathematician named Alan Turing invented a conceptual
masterpiece: the Turing machine. This theoretical contraption, made with
tape and symbols, held the key to understanding the very essence of
computational power. It proved that any conceivable calculation could be
reduced to this simple model. This revelation was not just an
intellectual exercise; it became the guiding light for future computer
architects, the Rosetta Stone translating theoretical possibilities into
practical machines.

In 1946, the colossal ENIAC (Electronic Numerical Integrator and
Computer) came onto the scene, a 30-ton tool that redefined the speed
and scale of computation. While its labyrinthine circuitry and
punched-card programming could leave modern users baffled, it crushed
numerical problems at unprecedented rates, revolutionizing scientific
and engineering calculations. ENIAC, though limited to government
projects, planted the seeds for smaller, more accessible machines to
sprout.

In 1951, Remington Rand unveiled the UNIVAC I (Universal Automatic
Computer), the first commercially available computer. This mechanical
tool brought computing to the realm of business. Companies eagerly
embraced its abilities to crunch numbers, manage records, and even
predict election results. UNIVAC I demonstrated that computers were not
just for government labs; they had the potential to transform everyday
life.

However, the revolution needed a leader, and IBM rose to the challenge.
From 1955 onwards, their series of mainframe computers, starting with
the iconic IBM 650, became the workhorses of corporate computing. These
powerful machines, housed in temperature-controlled chambers and tended
to by technicians in white coats, served as the centralized brains of
organizations, processing massive datasets and running complex business
applications. They were the gatekeepers of information, powering
everything from payroll calculations to inventory management.

Yet, these titans of silicon and steel needed a way to communicate.
Early computers spoke in the cryptic language of machine code, a binary
dialect understood only by a select few. But then came the dawn of
programming languages like Fortran and COBOL, simplifying the
complexities of machine code into human-readable instructions.
Scientists could now model molecules, businesses could analyze markets,
and programmers could weave elaborate software tapestries, all thanks to
these new linguistic tools.

These inventions, from Turing's theoretical elegance to UNIVAC's
commercial triumph, were not just isolated breakthroughs; they laid the
foundation for the personal computers that transformed homes and
offices, the mobile devices that fit in our pockets, and the vast
networks that connect us across the globe.

\textbf{\emph{The era of mini and microcomputers}}

The landscape of computing in the 1960s was dominated by hulking
mainframes that symbolized the technological elite. Yet, a shrinking
revolution would bring the power of computation down to size and into
the hands of individuals.

Digital Equipment Corporation (DEC) led the charge with their PDP
series, the first minicomputers. These smaller, more affordable machines
were utilized by universities, hospitals, and even businesses, tackling
specific tasks like crunching scientific equations, managing data, and
even delving into word processing. Compared to the previous machines,
minicomputers were just whispers in the metal jungle, offering
affordability, portability, and user-friendly programming.

In 1971, Intel unveiled the microprocessor, a miniature tool that
crammed the core of a computer onto a single circuit. This was not just
a size reduction; it was a power surge. Processors like the Intel 4004,
and their even mightier successors, the 8080 and Zilog Z80, were more
nimble and powerful than ever before, paving the way for a new era: the
era of personal computers.

The late 1970s and 1980s saw the birth of a digital family. IBM, with
its open architecture and standard operating system, brought us the IBM
PC, the patriarch of the industry, sparking an era of innovation and
competition. Apple, on the other hand, offered a different path with the
Apple II, offering a user-friendly interface and graphics that painted a
brighter future for personal computing. These machines were not just
number-crunchers; they were gateways to information, playgrounds for
software, and windows to a blossoming internet.

However, as PCs bloomed, so did the software industry. Companies like
Microsoft, WordPerfect, Lotus Development, and Adobe became the digital
tailors, stitching together applications for productivity,
entertainment, and education. From spreadsheets that juggled finances to
paint programs that unleashed creativity, each software offering added a
new verse to the prospering language of personal computing.

The downsizing revolution was not just about shrinking circuits; it was
about democratizing knowledge, empowering individuals, and painting a
future where technology was not confined to laboratories and
corporations. From the minicomputers that whispered the first promises
of accessibility to the personal computers that roared onto desktops,
this era was not just about technological advancements; it was about
humanity reaching out and claiming its rightful place in the digital
landscape.

\textbf{\emph{The transformative revolution of online connectivity}}

Long before cat videos ruled the internet and social media storms raged,
connectivity already pulsed through the wires of 1969. The ARPANET
(Advanced Research Projects Agency Network), a government-funded network
born from the anxieties of the Cold War, was the seed of what would
become the internet. This network, connecting four main hubs (the
Universities of California in Santa Barbara and Los Angeles, the
University of Utah, and SRI International), spoke a new language: the
revolutionary packet-switching technology. Instead of data traversing
dedicated pathways, it broke into smaller packets, zipping through
shared lines like nimble couriers in a bustling marketplace. This
innovation, the lifeblood of efficient network communication, was the
first block of the internet.

In this case, connectivity needed a common tongue. Enter the TCP/IP
protocol suite, the universal translator of the digital world. Emerging
in the early 1970s, it became the language of networks, a set of rules
dictating how data travels, finds its way, and corrects its stumbles.
With TCP/IP, once-isolated networks could converse, paving the way for a
truly global dialogue.

In 1990, Tim Berners-Lee, a visionary weaver of information, conjured
the World Wide Web. This ingenious system, made from hyperlinked
documents, transformed the internet from a text-based terrain into a
world heaving with images, sounds, and interactive possibilities. Enter
Mosaic, the first popular web browser, a digital program that unveiled
the web's vibrant canvas to the masses. With clicks and scrolls, the
internet was no longer the playground of tech titans; it was open to
anyone with a curious mind and a modem's hum.

The 1990s witnessed the internet's triumphant march into the commercial
realm. Dial-up connections emerged, and businesses, eager to capitalize
on this ground, entered online advertising, e-commerce, and a whirlwind
of new industries. This was the era of the dot-com boom, getting lost in
internet ventures and skyrocketing valuations. But bubbles, like
overinflated hopes, eventually burst. The dot-com crash of 2000 served
as a sobering reminder of the need for caution and adaptability in this
ever-evolving digital landscape.

From the ARPANET to the stridency of clicks and swipes, the story of the
internet is a testament to human ingenuity and its insatiable hunger for
connection. As we navigate the ever-shifting terrain of the web, it is
crucial to remember the threads that bind us, the protocols that enable
our discourse, and the pioneers who dared to dream of a world where
information flows freely across continents and cultures, click by click,
byte by byte.

\textbf{\emph{From static pages to sharing stages : the rise of web 2.0
and social media}}

Remember the internet of yesteryear? It was a static landscape of
text-heavy pages, served up from distant servers and devoured by passive
audiences. This was Web 1.0, a one-way street where information flowed
downhill, from creators to consumers. But just as VHS tapes gave way to
streaming services, the internet underwent a seismic shift, morphing
into the dynamic, interactive phenomenon we know as Web 2.0.

This was not just a technological upgrade; it was a philosophical
revolution. Web 2.0 made ways for user-generated content (UGC). Blogs
became soapboxes, wikis transformed into collaborative encyclopedias,
and social media platforms blossomed into vibrant digital town squares.
No longer were we mere receivers of information; we were creators,
curators, and collaborators, crafting information on the internet with
our own voices and ideas.

Interactivity became the watchword. We were commenting, rating, sharing,
and shaping the content we encountered. Thumbs up or down, witty banter
in forum threads, and passionate reblogs -- these became the currencies
of online engagement, forging connections and fostering communities
around shared interests.

And then came the social media giants. Facebook, Twitter, YouTube --
names that once felt exotic but now echo in every corner of the globe.
These platforms, with their infinite scroll and joy-giving algorithms,
connected us not just to friends and family, but to strangers who shared
our passions, our anxieties, our pup videos. Information, news, and
opinions swirled in this digital vortex, reshaping communication on a
global scale.

The social media landscape is forever changing its shape. TikTok's
short-form dances replace Facebook's photo albums, Instagram's curated
aesthetics morph into unfiltered authenticity. This perpetual evolution,
driven by user preferences and technological leaps, keeps us on our
toes, adapting, exploring, and reinventing the ways we connect and share
in the digital realm.

Web 2.0 and the social media revolution have not just changed how we
access information; they have changed how we think, how we interact, how
we see ourselves in the world. This is a story of empowerment, of
connection, of voices amplified and communities born online. In this
digital democracy, we are not just a spectator; we are a citizen, a
creator, a co-author of the ever-unfolding story of the internet.

\textbf{\emph{Cloud computing and big data : democratizing access and
insights}}

Imagine a time when businesses were chained to clunky server rooms,
their dreams shackled by the hefty price tag of on-premise IT
infrastructure. This was the digital landscape before the dawn of cloud
computing, a technological wizardry that would liberate data from its
physical confines and send it soaring into the virtual realm.

In the early 2000s, a transformation echoed through the corridors of
tech giants like Amazon, Microsoft, and Google. They envisioned a world
where powerful servers, storage, databases, and networking capabilities
would not be exclusive to the privileged few. Their answer? Remote
access, delivered on a subscription basis. This radical notion was the
catalyst for cloud computing.

With cloud computing, businesses of all sizes could ditch the shackles
of their server rooms and scale their computing resources. No more hefty
upfront investments, no more cluttering offices with whirring hardware
-- the cloud offered agility, cost-efficiency, and global reach. With
laptops replacing server racks, and internet connections becoming the
new power cables, the playing field leveled, empowering both tech-savvy
startups and established giants to compete with equal might.

However, businesses faced a new frontier: the ocean of big data. From
healthcare records to financial transactions, data gushed forth in
ever-increasing torrents, drowning traditional systems in its relentless
tide. Volume, velocity, and variety -- these were the three challenges
that threatened to paralyze progress.

But fear not, for frameworks like Hadoop and Spark emerged. They
distributed the data across vast networks of cheap, readily available
hardware, turning the very size of the enemy into its own vulnerability.
In-memory processing, the secret weapon of Spark, further accelerated
the analysis, extracting vital insights from the data deluge with
lightning speed.

Advanced analytics emerged, gleaning knowledge from the chaos, guiding
informed decision-making and proactive initiatives. Real-time analysis
became a reality, allowing businesses to adapt to a constantly shifting
landscape with unmatched efficiency. Predictive analytics peered into
the future, mitigating risks and capitalizing on emerging opportunities.
Even personalized experiences, tailored to individual data and
preferences, blossomed in this ground.

But the cloud and big data were not destined to reign in separate
kingdoms. They were destined to intertwine. Cloud platforms became the
grounds for big data analytics to flourish, offering the scalable
infrastructure to handle vast volumes while big data technologies
extracted essential information within those datasets. Through cloud
computing, the following factors were improved:

\begin{itemize}
\tightlist
\item
  Cost-efficiency: Cloud's elastic resources eliminated idle server
  expenses, ensuring efficient data processing.
\item
  Speed: Powerful cloud infrastructure and Spark's in-memory processing
  accelerated big data analysis, yielding quicker results.
\item
  Accessibility: Cloud platforms facilitated seamless collaboration on
  big data projects for geographically dispersed teams.
\item
  Democratization: Cloud and big data tools made advanced analytics
  accessible to organizations of all sizes, leveling the playing field
  once more.
\end{itemize}

This digital revolution is not without its shadows. Concerns over data
privacy and security loom large, demanding vigilant safeguarding of our
digital footprints. The ethical implications of big data, with its
potential for bias and discrimination, must be carefully considered. But
as we navigate these challenges, it is crucial to remember that the
cloud and big data are not forces to be feared, but tools to be wielded
wisely. They hold the key to unlocking a future where knowledge
empowers, decisions enlighten, and innovation soars on the wings of
virtual data.

\textbf{\emph{Mobile computing and IoT (internet of things):
transforming the landscape}}

Remember the chunky brick phones of the past, mere bricks tethered to
our ears? In their dusty wake rose a tech revolution: the proliferation
of mobile devices. Smartphones, sleek technologies in our pockets, and
tablets, digital canvases in our hands, transformed mobile computing
from rudimentary calls to a nearly constant digital embrace. These
powerful processors, flowing with apps and connectivity, have blurred
the lines between personal lives and information systems, forever
reshaping how we access, manage, and experience the world around us.

Mobile devices revolutionized information access. Emails answered on the
bus, breaking news streamed in real-time, bank accounts balanced
mid-coffee break -- these are mere bits of information now readily
available. Location-based services guide us through unfamiliar streets,
social media feeds pulse with the rhythm of the connected world, and
educational resources bloom like digital gardens, all accessible with a
swipe and a tap.

This mobile revolution is not confined to leisure; it is transforming
how we work. Mobile accessibility liberates us from cubicle walls,
empowering remote work and flexible schedules. Collaborative tools dance
across oceans and time zones, facilitating seamless communication and
project management no matter where we are. The office, once a physical
space, now extends as far as our Wi-Fi signal reaches, blurring the
lines between work and personal time, yet offering unprecedented
flexibility and connectivity.

But this digital tapestry is not woven solely from mobile threads. Enter
the Internet of Things (IoT), an assemblage of billions of physical
objects chock-full of sensors and actuators. From smartwatches on our
wrists to connected appliances in our homes and industrial sensors in
factories, these devices whisper a constant stream of data -- a chorus
of information about our world, its rhythms, and its needs.

This IoT data is not a mere digital chatter; it is a powerful current
flowing into information systems. Imagine factories where machines
predict their own maintenance needs, homes that adjust temperature and
lighting based on our preferences, and cities that optimize traffic flow
and energy consumption all in real-time, guided by the whispers of
connected devices.

The possibilities of these devices are staggering:

\begin{itemize}
\tightlist
\item
  \textbf{Real-time monitoring and analysis:} Track performance,
  identify trends, and optimize processes based on live data from
  connected devices.
\item
  \textbf{Predictive maintenance:} Analyze sensor data to anticipate
  equipment failures and schedule preventive maintenance before
  breakdowns occur.
\item
  \textbf{Personalized experiences:} Tailor services and information to
  individual needs and preferences based on collected data. Automated
  decision-making: Implement algorithms that analyze data and trigger
  automated actions based on predefined criteria.
\end{itemize}

However, security and privacy concerns echo loud, demanding robust
measures to protect the vast amounts of personal and sensitive data
generated by the IoT. Data integration across diverse device formats and
protocols can be a complex tango, requiring innovative solutions and
standardization. Additionally, the smooth flow of this digital music
relies on a robust network infrastructure with ample bandwidth to handle
the constant data deluge.

The mobile revolution and the IoT are not isolated phenomena; they are
partners in a grand digital waltz, reshaping our world at every turn.
From accessing information on the go to optimizing industrial processes,
these technologies offer unprecedented power and possibility. But like
any powerful tool, they demand careful consideration of their ethical
implications and responsible development. As we navigate this
exhilarating dance of technology and humanity, let us remember: the goal
is not simply to be connected, but to be connected wisely, ethically,
and for the benefit of all.

\section{Conclusion}\label{conclusion}

The history of information systems has been a fascinating journey that
has shaped the way we gather, store, and analyze data. From the early
days of manual record-keeping to the advent of computer-based systems,
we have witnessed significant advancements in technology that have
revolutionized how we manage information.

The evolution of information systems has had a profound impact on
various industries and sectors. Businesses now have access to real-time
data, enabling them to make more informed decisions and streamline their
operations. The healthcare sector has benefited from electronic health
records, improving patient care and reducing medical errors. Education
has been transformed through online learning platforms and digital
libraries, making knowledge more accessible than ever before.

As we look to the future, it is clear that information systems will
continue to play a crucial role in our lives. The rise of artificial
intelligence and big data analytics promises even greater possibilities
for leveraging information effectively. However, it is important to
remember that technology is only a tool; it is how we harness its power
that truly matters.

Understanding the history of information systems allows us to appreciate
the progress made so far and anticipate exciting developments yet to
come. It reminds us of the importance of adaptability and continuous
learning as we navigate an ever-changing digital landscape. Ultimately,
by embracing new technologies responsibly and ethically, we can harness
their potential for positive change in our personal lives, businesses,
and society as a whole.

\section{Assessment questions}\label{assessment-questions}

\begin{itemize}
\tightlist
\item
  Describe the major inventions and innovations that have shaped the
  evolution of information systems.
\item
  Compare and contrast different eras in the history of information
  systems, highlighting the defining characteristics of each period.
\item
  Analyze the relationship between technological advancements and
  broader social, economic, and political changes.
\end{itemize}

\section{Critical thinking questions}\label{critical-thinking-questions}

\begin{enumerate}
\def\labelenumi{\arabic{enumi}.}
\tightlist
\item
  How have information systems transformed the way we work, learn, and
  communicate? What are the positive and negative consequences of these
  changes?
\item
  What are the emerging trends and challenges in information systems? -
  How can we prepare for a future where technology is even more deeply
  embedded in our lives?
\item
  What are the long-term social and cultural implications of our
  increasingly data-driven world? What are the risks of surveillance,
  manipulation, and loss of agency?
\end{enumerate}

\section{References}\label{references}

\phantomsection\label{refs}
\begin{CSLReferences}{1}{0}
\bibitem[\citeproctext]{ref-abelson1996qc}
Abelson, Harold, and Gerald Jay Sussman. 1996. \emph{Structure and
Interpretation of Computer Programs}. 2nd ed. MIT Electrical Engineering
and Computer Science. London, England: MIT Press.

\bibitem[\citeproctext]{ref-aws_cdn}
Amazon Web Services. n.d. {``What Is a CDN (Content Delivery
Network)?''} n.d. \url{https://aws.amazon.com/what-is/cdn/}.

\bibitem[\citeproctext]{ref-aws_load_balancing}
---------. n.d. {``What Is Load Balancing?''} n.d.
\url{https://aws.amazon.com/what-is/load-balancing/}.

\bibitem[\citeproctext]{ref-amonoo2023stakeholder}
Amonoo Nkrumah, B., W. Qian, A. Kaur, and C. Tilt. 2023. {``Stakeholder
Accountability in the Era of Big Data: An Exploratory Study of Online
Platform Companies.''} \emph{Qualitative Research in Accounting \&
Management} 20 (4): 447--84.
\url{https://doi.org/10.1108/QRAM-03-2022-0042}.

\bibitem[\citeproctext]{ref-annunziato2020experience}
Annunziato, M. 2020. {``The {`Experience Age'} Is Already Here.''} The
Daily Campus. 2020.
\url{https://dailycampus.com/2020/10/09/the-experience-age-is-already-here/}.

\bibitem[\citeproctext]{ref-Awati2021bz}
Awati, Rahul, and Linda Rosencrance. 2021. {``Computer Hardware.''}
\url{https://www.techtarget.com/searchnetworking/definition/hardware};
TechTarget. October 2021.

\bibitem[\citeproctext]{ref-blinova2023corporatesustainability}
Blinova, E., T. Ponomarenko, and S. Tesovskaya. 2023. {``Key Corporate
Sustainability Assessment Methods for Coal Companies.''}
\emph{Sustainability (2071-1050)} 15 (7): 5763.
\url{https://doi.org/10.3390/su15075763}.

\bibitem[\citeproctext]{ref-busuena2023introduction}
Busueña, J., and J. Pomperada. 2023. \emph{Introduction to Information
Technology and Computer Fundamentals}. 2nd ed. Unlimited Books Library
Services; Publishing Inc.

\bibitem[\citeproctext]{ref-clarke2018individual}
Clarke, Blanaid. 2018. {``Individual Accountability in Irish Credit
Institutions---Lessons to Be Learned from the United Kingdom's Senior
Managers' Regime.''} \emph{Common Law World Review}, 35--52.

\bibitem[\citeproctext]{ref-cloudflare_vpc}
Cloudfare. n.d. {``What Is a Virtual Private Cloud (VPC)?''} n.d.
\url{https://www.cloudflare.com/learning/cloud/what-is-a-virtual-private-cloud/}.

\bibitem[\citeproctext]{ref-cocca2022dn}
Cocca, Germán. 2022. {``Programming Paradigms -- Paradigm Examples for
Beginners.''}
\url{https://www.freecodecamp.org/news/an-introduction-to-programming-paradigms/}.
May 2022.

\bibitem[\citeproctext]{ref-djordjevic2019corporate}
Đorđević, D. B., M. Vuković, S. Urošević, N. Štrbac, and A. Vuković.
2019. {``Studying the Corporate Social Responsibility in Apparel and
Textile Industry.''} \emph{Industria Textila} 70 (4): 336--41.
\url{https://doi.org/10.35530/IT.070.04.1572}.

\bibitem[\citeproctext]{ref-englander2021mq}
Englander, Irv, and Wilson Wong. 2021. \emph{The Architecture of
Computer Hardware, Systems Software, and Networking}. 6th ed. Nashville,
TN: John Wiley \& Sons.

\bibitem[\citeproctext]{ref-geeks_for_geeks_iot_challenges}
Geeks, Geeks for. 2023. {``Challenges in Internet of Things (IoT).''}
2023.
\url{https://www.geeksforgeeks.org/challenges-in-internet-of-things-iot/}.

\bibitem[\citeproctext]{ref-gillis2023iot}
Gillis, A. S. 2023. {``Internet of Things (IoT).''} 2023.
\url{https://www.techtarget.com/iotagenda/definition/Internet-of-Things-IoT}.

\bibitem[\citeproctext]{ref-gilster2001pc}
Gilster, Ron. 2001. \emph{{PC} Hardware: A Beginner's Guide}. New York:
Osborne/{McGraw}-Hill. \url{http://site.ebrary.com/id/10015274}.

\bibitem[\citeproctext]{ref-goodman2008information}
Goodman, Seymour, Detmar W. Straub, and Richard Baskerville. 2008.
\emph{Information Security: Policy, Processes, and Practices}.
Routledge.

\bibitem[\citeproctext]{ref-google_cloud_storage}
Google. n.d. {``What Is Cloud Storage?''} n.d.
\url{https://cloud.google.com/learn/what-is-cloud-storage}.

\bibitem[\citeproctext]{ref-hameed2023association}
Hameed, F., M. Alfaraj, and K. Hameed. 2023. {``The Association of Board
Characteristics and Corporate Social Responsibility Disclosure Quality:
Empirical Evidence from Pakistan.''} \emph{Sustainability (2071-1050)}
15 (24): 16849. \url{https://doi.org/10.3390/su152416849}.

\bibitem[\citeproctext]{ref-honigsberg2019individual}
Honigsberg, C. 2019. {``The Case for Individual Audit Partner
Accountability.''} \emph{Vanderbilt Law Review} 72 (6): 1871--1922.

\bibitem[\citeproctext]{ref-ibm_cloud_computing}
IBM. n.d. {``What Is Cloud Computing?''} n.d.
\url{https://www.ibm.com/topics/cloud-computing}.

\bibitem[\citeproctext]{ref-ibm_iot}
---------. n.d. {``What Is the Internet of Things (IoT)?''} n.d.
\url{https://www.ibm.com/topics/internet-of-things}.

\bibitem[\citeproctext]{ref-ivey2023iotapplications}
Ivey, A. 2023. {``7 Real-World IoT Applications and Examples.''} 2023.
\url{https://cointelegraph.com/news/7-iot-applications-and-examples}.

\bibitem[\citeproctext]{ref-jebaraj2023cloudcompanies}
Jebaraj, K. 2023. {``Top 10 Cloud Computing Companies of 2024.''} 2023.
\url{https://www.knowledgehut.com/blog/cloud-computing/top-cloud-computing-companies}.

\bibitem[\citeproctext]{ref-jeursen2022cover}
Jeursen, T. 2022. {``"Cover Your Ass": Individual Accountability, Visual
Documentation, and Everyday Policing in Miami.''} \emph{PoLAR: Political
\& Legal Anthropology Review} 45 (2): 186--200.
\url{https://doi.org/10.1111/plar.12505}.

\bibitem[\citeproctext]{ref-kang2022sustainabletraining}
Kang, Y.-C., H.-S. Hsiao, and J.-Y. Ni. 2022. {``The Role of Sustainable
Training and Reward in Influencing Employee Accountability Perception
and Behavior for Corporate Sustainability.''} \emph{Sustainability
(2071-1050)} 14 (18): 11589--N.PAG.
\url{https://doi.org/10.3390/su141811589}.

\bibitem[\citeproctext]{ref-keutzer1994he}
Keutzer, Kurt. 1994. {``Hardware/Software Co-Simulation.''} In
\emph{Proceedings of the 31st Annual Conference on Design Automation
Conference - {DAC} '94}. New York, New York, USA: ACM Press.

\bibitem[\citeproctext]{ref-lombardi2004information}
Lombardi, Olimpia. 2004. {``What Is Information?''} \emph{Foundations of
Science} 9: 105--34.
\url{https://doi.org/10.1023/B:FODA.0000025034.53313.7c}.

\bibitem[\citeproctext]{ref-lundin2023information}
Lundin, L. L. 2023. \emph{Information Security}. Salem Press
Encyclopedia.

\bibitem[\citeproctext]{ref-marr2023cloudtrends}
Marr, B. 2023. {``The 10 Biggest Cloud Computing Trends in 2024 Everyone
Must Be Ready for Now.''} 2023.
\url{https://www.forbes.com/sites/bernardmarr/2023/10/09/the-10-biggest-cloud-computing-trends-in-2024-everyone-must-be-ready-for-now/?sh=ea1da466d672}.

\bibitem[\citeproctext]{ref-microsoft2023securedata}
Microsoft. 2023. {``11 Best Practices for Securing Data in Cloud
Services.''} 2023.
\url{https://www.microsoft.com/en-us/security/blog/2023/07/05/11-best-practices-for-securing-data-in-cloud-services/}.

\bibitem[\citeproctext]{ref-muts2024iot}
Muts, I. 2024. {``10+ Best IoT Cloud Platforms in 2024.''} 2024.
\url{https://euristiq.com/best-iot-cloud-platforms/}.

\bibitem[\citeproctext]{ref-sebesta2015ek}
Sebesta, Robert W. 2015. \emph{Concepts of Programming Languages}. 11th
ed. Upper Saddle River, NJ: Pearson.

\bibitem[\citeproctext]{ref-stallings2016computer}
Stallings, William. 2016. \emph{Computer Organization and Architecture}.
10th edition. Pearson.

\bibitem[\citeproctext]{ref-sugandhi2023iotfuture}
Sugandhi, A. 2023. {``The Future of IoT: Trends and Predictions for
2024.''} 2023.
\url{https://www.knowledgehut.com/blog/web-development/iot-future}.

\bibitem[\citeproctext]{ref-velazquez2022iot}
Velazquez, R. 2022. {``IoT: The Internet of Things. What Is the Internet
of Things? How Does IoT Work?''} 2022.
\url{https://builtin.com/internet-things}.

\bibitem[\citeproctext]{ref-weiser1999og}
Weiser, Mark. 1999. {``The Computer for the 21 \(^{st}\) Century.''}
\emph{ACM SIGMOBILE Mob. Comput. Commun. Rev.} 3 (3): 3--11.

\bibitem[\citeproctext]{ref-williams2012using}
Williams, B. K., and S. C. Sawyer. 2012. \emph{Using Information
Technology: Introductory Edition}. McGraw-Hill Europe.

\bibitem[\citeproctext]{ref-zhang2023environmental}
Zhang, Y., M. Imeni, and S. A. Edalatpanah. 2023. {``Environmental
Dimension of Corporate Social Responsibility and Earnings Persistence:
An Exploration of the Moderator Roles of Operating Efficiency and
Financing Cost.''} \emph{Sustainability (2071-1050)} 15 (20): 14814.
\url{https://doi.org/10.3390/su152014814}.

\end{CSLReferences}

\section{Keywords}\label{keywords}

computing, data, computers, internet, information systems, machines,
communication, technology, software, web

\bookmarksetup{startatroot}

\chapter{Secure Information Systems}\label{secure-information-systems}

\section{Learning Objectives}\label{learning-objectives-1}

\begin{itemize}
\tightlist
\item
  At the end of this chapter, you should be able to:
\item
  Know what is Information Security;
\item
  Understand the concepts of information security;
\item
  Identify the risks and threats to information security.
\end{itemize}

We are in the Information Age, as historians believe to be. As modern
times shift from industrial production to the computerization of most
services, access to digitized information has become more common. The
Internet is one of the main examples and forces that influenced such
shifts. Upgraded telecommunications and computer sciences also
contributed to the event. These scenarios allowed global businesses,
academic institutions, medical organizations, and government units to
explore more opportunities. (Annunziato 2020) ``We live in the age of
information'' as the well-known sentence goes. Information is
transferred from one place to another through a variety of forms and
contexts. Information is communicated from peer-to-peer interactions up
to business-to-customer transactions. (Lombardi 2004) The details
provided during the communication phase may differ in value and
importance. Some information may require safeguarding more than others.
For instance, business data and customer information are more sensitive
thus demanding security to remain confidential. Let us also not forget
the existence of multiple organizations that aim to utilize this
classified information for their benefit. This threat is just one of a
few reasons why information security is necessary. To further understand
the topic, let us first define the terms ``information'' and
``security''.

While the word ``information'' is often thought of, its definition is
far more complex. Several concepts have been produced to encompass the
majority of its definitions. The concepts of information are as follows:

\begin{itemize}
\tightlist
\item
  \emph{`Information as a representation of knowledge'.} Researchers
  argue that the information-seeking behavior of an individual is a
  reflection of his state of knowledge. In addition, information may be
  stored in multiple mediums whether in the form of physical resources
  such as books or the increasing number of electronic media resources.
\item
  \emph{`Information as data in the environment'.} All living organisms
  convey messages by communicating verbally or physically. While verbal
  communication is common in all species, some use gestures and
  movements to pass messages. These practices are some of the simplest
  ways in which information is transferred.
\item
  \emph{`Information as part of the communication process'.} Messages
  passed on from sender to receiver acquire their meaning according to
  how the receiver interprets the information. Factors are to be
  considered during the delivery of the message as a particular receiver
  may interpret it differently compared to another.
\item
  \emph{`Information as a resource or commodity'.} Upon transmission of
  a message from the sender to the receiver, the message is interpreted
  by the receiver. Some messages are classified according to their state
  of focus. Messages may be highly focused while others may be
  looser(Busueña and Pomperada 2023).
\end{itemize}

Despite all these concepts and definitions associated with
`information', we shall focus on its definition associated with computer
science. As mentioned earlier, information can be communicated through
any form of media. However, in a computer application, information is
treated as the end-product of input data. The raw data input is rendered
by the computer to produce meaningful statements and comparisons(Busueña
and Pomperada 2023).

The term `security' in computer science refers to the protection of
assets -- including information, and keeping them from unauthorized use.
Multiple factors may contribute to the damage of assets. Sources may
either be malicious, accidental, or natural. A part of security in terms
of asset protection is dedicated to business continuity planning. During
the instances of information losses, an organization shall also possess
the ability to produce and retrieve information in return (Goodman,
Straub, and Baskerville 2008). As we progress in conducting all our
negotiations online, our information becomes open for public use. With
that in consideration, we sacrifice the security and privacy of our
data. The process of preventing information from being used, disclosed,
accessed, altered, or destroyed without authorization is known as
Information Security or InfoSec. Information security has two focal
areas -- information assurance and information technology security.
Information assurance is the ability to guarantee that data is not lost
as a result of theft, natural disasters, or technological failures, or
because of a breach in system security. Information technology security,
on the other hand, is focused on security for computer networks(Lundin
2023).

\section{CONCEPTS OF INFORMATION
SECURITY}\label{concepts-of-information-security}

Six (6) key concepts are considered in producing a good information
security system. These concepts are as follows (Lundin 2023):

\begin{itemize}
\tightlist
\item
  Confidentiality: Protecting sensitive information from unauthorized
  access. This involves measures such as encryption and access controls.
\item
  Integrity: Ensuring the accuracy and reliability of data by preventing
  unauthorized modifications. Techniques include data validation and
  checksums.
\item
  Availability: Ensuring that information and resources are accessible
  when needed. This involves strategies like redundancy, backups, and
  disaster recovery planning.
\item
  Authentication: Verifying the identity of users or systems. Common
  methods include passwords, biometrics, and multi-factor
  authentication.
\item
  Authorization: Granting appropriate permissions to users or systems
  based on their authenticated identity. Access controls and role-based
  access are crucial components.
\item
  Non-Repudiation: Ensuring that a user cannot deny their actions. This
  is often achieved through the use of digital signatures and audit
  trails.
\end{itemize}

\section{OVERVIEW OF THE THREAT
LANDSCAPE}\label{overview-of-the-threat-landscape}

Advanced communications and the use of digital devices have led to an
increase in many risks and threats. Security issues are a never-ending
concern since computer and data attacks continue to grow stronger and
more intricate (Busueña and Pomperada 2023). The following are aspects
of security to note:

\subsection{Cyberthreats}\label{cyberthreats}

Cyberthreats refer to potential harm or danger in the digital realm,
specifically targeting computer systems, networks, and the information
stored or transmitted through them. These threats exploit
vulnerabilities in technology, software, and human behavior to
compromise the confidentiality, integrity, and availability of data and
systems. Cyberthreats can take various forms, and they are constantly
evolving as technology advances. Here are some common types of
cyberthreats: * Malware: Malicious software designed to disrupt, damage,
or gain unauthorized access to computer systems. Examples include
viruses, worms, Trojans, spyware, and ransomware. * Phishing: Deceptive
attempts to trick individuals into revealing sensitive information, such
as passwords or financial details, by posing as a trustworthy entity.
This is often done through fake emails, websites, or messages. * Denial
of Service (DoS) and Distributed Denial of Service (DDoS) Attacks:
Attempts to overwhelm a network, service, or website with excessive
traffic, making it unavailable to legitimate users. * Social
Engineering: Manipulating people into divulging confidential information
or performing actions that may compromise security. This can involve
psychological manipulation, impersonation, or exploiting trust. *
Insider Threats: Risks posed by individuals within an organization who
misuse their access or privileges, either intentionally or
unintentionally, leading to security breaches. * Advanced Persistent
Threats (APTs): Coordinated and sophisticated attacks by well-funded and
skilled adversaries. APTs often involve long-term, targeted efforts to
infiltrate specific organizations or networks. * Ransomware: Malware
that encrypts files or entire systems, demanding payment (usually in
cryptocurrency) for the release of the data. Ransomware attacks can
result in data loss, financial losses, and operational disruptions. *
Zero-Day Exploits: Attacks that target software vulnerabilities unknown
to the software vendor. Cybercriminals exploit these vulnerabilities
before a patch or fix is developed and released. * Man-in-the-Middle
(MitM) Attacks: Interception and manipulation of communication between
two parties without their knowledge. This can lead to eavesdropping,
data alteration, or unauthorized access. * IoT (Internet of Things)
Vulnerabilities: Exploiting security weaknesses in connected devices,
such as smart appliances, cameras, and industrial systems, to gain
unauthorized access or control.

Understanding and mitigating these cyberthreats is crucial for
individuals, organizations, and governments to maintain the security and
privacy of digital assets. This involves implementing robust
cybersecurity measures, staying informed about emerging threats, and
promoting a culture of security awareness.

\subsection{Perpetrators of
Cybercrime}\label{perpetrators-of-cybercrime}

Cybercrime is committed by a diverse range of individuals and groups
with varying motives. Perpetrators of cybercrime can include:

\begin{itemize}
\tightlist
\item
  Hackers: Individuals or groups with advanced technical skills who
  exploit vulnerabilities in computer systems or networks to gain
  unauthorized access, steal information, or disrupt operations. Some
  hackers may act for financial gain, while others do it for ideological
  reasons or simply to demonstrate their skills.
\item
  Criminal Organizations: Organized crime groups engage in cybercriminal
  activities to generate revenue. These activities may include identity
  theft, credit card fraud, ransomware attacks, and other financially
  motivated crimes.
\item
  Nation-States: Governments or state-sponsored entities may engage in
  cyber-espionage, cyber-attacks, or information warfare for political,
  economic, or military purposes. Nation-state cyber-actors often have
  significant resources and advanced capabilities.
\item
  Insiders: Individuals within an organization who misuse their access
  or privileges for malicious purposes. Insider threats can result from
  disgruntled employees, negligence, or unintentional actions that
  compromise security.
\item
  Hacktivists: Activists or politically motivated individuals who use
  hacking as a means to promote their social or political agenda. Their
  activities may include defacing websites, disrupting services, or
  stealing and exposing sensitive information.
\item
  Script Kiddies: Individuals with limited technical skills who use
  pre-written scripts or tools to conduct cyber-attacks. While less
  sophisticated than other cybercriminals, script kiddies can still
  cause harm by exploiting known vulnerabilities.
\item
  Cyber Extortionists: Individuals or groups that engage in extortion by
  threatening to release sensitive information, launch a cyber-attack,
  or disrupt services unless a ransom is paid. Ransomware attacks fall
  under this category.
\item
  Phishers: Individuals who engage in phishing attacks to trick people
  into divulging sensitive information, such as passwords or financial
  details. Phishing can be conducted through emails, fake websites, or
  social engineering tactics.
\item
  Cyber Spies: Individuals or groups involved in corporate or industrial
  espionage. They may target organizations to steal proprietary
  information, trade secrets, or intellectual property for economic
  advantage.
\item
  Unethical Employees: Employees who act against the interests of their
  organization, either due to personal grievances, financial incentives,
  or coercion. It's important to note that the motivations behind
  cybercrime can vary widely, ranging from financial gain and political
  objectives to personal vendettas or ideological beliefs. As the
  digital landscape evolves, new types of cybercriminals and threats
  continue to emerge, requiring ongoing efforts to strengthen
  cybersecurity measures and international collaboration to combat
  cybercrime effectively (Williams and Sawyer 2012).
\end{itemize}

\subsection{ASSESSMENT QUESTIONS}\label{assessment-questions-1}

\begin{enumerate}
\def\labelenumi{\arabic{enumi}.}
\tightlist
\item
  Identify and explain three prominent cyber threats, detailing the
  potential impact on businesses and the strategies that organizations
  can employ to mitigate these threats.
\item
  Determine the roles of employees in the awareness and education on
  enhancing overall cybersecurity resilience.
\end{enumerate}

\subsection{CRITICAL THINKING
QUESTION}\label{critical-thinking-question}

\begin{itemize}
\tightlist
\item
  Critically evaluate the role of artificial intelligence (AI) in
  enhancing and challenging cybersecurity measures. Discuss the
  potential benefits and risks associated with the use of AI in
  cybersecurity, and analyze how organizations can strike a balance
  between leveraging AI for defense and addressing the ethical
  considerations and potential vulnerabilities that may arise.
\end{itemize}

\bookmarksetup{startatroot}

\chapter{Data Security}\label{data-security}

\section{Learning Objectives:}\label{learning-objectives-2}

\emph{At the end of this chapter, you should be able to:} * To know the
importance and benefits of data security * To determine the best
practices for ensuring data security and training * To get familiar with
different types of data security technology and when to use them

\subsection{Introduction}\label{introduction-1}

\begin{itemize}
\tightlist
\item
  Data Security -- Definition, Importance, and Benefits
\item
  Data security, or information security, is a critical aspect of
  information technology. It is generally defined as: the practice of
  protecting information from unauthorized use, disclosure, access,
  modification, or destruction. This term is applied to all information
  regardless of the form it takes and is comprised of two major
  categories: information assurance, which is ability to ensure data is
  not lost to a breakdown in system security, due to theft, natural
  disasters, or technological malfunction; and IT (information
  technology) security, which is the security applied to computer
  networks. (Lundin, 2023)
\end{itemize}

Data security is of paramount importance in today's digital age due to
the increasing volume and value of data generated and stored by
individuals, organizations, and governments. The importance of data
security lies in its role in safeguarding sensitive information from
unauthorized access, ensuring the integrity of data, and protecting
against potential threats. Here are key reasons why data security is
crucial and the associated benefits (OpenAI, 2024):

\begin{enumerate}
\def\labelenumi{\arabic{enumi}.}
\tightlist
\item
  Protection of Sensitive Information
\end{enumerate}

Data security safeguards sensitive and confidential information, such as
personal data, financial records, intellectual property, and trade
secrets, from falling into the wrong hands. This is critical for
preserving the privacy and trust of individuals and maintaining the
competitive advantage of businesses.

\begin{enumerate}
\def\labelenumi{\arabic{enumi}.}
\setcounter{enumi}{1}
\tightlist
\item
  Prevention of Unauthorized Access
\end{enumerate}

Unauthorized access to data can lead to identity theft, fraud, and
unauthorized use of resources. Implementing robust authentication and
access control measures helps prevent unauthorized individuals from
gaining access to sensitive information.

\begin{enumerate}
\def\labelenumi{\arabic{enumi}.}
\setcounter{enumi}{2}
\tightlist
\item
  Maintaining Data Integrity
\end{enumerate}

Data integrity ensures that information remains accurate and unaltered.
By implementing measures such as encryption, checksums, and access
controls, data security helps prevent unauthorized modifications,
corruption, or tampering of critical data.

\begin{enumerate}
\def\labelenumi{\arabic{enumi}.}
\setcounter{enumi}{3}
\tightlist
\item
  Prevention of Data Loss and Data Breaches
\end{enumerate}

Data breaches can result in significant financial losses, reputational
damage, and legal consequences. Data security measures, including
encryption, firewalls, and intrusion detection systems, help prevent and
mitigate the impact of data breaches.

\begin{enumerate}
\def\labelenumi{\arabic{enumi}.}
\setcounter{enumi}{4}
\tightlist
\item
  Compliance with Regulations
\end{enumerate}

Many industries and regions have specific regulations and laws governing
the handling and protection of sensitive data. Adhering to these
regulations is not only a legal requirement but also essential for
maintaining the trust of customers and partners.

\begin{enumerate}
\def\labelenumi{\arabic{enumi}.}
\setcounter{enumi}{5}
\tightlist
\item
  Preservation of Customer Trust
\end{enumerate}

Customers and clients are more likely to trust organizations that
demonstrate a commitment to protecting their sensitive information. A
strong data security posture contributes to building and maintaining
trust, which is crucial for customer loyalty and satisfaction.

\begin{enumerate}
\def\labelenumi{\arabic{enumi}.}
\setcounter{enumi}{6}
\tightlist
\item
  Business Continuity
\end{enumerate}

Data security is integral to business continuity. By implementing
measures such as regular backups, disaster recovery plans, and
redundancy, organizations can ensure that they can recover and resume
operations quickly in the event of data loss or system failure.

\begin{enumerate}
\def\labelenumi{\arabic{enumi}.}
\setcounter{enumi}{7}
\tightlist
\item
  Prevention of Cyber Threats
\end{enumerate}

Cyber threats, including malware, ransomware, and phishing attacks, are
constantly evolving. Data security technologies such as antivirus
software, firewalls, and intrusion prevention systems help protect
against these threats and minimize the risk of security incidents.

\begin{enumerate}
\def\labelenumi{\arabic{enumi}.}
\setcounter{enumi}{8}
\tightlist
\item
  Competitive Advantage
\end{enumerate}

Organizations that prioritize data security gain a competitive advantage
by demonstrating a commitment to protecting customer information and
intellectual property. This can be a differentiator in the marketplace
and contribute to business success.

\begin{enumerate}
\def\labelenumi{\arabic{enumi}.}
\setcounter{enumi}{9}
\tightlist
\item
  Employee Productivity and Trust
\end{enumerate}

Employees are more productive when they feel confident that their
work-related data is secure. Additionally, data security measures
contribute to a sense of trust and confidence among employees, fostering
a positive work environment.

\begin{enumerate}
\def\labelenumi{\arabic{enumi}.}
\setcounter{enumi}{10}
\tightlist
\item
  Protection of Intellectual Property
\end{enumerate}

For businesses, intellectual property is a valuable asset. Data security
helps protect intellectual property by preventing unauthorized access,
copying, or theft of proprietary information, innovations, and trade
secrets.

\begin{enumerate}
\def\labelenumi{\arabic{enumi}.}
\setcounter{enumi}{11}
\tightlist
\item
  Reduction of Legal and Financial Risks
\end{enumerate}

Implementing strong data security measures reduces the risk of legal
action and financial losses associated with data breaches, regulatory
non-compliance, and the mishandling of sensitive information. In
summary, data security is not only a necessity for safeguarding
sensitive information but also a strategic imperative for organizations
in the modern digital landscape. The benefits extend beyond compliance
and risk mitigation to include customer trust, business continuity, and
a competitive edge in the marketplace. Organizations that prioritize and
invest in data security are better positioned to thrive in an
environment where data is a critical asset. \#\#\# Best Practices for
Data Security and Training By implementing these best practices,
organizations can create a more robust and comprehensive approach to
data security, mitigating risks and protecting sensitive information
from unauthorized access or malicious activities. Here are some of the
best practices from de Groot (2022) and Marr (2023):

\begin{enumerate}
\def\labelenumi{\arabic{enumi}.}
\tightlist
\item
  Data Encryption
\end{enumerate}

Encrypt sensitive data both in transit and at rest to protect it from
unauthorized access. Use strong encryption algorithms to secure data
during communication and storage.

\begin{enumerate}
\def\labelenumi{\arabic{enumi}.}
\setcounter{enumi}{1}
\tightlist
\item
  Access Control
\end{enumerate}

Implement strict access controls to ensure that only authorized
personnel can access sensitive data. Use role-based access control
(RBAC) to assign permissions based on job roles and responsibilities.

\begin{enumerate}
\def\labelenumi{\arabic{enumi}.}
\setcounter{enumi}{2}
\tightlist
\item
  Regular Audits and Monitoring
\end{enumerate}

Conduct regular audits to review access logs, user activities, and
system configurations. Implement real-time monitoring to detect and
respond to any suspicious activities promptly.

\begin{enumerate}
\def\labelenumi{\arabic{enumi}.}
\setcounter{enumi}{3}
\tightlist
\item
  Employee Training and Awareness
\end{enumerate}

Educate employees about the importance of data security and train them
on best practices. Foster a security-aware culture to reduce the risk of
human error and improve overall security posture.

\begin{enumerate}
\def\labelenumi{\arabic{enumi}.}
\setcounter{enumi}{4}
\tightlist
\item
  Data Classification
\end{enumerate}

Classify data based on its sensitivity and importance. Apply different
security controls based on the classification level, ensuring that
highly sensitive information receives the highest level of protection.
6. Regular Software Updates and Patch Management

Keep all software, including operating systems, applications, and
security tools, up-to-date with the latest patches. Regularly apply
security updates to address vulnerabilities and enhance overall system
security.

\begin{enumerate}
\def\labelenumi{\arabic{enumi}.}
\setcounter{enumi}{7}
\tightlist
\item
  Endpoint Security
\end{enumerate}

Secure endpoints such as computers, mobile devices, and servers. Use
antivirus software, firewalls, and other security measures to protect
against malware and unauthorized access.

\begin{enumerate}
\def\labelenumi{\arabic{enumi}.}
\setcounter{enumi}{8}
\tightlist
\item
  Secure Password Policies
\end{enumerate}

Enforce strong password policies, including the use of complex
passwords, regular password changes, and multi-factor authentication
(MFA) to add an extra layer of security.

\begin{enumerate}
\def\labelenumi{\arabic{enumi}.}
\setcounter{enumi}{9}
\tightlist
\item
  Incident Response Plan
\end{enumerate}

Develop and regularly test an incident response plan to ensure a rapid
and effective response to security incidents. This plan should include
steps for identifying, containing, eradicating, recovering from, and
learning from security incidents.

\begin{enumerate}
\def\labelenumi{\arabic{enumi}.}
\setcounter{enumi}{10}
\item
  Secure Data BackupsRegularly back up critical data and ensure that
  backups are securely stored. Test data restoration processes to ensure
  the ability to recover data in case of a cyber attack, data
  corruption, or other disasters.
\item
  Vendor Security Assessments
\end{enumerate}

If third-party vendors handle your data, assess their security
practices. Ensure that they adhere to security standards and
regulations, and that they have robust security measures in place.

\begin{enumerate}
\def\labelenumi{\arabic{enumi}.}
\setcounter{enumi}{12}
\tightlist
\item
  Compliance with Data Protection Regulations
\end{enumerate}

Stay informed about and comply with relevant data protection
regulations, such as GDPR, HIPAA, or others applicable to your industry.
This includes obtaining explicit consent for data processing and
respecting individuals' privacy rights. \#\#\# Types of Data Security
There are various types of security measures designed to safeguard data.
They are often implemented in combination to create a multi-layered and
comprehensive defense against potential threats and vulnerabilities.
Here are some examples from Johnson (2023) and Fortinet (2023) : 1.
Encryption

Encryption is the process of converting information or data into a code
to prevent unauthorized access. This transformation is done using an
algorithm and a key, where the algorithm serves as a set of instructions
for the encryption and decryption processes, and the key is a piece of
information that controls the transformation. The encrypted data can
only be decrypted and understood by someone who possesses the correct
key.

The primary purpose of encryption is to ensure the confidentiality and
security of sensitive information, such as personal data, financial
transactions, and communications. It is widely used in various
applications, including secure communication over the internet,
protection of sensitive files and data storage, and safeguarding
information during electronic transactions. Encryption plays a crucial
role in maintaining the privacy and integrity of digital information in
an era where cyber threats and unauthorized access are prevalent.

\begin{enumerate}
\def\labelenumi{\arabic{enumi}.}
\setcounter{enumi}{1}
\tightlist
\item
  Data Erasure
\end{enumerate}

Data erasure, also known as data wiping or data destruction, is the
process of securely and permanently removing all data from a storage
device, such as a hard drive, solid-state drive (SSD), or other media,
to ensure that the information cannot be recovered or accessed by
unauthorized individuals. The goal of data erasure is to render the data
on the storage device unrecoverable, even with advanced data recovery
techniques.

The process typically involves overwriting the existing data on the
storage device with random patterns of binary code, making it
challenging or impossible to reconstruct the original information. Data
erasure is an essential security measure, especially when disposing of
or repurposing storage devices, to prevent sensitive or confidential
information from falling into the wrong hands.

Organizations and individuals often use specialized software or hardware
solutions designed for data erasure to ensure thorough and effective
removal of data. It is important to note that simply deleting files or
formatting a storage device may not provide sufficient protection
against data recovery efforts, as these methods often leave traces that
could be exploited by determined individuals with the right tools and
knowledge. 3. Data Masking

Data masking, also known as data obfuscation or data anonymization, is a
technique used to protect sensitive information by replacing,
encrypting, or scrambling original data with fictional or pseudonymous
data. The purpose of data masking is to create a version of the dataset
that looks and feels like the real data but does not expose sensitive
information. This is particularly important when organizations need to
share or use data for testing, development, or analytical purposes while
ensuring that confidential information remains private and secure.

Key aspects of data masking include: - Substitution: Sensitive data,
such as names, addresses, or social security numbers, is replaced with
fictional or randomly generated information. For example, real names
might be replaced with placeholder names. - Encryption: Sensitive data
can be encrypted, making it unreadable without the appropriate
decryption key. This allows the data to be used for certain purposes
while still maintaining confidentiality. - Shuffling or Perturbation:
The order or relationships between data elements may be changed to make
it more challenging to identify individuals or specific records. Data
masking helps organizations comply with privacy regulations, such as the
General Data Protection Regulation (GDPR) or the Health Insurance
Portability and Accountability Act (HIPAA), by ensuring that personally
identifiable information (PII) or other sensitive data is adequately
protected even in non-production environments. It allows for the use of
realistic data for testing and analysis without exposing individuals to
privacy risks.

\begin{enumerate}
\def\labelenumi{\arabic{enumi}.}
\setcounter{enumi}{3}
\tightlist
\item
  Data Resiliency
\end{enumerate}

Data resiliency refers to the ability of a system or organization to
ensure the availability, integrity, and recoverability of its data in
the face of various challenges, including hardware failures, software
errors, accidental deletions, cyberattacks, natural disasters, or other
disruptive events. The goal of data resiliency is to minimize the impact
of disruptions on data and maintain business continuity.

Key components of data resiliency include: - Data Availability: Ensuring
that data is consistently accessible and usable, even in the event of
hardware failures, network issues, or other disruptions - Data
Integrity: Guaranteeing the accuracy and reliability of data. Data
integrity measures prevent unauthorized or accidental changes to data,
ensuring that it remains accurate and trustworthy. - Data Recovery:
Implementing mechanisms to recover data quickly and efficiently in the
event of data loss or corruption. This may involve regular data backups,
snapshots, or other recovery methods. - Redundancy: Creating redundant
copies of critical data in geographically dispersed locations or on
different storage systems to mitigate the risk of data loss due to
hardware failures or disasters. - Security Measures: Implementing robust
security measures to protect data from unauthorized access,
cyberattacks, and other security threats. - Regular Testing and
Validation: Conducting regular testing and validation of data backup and
recovery procedures to ensure their effectiveness. This includes
simulating disaster scenarios to assess the organization's ability to
recover data.

Data resiliency is a critical aspect of overall business resilience. It
involves a combination of technological solutions, policies, and
procedures to safeguard data and ensure that organizations can continue
their operations even in the face of unexpected challenges. This is
particularly important in today's digital landscape, where organizations
rely heavily on data for their day-to-day operations.

\section{Assessment Questions:}\label{assessment-questions-2}

\begin{enumerate}
\def\labelenumi{\arabic{enumi}.}
\tightlist
\item
  What is the process of transforming data into an unreadable form to
  anyone who does not know the key?

  \begin{enumerate}
  \def\labelenumii{\alph{enumii}.}
  \tightlist
  \item
    Data security
  \item
    Data Encryption
  \item
    Data classification
  \item
    Data masking
  \end{enumerate}
\item
  Which of the following is a characteristic of data security?

  \begin{enumerate}
  \def\labelenumii{\alph{enumii}.}
  \tightlist
  \item
    Anticipates problems
  \item
    Protects computers
  \item
    Protects information
  \item
    All of the above \#\# Critical Thinking Questions:
  \end{enumerate}
\item
  If you were a hacker and your prospective victim is your company, what
  would be the first thing that you do? Why?
\item
  If there is one best way to secure data, what would that be? Why?
\end{enumerate}

\section{References}\label{references-1}

\phantomsection\label{refs}
\begin{CSLReferences}{1}{0}
\bibitem[\citeproctext]{ref-abelson1996qc}
Abelson, Harold, and Gerald Jay Sussman. 1996. \emph{Structure and
Interpretation of Computer Programs}. 2nd ed. MIT Electrical Engineering
and Computer Science. London, England: MIT Press.

\bibitem[\citeproctext]{ref-aws_cdn}
Amazon Web Services. n.d. {``What Is a CDN (Content Delivery
Network)?''} n.d. \url{https://aws.amazon.com/what-is/cdn/}.

\bibitem[\citeproctext]{ref-aws_load_balancing}
---------. n.d. {``What Is Load Balancing?''} n.d.
\url{https://aws.amazon.com/what-is/load-balancing/}.

\bibitem[\citeproctext]{ref-amonoo2023stakeholder}
Amonoo Nkrumah, B., W. Qian, A. Kaur, and C. Tilt. 2023. {``Stakeholder
Accountability in the Era of Big Data: An Exploratory Study of Online
Platform Companies.''} \emph{Qualitative Research in Accounting \&
Management} 20 (4): 447--84.
\url{https://doi.org/10.1108/QRAM-03-2022-0042}.

\bibitem[\citeproctext]{ref-annunziato2020experience}
Annunziato, M. 2020. {``The {`Experience Age'} Is Already Here.''} The
Daily Campus. 2020.
\url{https://dailycampus.com/2020/10/09/the-experience-age-is-already-here/}.

\bibitem[\citeproctext]{ref-Awati2021bz}
Awati, Rahul, and Linda Rosencrance. 2021. {``Computer Hardware.''}
\url{https://www.techtarget.com/searchnetworking/definition/hardware};
TechTarget. October 2021.

\bibitem[\citeproctext]{ref-blinova2023corporatesustainability}
Blinova, E., T. Ponomarenko, and S. Tesovskaya. 2023. {``Key Corporate
Sustainability Assessment Methods for Coal Companies.''}
\emph{Sustainability (2071-1050)} 15 (7): 5763.
\url{https://doi.org/10.3390/su15075763}.

\bibitem[\citeproctext]{ref-busuena2023introduction}
Busueña, J., and J. Pomperada. 2023. \emph{Introduction to Information
Technology and Computer Fundamentals}. 2nd ed. Unlimited Books Library
Services; Publishing Inc.

\bibitem[\citeproctext]{ref-clarke2018individual}
Clarke, Blanaid. 2018. {``Individual Accountability in Irish Credit
Institutions---Lessons to Be Learned from the United Kingdom's Senior
Managers' Regime.''} \emph{Common Law World Review}, 35--52.

\bibitem[\citeproctext]{ref-cloudflare_vpc}
Cloudfare. n.d. {``What Is a Virtual Private Cloud (VPC)?''} n.d.
\url{https://www.cloudflare.com/learning/cloud/what-is-a-virtual-private-cloud/}.

\bibitem[\citeproctext]{ref-cocca2022dn}
Cocca, Germán. 2022. {``Programming Paradigms -- Paradigm Examples for
Beginners.''}
\url{https://www.freecodecamp.org/news/an-introduction-to-programming-paradigms/}.
May 2022.

\bibitem[\citeproctext]{ref-djordjevic2019corporate}
Đorđević, D. B., M. Vuković, S. Urošević, N. Štrbac, and A. Vuković.
2019. {``Studying the Corporate Social Responsibility in Apparel and
Textile Industry.''} \emph{Industria Textila} 70 (4): 336--41.
\url{https://doi.org/10.35530/IT.070.04.1572}.

\bibitem[\citeproctext]{ref-englander2021mq}
Englander, Irv, and Wilson Wong. 2021. \emph{The Architecture of
Computer Hardware, Systems Software, and Networking}. 6th ed. Nashville,
TN: John Wiley \& Sons.

\bibitem[\citeproctext]{ref-geeks_for_geeks_iot_challenges}
Geeks, Geeks for. 2023. {``Challenges in Internet of Things (IoT).''}
2023.
\url{https://www.geeksforgeeks.org/challenges-in-internet-of-things-iot/}.

\bibitem[\citeproctext]{ref-gillis2023iot}
Gillis, A. S. 2023. {``Internet of Things (IoT).''} 2023.
\url{https://www.techtarget.com/iotagenda/definition/Internet-of-Things-IoT}.

\bibitem[\citeproctext]{ref-gilster2001pc}
Gilster, Ron. 2001. \emph{{PC} Hardware: A Beginner's Guide}. New York:
Osborne/{McGraw}-Hill. \url{http://site.ebrary.com/id/10015274}.

\bibitem[\citeproctext]{ref-goodman2008information}
Goodman, Seymour, Detmar W. Straub, and Richard Baskerville. 2008.
\emph{Information Security: Policy, Processes, and Practices}.
Routledge.

\bibitem[\citeproctext]{ref-google_cloud_storage}
Google. n.d. {``What Is Cloud Storage?''} n.d.
\url{https://cloud.google.com/learn/what-is-cloud-storage}.

\bibitem[\citeproctext]{ref-hameed2023association}
Hameed, F., M. Alfaraj, and K. Hameed. 2023. {``The Association of Board
Characteristics and Corporate Social Responsibility Disclosure Quality:
Empirical Evidence from Pakistan.''} \emph{Sustainability (2071-1050)}
15 (24): 16849. \url{https://doi.org/10.3390/su152416849}.

\bibitem[\citeproctext]{ref-honigsberg2019individual}
Honigsberg, C. 2019. {``The Case for Individual Audit Partner
Accountability.''} \emph{Vanderbilt Law Review} 72 (6): 1871--1922.

\bibitem[\citeproctext]{ref-ibm_cloud_computing}
IBM. n.d. {``What Is Cloud Computing?''} n.d.
\url{https://www.ibm.com/topics/cloud-computing}.

\bibitem[\citeproctext]{ref-ibm_iot}
---------. n.d. {``What Is the Internet of Things (IoT)?''} n.d.
\url{https://www.ibm.com/topics/internet-of-things}.

\bibitem[\citeproctext]{ref-ivey2023iotapplications}
Ivey, A. 2023. {``7 Real-World IoT Applications and Examples.''} 2023.
\url{https://cointelegraph.com/news/7-iot-applications-and-examples}.

\bibitem[\citeproctext]{ref-jebaraj2023cloudcompanies}
Jebaraj, K. 2023. {``Top 10 Cloud Computing Companies of 2024.''} 2023.
\url{https://www.knowledgehut.com/blog/cloud-computing/top-cloud-computing-companies}.

\bibitem[\citeproctext]{ref-jeursen2022cover}
Jeursen, T. 2022. {``"Cover Your Ass": Individual Accountability, Visual
Documentation, and Everyday Policing in Miami.''} \emph{PoLAR: Political
\& Legal Anthropology Review} 45 (2): 186--200.
\url{https://doi.org/10.1111/plar.12505}.

\bibitem[\citeproctext]{ref-kang2022sustainabletraining}
Kang, Y.-C., H.-S. Hsiao, and J.-Y. Ni. 2022. {``The Role of Sustainable
Training and Reward in Influencing Employee Accountability Perception
and Behavior for Corporate Sustainability.''} \emph{Sustainability
(2071-1050)} 14 (18): 11589--N.PAG.
\url{https://doi.org/10.3390/su141811589}.

\bibitem[\citeproctext]{ref-keutzer1994he}
Keutzer, Kurt. 1994. {``Hardware/Software Co-Simulation.''} In
\emph{Proceedings of the 31st Annual Conference on Design Automation
Conference - {DAC} '94}. New York, New York, USA: ACM Press.

\bibitem[\citeproctext]{ref-lombardi2004information}
Lombardi, Olimpia. 2004. {``What Is Information?''} \emph{Foundations of
Science} 9: 105--34.
\url{https://doi.org/10.1023/B:FODA.0000025034.53313.7c}.

\bibitem[\citeproctext]{ref-lundin2023information}
Lundin, L. L. 2023. \emph{Information Security}. Salem Press
Encyclopedia.

\bibitem[\citeproctext]{ref-marr2023cloudtrends}
Marr, B. 2023. {``The 10 Biggest Cloud Computing Trends in 2024 Everyone
Must Be Ready for Now.''} 2023.
\url{https://www.forbes.com/sites/bernardmarr/2023/10/09/the-10-biggest-cloud-computing-trends-in-2024-everyone-must-be-ready-for-now/?sh=ea1da466d672}.

\bibitem[\citeproctext]{ref-microsoft2023securedata}
Microsoft. 2023. {``11 Best Practices for Securing Data in Cloud
Services.''} 2023.
\url{https://www.microsoft.com/en-us/security/blog/2023/07/05/11-best-practices-for-securing-data-in-cloud-services/}.

\bibitem[\citeproctext]{ref-muts2024iot}
Muts, I. 2024. {``10+ Best IoT Cloud Platforms in 2024.''} 2024.
\url{https://euristiq.com/best-iot-cloud-platforms/}.

\bibitem[\citeproctext]{ref-sebesta2015ek}
Sebesta, Robert W. 2015. \emph{Concepts of Programming Languages}. 11th
ed. Upper Saddle River, NJ: Pearson.

\bibitem[\citeproctext]{ref-stallings2016computer}
Stallings, William. 2016. \emph{Computer Organization and Architecture}.
10th edition. Pearson.

\bibitem[\citeproctext]{ref-sugandhi2023iotfuture}
Sugandhi, A. 2023. {``The Future of IoT: Trends and Predictions for
2024.''} 2023.
\url{https://www.knowledgehut.com/blog/web-development/iot-future}.

\bibitem[\citeproctext]{ref-velazquez2022iot}
Velazquez, R. 2022. {``IoT: The Internet of Things. What Is the Internet
of Things? How Does IoT Work?''} 2022.
\url{https://builtin.com/internet-things}.

\bibitem[\citeproctext]{ref-weiser1999og}
Weiser, Mark. 1999. {``The Computer for the 21 \(^{st}\) Century.''}
\emph{ACM SIGMOBILE Mob. Comput. Commun. Rev.} 3 (3): 3--11.

\bibitem[\citeproctext]{ref-williams2012using}
Williams, B. K., and S. C. Sawyer. 2012. \emph{Using Information
Technology: Introductory Edition}. McGraw-Hill Europe.

\bibitem[\citeproctext]{ref-zhang2023environmental}
Zhang, Y., M. Imeni, and S. A. Edalatpanah. 2023. {``Environmental
Dimension of Corporate Social Responsibility and Earnings Persistence:
An Exploration of the Moderator Roles of Operating Efficiency and
Financing Cost.''} \emph{Sustainability (2071-1050)} 15 (20): 14814.
\url{https://doi.org/10.3390/su152014814}.

\end{CSLReferences}

\bookmarksetup{startatroot}

\chapter{Corporate and Individual Accountability: Ethical, Legal, and
Social
Issues}\label{corporate-and-individual-accountability-ethical-legal-and-social-issues}

\section{Understanding
Accountability}\label{understanding-accountability}

\subsection{Introduction}\label{introduction-2}

Alright, buckle up, you aspiring corporate superheroes! In this chapter,
we're going to demystify the intricate world of accountability -- the
superhero cape of the business realm. Get ready for a rollercoaster ride
through definitions, historical escapades, contemporary twists, and the
many dimensions of this superhero concept.

\subsubsection{Definition and
Conceptualization}\label{definition-and-conceptualization}

Let's kick things off with the fundamental question: What the heck is
accountability? Imagine you're the superhero of your own story, and
accountability is your iconic cape. It's the commitment to owning up to
your actions, decisions, and, let's face it, the occasional office
shenanigans. In the corporate universe, accountability is the unspoken
agreement to be the grown-up in the room. It's not blaming your
coworkers for the missing stapler, and it's definitely not hiding behind
the ``Reply All'' button in the email mishap of 2022. It's about being
transparent, reliable, and not passing the blame like a hot potato. Now,
let's delve into the conceptualization -- it's like building a superhero
headquarters. You need blueprints, a vision, and a cool theme song. In
corporate lingo, a clear conceptualization of accountability is crucial
for effective governance. It's the roadmap that tells everyone, ``Here's
how we roll, and here's who's in charge.'' Think of it this way: you
wouldn't embark on a quest to find the Holy Grail without a map, right?
The same goes for navigating the corporate landscape. Accountability
provides the GPS coordinates for ethical decision-making, transparent
conduct, and a smooth sail through the stormy seas of corporate
responsibility.

\subsubsection{Historical Evolution of
Accountability}\label{historical-evolution-of-accountability}

Now, let's hop on our accountability time machine and cruise through the
historical evolution. Picture this: accountability in the olden days was
like a superhero in training -- it had potential, but it wasn't rocking
the cape just yet. Back in the day, accountability was mostly about
balancing the books and making sure the gold coins matched the
expenditures. It was a simple, one-dimensional hero, a bit like your
first draft of a superhero comic -- you had the basics, but the story
lacked depth. Then came the corporate scandals -- the supervillains of
the business world. Enron, WorldCom, and their cronies started wreaking
havoc. Accountability had to level up its game. It wasn't just about
financial transparency anymore; it needed to combat fraud, corruption,
and unethical practices. As the years rolled on, societal expectations
soared. People weren't satisfied with a superhero who only cared about
the bottom line. They wanted one who cared about social justice, the
environment, and making the world a better place. Accountability got a
makeover, transforming from a numbers guy to a multi-dimensional,
socially responsible hero. Today, accountability is the Tony Stark of
the corporate Avengers -- tech-savvy, socially conscious, and with a
suit that can withstand any ethical challenge. The historical journey of
accountability is like watching a superhero trilogy -- from origin story
to overcoming epic challenges, it's been a wild ride.

\subsubsection{Contemporary Relevance}\label{contemporary-relevance}

Now that we've time-traveled through history, let's land back in the
present. Why is accountability the superhero we all need in today's
business environment? Imagine accountability as the North Star in the
corporate sky. It's the guiding light that keeps your ship from veering
off course and crashing into the iceberg of ethical dilemmas. In a world
where scandal headlines can break the internet, accountability is the
shield that protects organizations from the arrows of public scrutiny.
But what about globalization? That's like accountability's sidekick,
making the superhero's job both more challenging and more critical. With
businesses stretching across borders, accountability isn't just about
saving the city anymore; it's about saving the entire planet. It's the
superhero's responsibility to navigate the cultural mazes, legal
minefields, and ethical quagmires on a global scale. Contemporary
business practices are the superhero's toolkit. Technology, stakeholder
engagement, and ethical decision-making are like the gadgets and gizmos
in Batman's utility belt. They enhance the superhero's abilities, making
sure they can face the challenges of the modern world. But beware -- not
all challenges are created equal. Some are like power-ups, boosting the
superhero's strength and agility. Others are obstacles, like the final
boss level in a video game, testing the superhero's mettle. Complex
supply chains, evolving consumer expectations, and the ever-watchful
eyes of social media -- these are the dragons our superhero must slay.
In a nutshell, accountability is the unsung hero of contemporary
business. It's not just a buzzword; it's the glue that holds ethical
conduct, transparency, and responsible decision-making together in a
sleek, modern package.

\subsubsection{Types and Dimensions of
Accountability}\label{types-and-dimensions-of-accountability}

Now that we've covered the basics, let's dive into the many dimensions
of accountability -- it's not just a black-and-white concept; it's a
rainbow of responsibilities. Picture this: internal accountability is
like your team of superheroes working together inside the superhero
headquarters. It's about taking responsibility for your actions within
the organization, ensuring everyone is on the same page, and there's no
chaos in the Batcave. External accountability, on the other hand, is
when your superhero takes the stage in front of the cheering crowd. It's
about answering to the fans -- customers, regulators, and the broader
community. Your superhero might have a secret identity, but their
actions are out in the open for everyone to see. Now, let's add some
color to the superhero cape -- the social and environmental dimensions
of accountability. It's not just about saving the day; it's about saving
the planet too. It's like Superman deciding to fight for climate justice
and Batman becoming an advocate for social equality. These dimensions
add depth and richness to the superhero narrative. Balancing these
competing demands is like spinning plates. It requires skill, focus, and
a bit of magic. Organizations need to juggle societal expectations,
ethical conduct, and sustainable practices without dropping the
superhero shield. It's about being a responsible corporate citizen and
ensuring your actions align with the values of the cheering crowd. Now,
every superhero needs a sidekick, and in the corporate world, it's
personal accountability. It's about taking responsibility for your
actions, being a team player, and making sure your superhero team
shines. Personal accountability is the Robin to the corporate Batman --
not always in the spotlight but crucial for a dynamic duo. In a
nutshell, accountability isn't a one-size-fits-all concept. It's a
multi-dimensional, technicolor superhero cape that organizations wear to
navigate the complexities of the modern business landscape. And with
great accountability comes great responsibility -- cue the superhero
theme music!

\subsection{Assessment Questions:}\label{assessment-questions-3}

\begin{itemize}
\item
  How do you define accountability within a corporate context?
\item
  Evaluate the significance of historical perspectives in shaping
  contemporary views on accountability.
\end{itemize}

\subsection{Critical Thinking
Questions:}\label{critical-thinking-questions-1}

\begin{itemize}
\item
  To what extent is accountability shared among corporate entities and
  individuals?
\item
  How do cultural and societal factors influence the perception of
  accountability in different contexts?
\end{itemize}

\section{Ethical Issues in Corporate
Accountability}\label{ethical-issues-in-corporate-accountability}

\subsection{Introduction}\label{introduction-3}

Welcome to the circus of ethical acrobatics, where business decisions
walk the tightrope between right and wrong! In this chapter, we're
diving into the world of Ethical Issues in Corporate Accountability.
Grab your popcorn -- it's going to be a show full of business ethics,
corporate social responsibility antics, and a few ethical dilemmas
that'll make your head spin.

\subsection{Ethics in Business: Foundations and
Theories}\label{ethics-in-business-foundations-and-theories}

Picture this: the business world is a stage, and ethics is the director
yelling, ``Action!'' Ethics in business isn't just about avoiding the
bad stuff; it's about putting on an Oscar-worthy performance of
integrity, fairness, and responsibility. Now, let's take a stroll
through the foundations of business ethics. It's like building a sturdy
house -- you need a solid foundation. In business, that foundation is
made of ethical theories. Ever heard of consequentialism? It's like the
Hollywood blockbuster of ethical theories -- the end justifies the
means. Then there's deontology, the indie film that says some actions
are just plain wrong, no matter the outcome. But wait, there's more!
Virtue ethics is the character-driven drama, focusing on being a good
person overall. It's not just about saving the day; it's about being a
superhero 24/7. These theories are the ethical toolkits that guide
business decisions -- like a moral compass pointing north.

\subsection{Corporate Social Responsibility
(CSR)}\label{corporate-social-responsibility-csr}

Alright, let's talk about CSR -- the superhero cape that businesses wear
to show they're not just about profits; they're also about saving the
world. Corporate Social Responsibility is like the philanthropy of the
business world, but instead of donating to a charity, companies become
the charity themselves. CSR is the superhero that fights for social
justice, environmental sustainability, and making the world a better
place. Imagine Batman not only saving Gotham from villains but also
planting trees and supporting local communities. That's CSR in action.
But here's the plot twist -- it's not just about being a do-gooder for
show. CSR is also a strategic move. It's like being the popular kid in
high school -- everyone wants to be friends with the socially
responsible company. It's good for the brand, good for the planet, and
good for the bottom line.

\subsection{Ethical Decision-Making
Processes}\label{ethical-decision-making-processes}

Now, let's step into the ethical decision-making arena. Picture this as
a chess game -- each move determines the fate of the kingdom (or, in
this case, the company). Ethical decision-making is the art of making
choices that align with those shiny ethical theories we talked about
earlier. It's not as easy as flipping a coin. Ethical decision-making
involves a step-by-step process. First, you need to recognize there's an
ethical dilemma -- that's like realizing you're in a suspense thriller.
Next, gather all the information, assess the options, and weigh the
consequences. It's a bit like being a detective -- solving the mystery
of which choice is the most ethical. Then comes the hard part -- making
the decision. It's like being a judge in a courtroom drama. You have to
be fair, unbiased, and ensure your decision stands up to scrutiny.
Finally, you implement the decision, monitor the outcomes, and adjust
course if needed. It's the grand finale of the ethical decision-making
show.

\subsection{Ethical Dilemmas in Corporate
Settings}\label{ethical-dilemmas-in-corporate-settings}

Now, let's plunge into the heart of the ethical rollercoaster -- the
dilemmas that keep business leaders up at night. Ethical dilemmas are
the crossroads where right and wrong meet, shake hands, and confuse
everyone involved. Imagine this: you're the CEO of a company, and you
discover your top-performing product is causing environmental harm. Do
you continue making profits or halt production to save the planet?
That's an ethical dilemma, my friend. Another classic: your company is
faced with a financial crisis, and you can save jobs by cutting corners
on safety regulations. Do you prioritize the well-being of your
employees or the financial survival of the company? Cue the ethical
dilemma drumroll. These dilemmas are like moral Sudoku -- every move
affects the outcome, and there's no easy solution. They challenge the
very fabric of business ethics and force companies to choose between
competing values. But here's the catch -- ethical dilemmas are not just
problems; they're opportunities. They're the chance for businesses to
showcase their commitment to ethical decision-making. It's like a
superhero choosing the harder path because it's the right one. In the
end, understanding and navigating these ethical issues is like mastering
the art of juggling chainsaws -- tricky, but with practice, you'll
become the star of the ethical circus. So, grab your ethical popcorn and
enjoy the show!

\subsection{Assessment Questions:}\label{assessment-questions-4}

\begin{itemize}
\item
  How does corporate social responsibility contribute to ethical
  accountability?
\item
  Analyze the role of ethical decision-making in navigating complex
  corporate scenarios.
\end{itemize}

\subsection{Critical Thinking
Questions:}\label{critical-thinking-questions-2}

\begin{itemize}
\item
  Can profit-driven corporations authentically balance ethical
  considerations?
\item
  How do individual ethical values impact corporate culture and
  accountability?
\end{itemize}

\section{Legal Dimensions of
Accountability}\label{legal-dimensions-of-accountability}

\subsection{Introduction}\label{introduction-4}

Alright, fellow scholars, buckle up for a legal rollercoaster! In this
chapter, we're diving into the thrilling world of Legal Dimensions of
Accountability. Get ready for a journey through global legal frameworks,
the wild terrain of regulatory compliance, a solo exploration into
individual legal accountability, and a spotlight on real-life legal
drama with case studies.

\subsection{Global Legal Frameworks Governing
Corporations}\label{global-legal-frameworks-governing-corporations}

Imagine the legal world as a giant chessboard, and corporations are the
kings and queens making strategic moves. Global Legal Frameworks are
like the rulebook of this chess game -- defining how corporations can
move and ensuring they play fair on the global stage. First up, we've
got international law -- the Avengers of the legal world. Treaties,
agreements, and conventions are the superheroes that bring nations
together to set the rules. They're like the United Nations making sure
everyone follows the script. Then there's the European Union, the
Justice League of regional legal frameworks. It's a powerhouse of rules
and regulations that member countries follow, ensuring a harmonious
legal melody across Europe. Oh, and don't forget about the World Trade
Organization (WTO) -- the referee in our legal boxing ring. It settles
disputes and ensures fair play in the global economic arena.

\subsection{Regulatory Compliance and
Reporting}\label{regulatory-compliance-and-reporting}

Now, let's waltz into the complex world of regulatory compliance -- the
glamorous ballroom where corporations put on their finest legal attire.
Imagine regulations as the dance steps, and companies must cha-cha-cha
their way through to comply. Regulatory compliance is like the corporate
checklist for legal dance moves. Whether it's financial regulations,
environmental standards, or labor laws -- every twirl and dip must align
with the music of the law. Non-compliance is the equivalent of stepping
on your partner's toes; it's frowned upon, and you might end up with
legal bruises. Reporting is the after-party of regulatory compliance.
It's like showing off your dance moves to the world. Corporations have
to document their steps, ensuring they have a trail of evidence for
every legal tango. Transparency is the key -- no hidden dance moves
allowed!

\subsection{Individual Legal
Accountability}\label{individual-legal-accountability}

Now, let's zoom in on the individual actors in our legal drama. It's
time for a solo performance -- Individual Legal Accountability. Imagine
each employee as a legal soloist, responsible for their own legal jazz.
In this act, individuals are like the rock stars of legal
accountability. They're not just cogs in the corporate machine; they're
legally liable for their actions. Think of it as a legal superhero
origin story -- with great power (and a job), comes great legal
responsibility. Corporate officers and directors are the headliners in
this legal concert. They're like the lead singers of the corporation,
and their actions can either create a legal symphony or a cacophony of
legal troubles. But let's not forget the supporting cast -- every
employee. From the intern playing the triangle to the CEO shredding the
guitar solo, everyone contributes to the legal melody. Legal
accountability isn't just for the big shots; it's a band effort.

\subsection{Case Studies on Legal
Accountability}\label{case-studies-on-legal-accountability}

Lights, camera, action! Now, let's turn our attention to the real-life
legal blockbusters -- Case Studies on Legal Accountability. These are
the courtroom dramas that make even the most gripping Netflix series
look like child's play. First on our watchlist is the Enron scandal --
the granddaddy of corporate fraud cases. Enron executives cooked the
financial books like Gordon Ramsay in a high-stakes cooking competition.
Spoiler alert: it didn't end well for them. Next up, Volkswagen's
emission scandal -- the automotive version of a legal thriller. They
rigged their cars to pass emission tests while polluting like a factory
on overdrive. Cue the courtroom showdown with environmental regulators.
And who could forget the epic battle between Apple and the FBI? It's a
legal action movie where privacy rights clash with national security.
It's the kind of drama that keeps law students on the edge of their
seats. These case studies aren't just legal soap operas; they're lessons
in legal accountability. Each one teaches us about the consequences of
bending or breaking the rules. It's like a legal crash course -- without
the boring textbooks. In conclusion, the legal dimensions of
accountability are like a giant legal circus. We've got global
frameworks swinging on the trapeze, regulatory compliance performing an
elegant ballet, individual legal accountability rocking the stage, and
case studies stealing the spotlight. So, grab your legal popcorn, and
let's enjoy the show!

\subsection{Assessment Questions:}\label{assessment-questions-5}

\begin{itemize}
\item
  Examine the legal responsibilities of corporations in different
  jurisdictions.
\item
  How does regulatory compliance contribute to corporate accountability?
\end{itemize}

\subsection{Critical Thinking
Questions:}\label{critical-thinking-questions-3}

\begin{itemize}
\item
  To what extent do existing legal frameworks address the complexities
  of corporate accountability?
\item
  Should legal accountability for corporations be stricter or more
  lenient?
\end{itemize}

\section{Social Implications of
Accountability}\label{social-implications-of-accountability}

\subsection{Introduction}\label{introduction-5}

Hey there, intellectual explorers! Ready to dive into the social
playground of accountability? In this chapter, we're unraveling the
Social Implications of Accountability. It's like a mixtape of
stakeholder engagement, corporate reputation management beats, the
social media dance, and the virtuoso performance of corporate
citizenship. Get your intellectual dancing shoes on -- we're about to
groove through the world of social responsibility.

\subsection{Stakeholder Engagement and
Accountability}\label{stakeholder-engagement-and-accountability}

Cue the spotlight -- it's time to talk about Stakeholder Engagement and
Accountability, the rockstars of the social responsibility arena.
Imagine stakeholders as the VIPs at a concert, and the band is the
corporation. Engaging with stakeholders is like having a backstage pass
-- it's all about building connections, hearing concerns, and ensuring
the concert (or business) is a hit. So, who are these stakeholders?
Think of them as the entourage: customers, employees, investors,
communities, and the environment. They're not just spectators; they're
part of the show, influencing the performance with their cheers (or
boos). Stakeholder engagement is the art of keeping everyone happy. It's
like planning a surprise party -- you want to know what each guest
likes, dislikes, and if they're allergic to peanuts. In the business
world, it involves listening to feedback, understanding expectations,
and ensuring everyone feels like a VIP. Now, let's talk about
accountability in this social concert. It's not just about putting on a
good show; it's about being answerable to the crowd. If the drummer
messes up, they need to take responsibility and promise a better beat
next time. Accountability is the encore of a successful engagement -- it
keeps the audience coming back for more.

\subsection{Corporate Reputation
Management}\label{corporate-reputation-management}

Lights, camera, action! It's time for the blockbuster drama of Corporate
Reputation Management. Imagine corporate reputation as the leading actor
in a Hollywood movie -- it can either win an Oscar or become a box
office disaster. Reputation is the currency of the business world, and
managing it is like producing a hit movie. Your corporate reputation is
what people say about you when you're not in the room. It's like the
rumor mill at high school -- you want to be the cool kid, not the one
with spinach stuck in their teeth. Reputation is fragile, and once
damaged, it's harder to fix than a broken vase. Managing corporate
reputation is a bit like a chess game. You've got to think several moves
ahead and anticipate how your actions will be perceived. It's not just
about providing great products or services; it's about being socially
responsible, ethical, and ensuring your business practices align with
public values. Reputation management involves crisis control too.
Imagine your lead actor (read: CEO) accidentally posts a rant on social
media -- that's a PR disaster! Reputation management steps in like a
superhero, ready to save the day by acknowledging mistakes, apologizing,
and taking corrective action. Remember, reputation is the red carpet you
walk on. If it's full of potholes, you're more likely to trip and fall.
But if it's well-maintained, you'll be the star of the show, getting
invited to all the glamorous after-parties.

\subsection{Social Media's Impact on
Accountability}\label{social-medias-impact-on-accountability}

Alright, digital citizens, let's talk about the game-changer in the
world of accountability -- Social Media's Impact. Social media is like
the cool kid in high school who knows all the gossip and spreads it
faster than wildfire. In the realm of accountability, social media is
both the hero and the villain. First, the hero side: social media has
democratized information. Everyone can be a journalist, exposing
corporate secrets or praising ethical practices. It's like the people's
voice on steroids -- loud, powerful, and impossible to ignore. Companies
are no longer operating in the shadows; they're under the spotlight of
the digital stage. Now, the villain side: social media can be a brutal
judge. One viral tweet or a damning Facebook post can tarnish a
reputation faster than you can say ``hashtag.'' It's like a digital
courtroom where judgments are swift and public. Companies need to be on
their toes, ready to respond to the social media jury. Remember the
United Airlines incident where a passenger was forcefully removed?
Social media erupted like a volcano, and United Airlines felt the heat.
The viral video damaged their reputation and stock value. It's a lesson
that in the digital age, accountability is not just about legality; it's
about perception. Social media also gives power to stakeholders. Angry
customers can tweet their grievances, and employees can expose workplace
issues. It's a digital revolution where accountability is not just
enforced by laws but by the collective voice of the people.

\subsection{Corporate Citizenship and Social
Responsibility}\label{corporate-citizenship-and-social-responsibility}

Drumroll, please! It's time for the grand finale -- Corporate
Citizenship and Social Responsibility. Think of corporate citizenship as
the ultimate virtuoso performance, where the company isn't just a player
in the market but a responsible member of society. It's not about having
a seat at the table; it's about setting the table for everyone.
Corporate social responsibility (CSR) is like the company's superhero
cape. It's the commitment to doing business in an ethical and socially
responsible way. Imagine CSR as a superhero origin story -- a company
realizing it has the power to make the world a better place and deciding
to use that power for good. CSR is not just about writing checks to
charities (although that's nice too). It's about integrating ethical
practices into every aspect of the business. It's the company saying,
``I want to be a good citizen, not just a good business.'' Let's talk
about sustainability -- it's not just a buzzword; it's a lifestyle.
Sustainable business practices ensure that today's profits don't come at
the expense of tomorrow's resources. It's like eating your cake and
having it too -- enjoying current success without jeopardizing future
opportunities. And don't forget about philanthropy. It's not just about
donating money; it's about giving back to the community. It's like being
the Robin Hood of the business world -- taking from profits to give to
the greater good. Corporate citizenship is also about diversity and
inclusion. It's not just a checkbox; it's about ensuring everyone has a
seat at the table, regardless of race, gender, or background. It's a
celebration of differences, making the corporate culture a rich tapestry
of perspectives. In conclusion, the social implications of
accountability are like a symphony -- stakeholder engagement creates the
harmonious melody, corporate reputation management conducts the
orchestra, social media adds the digital notes, and corporate
citizenship performs the grand finale. So, let's give a standing ovation
to accountability in the social spotlight!

\subsection{Assessment Questions:}\label{assessment-questions-6}

\begin{itemize}
\item
  Analyze the impact of stakeholder engagement on corporate
  accountability.
\item
  Evaluate the role of social media in holding corporations accountable.
\end{itemize}

\subsection{Critical Thinking
Questions:}\label{critical-thinking-questions-4}

\begin{itemize}
\item
  How can corporations balance the interests of various stakeholders to
  ensure accountability?
\item
  Is corporate citizenship a genuine commitment or a strategic public
  relations move?
\end{itemize}

\section{Integrating Corporate and Individual
Accountability}\label{integrating-corporate-and-individual-accountability}

\subsection{Introduction}\label{introduction-6}

Greetings, knowledge enthusiasts! Today, we're embarking on an
intellectual adventure exploring the dynamic realm of Integrating
Corporate and Individual Accountability. It's like finding the perfect
blend of peanut butter and jelly -- a harmonious mix that transforms the
ordinary into something extraordinary. Join us as we unravel the
interconnectedness of responsibilities, build a culture of
accountability, peek into real-life success stories, and face head-on
the challenges of implementing this integrated accountability approach.

\subsection{Interconnectedness of Corporate and Individual
Responsibilities}\label{interconnectedness-of-corporate-and-individual-responsibilities}

Let's kick off our journey by diving deep into the interconnectedness of
Corporate and Individual Responsibilities. Picture this as a dance --
the tango of the organizational world, where the corporation and its
employees move in perfect harmony, creating a masterpiece of
accountability. At the heart of this dance is the recognition that
corporations are not soulless entities; they are a collective of
individuals working towards common goals. It's like a superhero team --
each member contributes unique strengths, and their actions impact the
success of the entire mission. Individual employees are the superheroes
in this narrative. They are not just cogs in the corporate machine; they
are the lifeblood, the decision-makers, and the ambassadors of the
organization. When they embrace accountability, it's like each superhero
putting on their cape and taking responsibility for their part in the
grand storyline. Now, let's talk about corporate responsibility -- the
guiding force that sets the tone for the entire dance. The corporation,
like a wise dance instructor, must establish clear expectations, provide
resources, and create an environment where individual superheroes can
shine. It's a partnership, a collaborative effort to ensure that the
dance of accountability is a dazzling spectacle. Imagine a scenario
where a corporation is like a cruise ship, and individual employees are
the crew. If everyone does their job responsibly, the ship sails
smoothly. But if one crew member slacks off, it can lead to chaos.
That's the interconnectedness -- a shared responsibility for the success
of the voyage.

\subsection{Building a Culture of
Accountability}\label{building-a-culture-of-accountability}

Now that we've mastered the dance steps, it's time to focus on Building
a Culture of Accountability -- the backstage magic that turns a one-time
performance into a long-running show. Building a culture of
accountability is not just a workshop or a seminar; it's a lifestyle, a
way of doing things that permeates every nook and cranny of the
organization. Think of it as creating a garden. The seeds are the values
and principles that define accountability, and the culture is the
ecosystem that nurtures their growth. It's not just about planting the
seeds; it's about ensuring the right soil, sunlight, and water for them
to flourish. Leadership plays a pivotal role in cultivating this garden.
It's like having green-thumbed gardeners who set the tone, model
accountability, and create an environment where it's cool to be
responsible. Leaders are the architects of the organizational culture,
designing a blueprint that promotes accountability from the ground up.
Communication is the water that nourishes the accountability garden.
It's not just about shouting instructions from a megaphone; it's about
having open, transparent, and honest conversations. It's the fertilizer
that helps accountability values take root and spread. Accountability
should be as ingrained in the corporate culture as the morning coffee
routine. It's not a chore; it's a habit. Employees should wake up and
think, ``Ah, time to be accountable today!'' When accountability becomes
second nature, it's not just a cultural trait; it's the heartbeat of the
organization. Now, let's talk about trust -- the secret ingredient that
binds the accountability garden together. Trust is like the magical
potion that makes everything flourish. When employees trust their
leaders, and leaders trust their employees, it creates a fertile ground
where accountability can thrive.

\subsection{Case Studies on Successful
Integration}\label{case-studies-on-successful-integration}

Time to spotlight some success stories! We're diving into real-life Case
Studies on Successful Integration of Corporate and Individual
Accountability. These are not just tales of triumph; they are living
proof that when the dance is in sync, and the garden is well-tended, the
results can be extraordinary. Let's start with Google -- the tech giant
that doesn't just search for information; it searches for accountability
in every nook and cranny of its corporate culture. Google's 20\% time
policy, where employees are encouraged to spend a fifth of their work
hours on personal projects, is a testament to individual empowerment and
accountability. This policy has birthed some of Google's most successful
products, like Gmail and Google Maps. Next up, we have Zappos -- the
online shoe and clothing company that didn't just revolutionize
e-commerce; it rewrote the rules of corporate accountability. Zappos
places a premium on employee happiness, and they empower their customer
service reps to do whatever it takes to make customers happy. This
freedom and accountability combo has created a company known for its
exceptional customer service. Then, there's Patagonia -- the outdoor
clothing company that's not just about making jackets; it's about making
a statement. Patagonia has integrated corporate and individual
accountability by embedding environmental and social responsibility into
its DNA. It's a living example of how a company can be profitable while
staying true to its values. These case studies aren't just success
stories; they are blueprints for organizations looking to weave
individual and corporate accountability into their fabric. They prove
that accountability is not a restrictive dance; it's a creative
expression that can lead to innovation, exceptional service, and a
positive impact on the world.

\subsection{Challenges in Implementing Integrated
Accountability}\label{challenges-in-implementing-integrated-accountability}

Lights, camera, but wait -- it's time to face the Challenges in
Implementing Integrated Accountability. Every great story has its
obstacles, and the integration of corporate and individual
accountability is no exception. Let's roll up our sleeves and tackle the
dragons in the accountability arena. One significant challenge is
resistance to change. It's like trying to introduce a new dance move to
a group of seasoned dancers. People get comfortable with the status quo,
and any attempt to shake things up is met with skepticism. Overcoming
this resistance requires effective communication, a compelling vision,
and sometimes, a bit of patience. Another hurdle is the lack of clear
guidelines. Picture this as trying to perform a dance routine without a
choreographer. When individuals and the corporation don't have a clear
roadmap for accountability, it can lead to confusion, misunderstandings,
and a dance that resembles chaos more than coordination. Establishing
clear expectations, providing guidance, and offering support can help
address this challenge. Accountability fatigue is a real thing. It's
like trying to maintain enthusiasm for the dance

(Amonoo Nkrumah et al. 2023) (Đorđević et al. 2019) (Kang, Hsiao, and Ni
2022) (Honigsberg 2019) (Jeursen 2022) (Hameed, Alfaraj, and Hameed
2023) (Clarke 2018) (Blinova, Ponomarenko, and Tesovskaya 2023) (Zhang,
Imeni, and Edalatpanah 2023)

\subsection{Assessment Questions:}\label{assessment-questions-7}

\begin{itemize}
\item
  Explain the concept of integrated accountability and its significance.
\item
  How can organizations foster a culture of accountability among
  individuals?
\end{itemize}

\subsection{Critical Thinking
Questions:}\label{critical-thinking-questions-5}

\begin{itemize}
\item
  What are the main obstacles to successfully integrating corporate and
  individual accountability?
\item
  In what ways can leadership contribute to a more accountable
  organizational culture?
\end{itemize}

\section{Appendix: Answers}\label{appendix-answers}

\subsection{Chapter 1: Understanding
Accountability}\label{chapter-1-understanding-accountability}

\subsubsection{Assessment Questions:}\label{assessment-questions-8}

\begin{itemize}
\item
  \emph{How do you define accountability within a corporate context?}
  Accountability in the corporate context refers to the obligation of
  individuals and organizations to take responsibility for their
  actions, decisions, and their impact on stakeholders. It involves
  transparency, reliability, and answering for the consequences of one's
  actions.
\item
  \emph{Evaluate the significance of historical perspectives in shaping
  contemporary views on accountability.} Historical perspectives on
  accountability, such as the evolution from strict legal compliance to
  a broader societal and ethical focus, have shaped contemporary views
  by highlighting the need for a more holistic and socially responsible
  approach. These perspectives underscore the dynamic nature of
  accountability, evolving beyond a mere legal obligation. \#\#\#\#
  Critical Thinking Questions:
\item
  \emph{To what extent is accountability shared among corporate entities
  and individuals?} Accountability is shared among corporate entities
  and individuals in a symbiotic relationship, where each plays a
  crucial role. Corporations are accountable for their policies and
  practices, while individuals are responsible for their actions within
  the organizational context. This shared responsibility fosters a
  culture of accountability.
\item
  \emph{How do cultural and societal factors influence the perception of
  accountability in different contexts?} Cultural and societal factors
  influence accountability by shaping ethical norms and expectations.
  Cultural values can impact the emphasis on collective versus
  individual responsibility, influencing how accountability is perceived
  and practiced in various organizational and social contexts. \#\#\#
  Chapter 2: Ethical Issues in Corporate Accountability
\end{itemize}

\subsubsection{Assessment Questions:}\label{assessment-questions-9}

\begin{itemize}
\item
  \emph{How does corporate social responsibility contribute to ethical
  accountability?} Corporate social responsibility contributes to
  ethical accountability by requiring organizations to consider the
  social and environmental impact of their actions. It fosters a
  commitment to ethical behavior beyond legal compliance, promoting a
  positive influence on society.
\item
  \emph{Analyze the role of ethical decision-making in navigating
  complex corporate scenarios.} Ethical decision-making in corporate
  settings involves considering moral principles, values, and potential
  consequences. It guides leaders in making choices that align with
  ethical standards, fostering trust and integrity within the
  organization. \#\#\#\# Critical Thinking Questions:
\item
  \emph{Can profit-driven corporations authentically balance ethical
  considerations?} Achieving a balance between profit motives and
  ethical considerations is challenging but crucial. The challenge lies
  in aligning profit goals with ethical standards to build long-term
  sustainability and trust with stakeholders.
\item
  \emph{How do individual ethical values impact corporate culture and
  accountability?} Individual ethical values influence corporate culture
  by shaping the collective moral compass within an organization. When
  aligned, individual values contribute to a strong ethical culture,
  fostering a sense of responsibility and accountability among
  employees.
\end{itemize}

\subsection{Chapter 3: Legal Dimensions of
Accountability}\label{chapter-3-legal-dimensions-of-accountability}

\subsubsection{Assessment Questions:}\label{assessment-questions-10}

\begin{itemize}
\item
  \emph{Examine the legal responsibilities of corporations in different
  jurisdictions.} Corporations bear legal responsibilities that vary
  across jurisdictions, encompassing compliance with local regulations,
  international standards, and industry-specific laws.
\item
  \emph{How does regulatory compliance contribute to corporate
  accountability?} Regulatory compliance contributes to corporate
  accountability by establishing standards and guidelines that
  organizations must follow. It ensures adherence to legal obligations,
  promoting transparency and trust. \#\#\#\# Critical Thinking
  Questions:
\item
  \emph{To what extent do existing legal frameworks address the
  complexities of corporate accountability?} Existing legal frameworks
  provide a foundation for corporate accountability, but their
  effectiveness depends on the adaptability to evolving business
  landscapes. Continuous evaluation and updates are necessary to address
  emerging challenges.
\item
  \emph{Should legal accountability for corporations be stricter or more
  lenient?} Striking a balance is essential. While strict legal
  accountability is necessary to deter unethical behavior, flexibility
  is required to accommodate diverse business models and contextual
  factors. \#\#\# Chapter 4: Social Implications of Accountability
\end{itemize}

\subsubsection{Assessment Questions:}\label{assessment-questions-11}

\begin{itemize}
\item
  \emph{Analyze the impact of stakeholder engagement on corporate
  accountability.} Stakeholder engagement positively impacts corporate
  accountability by fostering communication, understanding, and mutual
  responsibility. Engaged stakeholders hold organizations accountable
  for their actions.
\item
  \emph{Evaluate the role of social media in holding corporations
  accountable.} Social media plays a significant role in holding
  corporations accountable by providing a platform for public scrutiny,
  activism, and real-time dissemination of information. It enhances
  transparency and accountability. \#\#\#\# Critical Thinking Questions:
\item
  \emph{How can corporations balance the interests of various
  stakeholders to ensure accountability?} Balancing stakeholder
  interests involves effective communication, responsiveness, and
  considering diverse perspectives. Organizations must prioritize
  ethical decision-making to address the varied needs of stakeholders.
\item
  \emph{Is corporate citizenship a genuine commitment or a strategic
  public relations move?} Corporate citizenship should be a genuine
  commitment, rooted in ethical values and a desire to contribute
  positively to society. While public relations benefits may result,
  authenticity is essential for long-term credibility. \#\#\# Chapter 5:
  Integrating Corporate and Individual Accountability
\end{itemize}

\subsubsection{Assessment Questions:}\label{assessment-questions-12}

\begin{itemize}
\item
  \emph{Explain the concept of integrated accountability and its
  significance.} Integrated accountability emphasizes the
  interconnectedness of corporate and individual responsibilities,
  promoting a holistic approach to ethical, legal, and social
  considerations. It ensures alignment between individual actions and
  organizational goals.
\item
  \emph{How can organizations foster a culture of accountability among
  individuals?} Fostering a culture of accountability involves
  leadership commitment, clear communication of expectations, and
  providing support for ethical decision-making. It requires an
  organizational environment that encourages responsibility and learning
  from mistakes. \#\#\#\# Critical Thinking Questions:
\item
  \emph{What are the main obstacles to successfully integrating
  corporate and individual accountability?} Obstacles may include
  resistance to change, lack of awareness, and organizational culture
  that does not prioritize accountability. Addressing these challenges
  requires strategic planning and sustained effort.
\item
  \emph{In what ways can leadership contribute to a more accountable
  organizational culture?} Leadership plays a pivotal role by setting an
  example, communicating expectations, and creating an environment that
  encourages openness and learning. Leaders must prioritize ethical
  decision-making and hold themselves accountable.
\end{itemize}

\section{References}\label{references-2}

\phantomsection\label{refs}
\begin{CSLReferences}{1}{0}
\bibitem[\citeproctext]{ref-abelson1996qc}
Abelson, Harold, and Gerald Jay Sussman. 1996. \emph{Structure and
Interpretation of Computer Programs}. 2nd ed. MIT Electrical Engineering
and Computer Science. London, England: MIT Press.

\bibitem[\citeproctext]{ref-aws_cdn}
Amazon Web Services. n.d. {``What Is a CDN (Content Delivery
Network)?''} n.d. \url{https://aws.amazon.com/what-is/cdn/}.

\bibitem[\citeproctext]{ref-aws_load_balancing}
---------. n.d. {``What Is Load Balancing?''} n.d.
\url{https://aws.amazon.com/what-is/load-balancing/}.

\bibitem[\citeproctext]{ref-amonoo2023stakeholder}
Amonoo Nkrumah, B., W. Qian, A. Kaur, and C. Tilt. 2023. {``Stakeholder
Accountability in the Era of Big Data: An Exploratory Study of Online
Platform Companies.''} \emph{Qualitative Research in Accounting \&
Management} 20 (4): 447--84.
\url{https://doi.org/10.1108/QRAM-03-2022-0042}.

\bibitem[\citeproctext]{ref-annunziato2020experience}
Annunziato, M. 2020. {``The {`Experience Age'} Is Already Here.''} The
Daily Campus. 2020.
\url{https://dailycampus.com/2020/10/09/the-experience-age-is-already-here/}.

\bibitem[\citeproctext]{ref-Awati2021bz}
Awati, Rahul, and Linda Rosencrance. 2021. {``Computer Hardware.''}
\url{https://www.techtarget.com/searchnetworking/definition/hardware};
TechTarget. October 2021.

\bibitem[\citeproctext]{ref-blinova2023corporatesustainability}
Blinova, E., T. Ponomarenko, and S. Tesovskaya. 2023. {``Key Corporate
Sustainability Assessment Methods for Coal Companies.''}
\emph{Sustainability (2071-1050)} 15 (7): 5763.
\url{https://doi.org/10.3390/su15075763}.

\bibitem[\citeproctext]{ref-busuena2023introduction}
Busueña, J., and J. Pomperada. 2023. \emph{Introduction to Information
Technology and Computer Fundamentals}. 2nd ed. Unlimited Books Library
Services; Publishing Inc.

\bibitem[\citeproctext]{ref-clarke2018individual}
Clarke, Blanaid. 2018. {``Individual Accountability in Irish Credit
Institutions---Lessons to Be Learned from the United Kingdom's Senior
Managers' Regime.''} \emph{Common Law World Review}, 35--52.

\bibitem[\citeproctext]{ref-cloudflare_vpc}
Cloudfare. n.d. {``What Is a Virtual Private Cloud (VPC)?''} n.d.
\url{https://www.cloudflare.com/learning/cloud/what-is-a-virtual-private-cloud/}.

\bibitem[\citeproctext]{ref-cocca2022dn}
Cocca, Germán. 2022. {``Programming Paradigms -- Paradigm Examples for
Beginners.''}
\url{https://www.freecodecamp.org/news/an-introduction-to-programming-paradigms/}.
May 2022.

\bibitem[\citeproctext]{ref-djordjevic2019corporate}
Đorđević, D. B., M. Vuković, S. Urošević, N. Štrbac, and A. Vuković.
2019. {``Studying the Corporate Social Responsibility in Apparel and
Textile Industry.''} \emph{Industria Textila} 70 (4): 336--41.
\url{https://doi.org/10.35530/IT.070.04.1572}.

\bibitem[\citeproctext]{ref-englander2021mq}
Englander, Irv, and Wilson Wong. 2021. \emph{The Architecture of
Computer Hardware, Systems Software, and Networking}. 6th ed. Nashville,
TN: John Wiley \& Sons.

\bibitem[\citeproctext]{ref-geeks_for_geeks_iot_challenges}
Geeks, Geeks for. 2023. {``Challenges in Internet of Things (IoT).''}
2023.
\url{https://www.geeksforgeeks.org/challenges-in-internet-of-things-iot/}.

\bibitem[\citeproctext]{ref-gillis2023iot}
Gillis, A. S. 2023. {``Internet of Things (IoT).''} 2023.
\url{https://www.techtarget.com/iotagenda/definition/Internet-of-Things-IoT}.

\bibitem[\citeproctext]{ref-gilster2001pc}
Gilster, Ron. 2001. \emph{{PC} Hardware: A Beginner's Guide}. New York:
Osborne/{McGraw}-Hill. \url{http://site.ebrary.com/id/10015274}.

\bibitem[\citeproctext]{ref-goodman2008information}
Goodman, Seymour, Detmar W. Straub, and Richard Baskerville. 2008.
\emph{Information Security: Policy, Processes, and Practices}.
Routledge.

\bibitem[\citeproctext]{ref-google_cloud_storage}
Google. n.d. {``What Is Cloud Storage?''} n.d.
\url{https://cloud.google.com/learn/what-is-cloud-storage}.

\bibitem[\citeproctext]{ref-hameed2023association}
Hameed, F., M. Alfaraj, and K. Hameed. 2023. {``The Association of Board
Characteristics and Corporate Social Responsibility Disclosure Quality:
Empirical Evidence from Pakistan.''} \emph{Sustainability (2071-1050)}
15 (24): 16849. \url{https://doi.org/10.3390/su152416849}.

\bibitem[\citeproctext]{ref-honigsberg2019individual}
Honigsberg, C. 2019. {``The Case for Individual Audit Partner
Accountability.''} \emph{Vanderbilt Law Review} 72 (6): 1871--1922.

\bibitem[\citeproctext]{ref-ibm_cloud_computing}
IBM. n.d. {``What Is Cloud Computing?''} n.d.
\url{https://www.ibm.com/topics/cloud-computing}.

\bibitem[\citeproctext]{ref-ibm_iot}
---------. n.d. {``What Is the Internet of Things (IoT)?''} n.d.
\url{https://www.ibm.com/topics/internet-of-things}.

\bibitem[\citeproctext]{ref-ivey2023iotapplications}
Ivey, A. 2023. {``7 Real-World IoT Applications and Examples.''} 2023.
\url{https://cointelegraph.com/news/7-iot-applications-and-examples}.

\bibitem[\citeproctext]{ref-jebaraj2023cloudcompanies}
Jebaraj, K. 2023. {``Top 10 Cloud Computing Companies of 2024.''} 2023.
\url{https://www.knowledgehut.com/blog/cloud-computing/top-cloud-computing-companies}.

\bibitem[\citeproctext]{ref-jeursen2022cover}
Jeursen, T. 2022. {``"Cover Your Ass": Individual Accountability, Visual
Documentation, and Everyday Policing in Miami.''} \emph{PoLAR: Political
\& Legal Anthropology Review} 45 (2): 186--200.
\url{https://doi.org/10.1111/plar.12505}.

\bibitem[\citeproctext]{ref-kang2022sustainabletraining}
Kang, Y.-C., H.-S. Hsiao, and J.-Y. Ni. 2022. {``The Role of Sustainable
Training and Reward in Influencing Employee Accountability Perception
and Behavior for Corporate Sustainability.''} \emph{Sustainability
(2071-1050)} 14 (18): 11589--N.PAG.
\url{https://doi.org/10.3390/su141811589}.

\bibitem[\citeproctext]{ref-keutzer1994he}
Keutzer, Kurt. 1994. {``Hardware/Software Co-Simulation.''} In
\emph{Proceedings of the 31st Annual Conference on Design Automation
Conference - {DAC} '94}. New York, New York, USA: ACM Press.

\bibitem[\citeproctext]{ref-lombardi2004information}
Lombardi, Olimpia. 2004. {``What Is Information?''} \emph{Foundations of
Science} 9: 105--34.
\url{https://doi.org/10.1023/B:FODA.0000025034.53313.7c}.

\bibitem[\citeproctext]{ref-lundin2023information}
Lundin, L. L. 2023. \emph{Information Security}. Salem Press
Encyclopedia.

\bibitem[\citeproctext]{ref-marr2023cloudtrends}
Marr, B. 2023. {``The 10 Biggest Cloud Computing Trends in 2024 Everyone
Must Be Ready for Now.''} 2023.
\url{https://www.forbes.com/sites/bernardmarr/2023/10/09/the-10-biggest-cloud-computing-trends-in-2024-everyone-must-be-ready-for-now/?sh=ea1da466d672}.

\bibitem[\citeproctext]{ref-microsoft2023securedata}
Microsoft. 2023. {``11 Best Practices for Securing Data in Cloud
Services.''} 2023.
\url{https://www.microsoft.com/en-us/security/blog/2023/07/05/11-best-practices-for-securing-data-in-cloud-services/}.

\bibitem[\citeproctext]{ref-muts2024iot}
Muts, I. 2024. {``10+ Best IoT Cloud Platforms in 2024.''} 2024.
\url{https://euristiq.com/best-iot-cloud-platforms/}.

\bibitem[\citeproctext]{ref-sebesta2015ek}
Sebesta, Robert W. 2015. \emph{Concepts of Programming Languages}. 11th
ed. Upper Saddle River, NJ: Pearson.

\bibitem[\citeproctext]{ref-stallings2016computer}
Stallings, William. 2016. \emph{Computer Organization and Architecture}.
10th edition. Pearson.

\bibitem[\citeproctext]{ref-sugandhi2023iotfuture}
Sugandhi, A. 2023. {``The Future of IoT: Trends and Predictions for
2024.''} 2023.
\url{https://www.knowledgehut.com/blog/web-development/iot-future}.

\bibitem[\citeproctext]{ref-velazquez2022iot}
Velazquez, R. 2022. {``IoT: The Internet of Things. What Is the Internet
of Things? How Does IoT Work?''} 2022.
\url{https://builtin.com/internet-things}.

\bibitem[\citeproctext]{ref-weiser1999og}
Weiser, Mark. 1999. {``The Computer for the 21 \(^{st}\) Century.''}
\emph{ACM SIGMOBILE Mob. Comput. Commun. Rev.} 3 (3): 3--11.

\bibitem[\citeproctext]{ref-williams2012using}
Williams, B. K., and S. C. Sawyer. 2012. \emph{Using Information
Technology: Introductory Edition}. McGraw-Hill Europe.

\bibitem[\citeproctext]{ref-zhang2023environmental}
Zhang, Y., M. Imeni, and S. A. Edalatpanah. 2023. {``Environmental
Dimension of Corporate Social Responsibility and Earnings Persistence:
An Exploration of the Moderator Roles of Operating Efficiency and
Financing Cost.''} \emph{Sustainability (2071-1050)} 15 (20): 14814.
\url{https://doi.org/10.3390/su152014814}.

\end{CSLReferences}

\bookmarksetup{startatroot}

\chapter{Hardware and Software}\label{hardware-and-software}

\bookmarksetup{startatroot}

\chapter{Hardware and Software}\label{hardware-and-software-1}

\section{Learning Objectives}\label{learning-objectives-3}

\emph{After studying this chapter, you should be able to:}

\begin{itemize}
\tightlist
\item
  Explain the organization of a computing system and understand the
  relationships and interactions between hardware components.
\item
  Define the general functions and identify key hardware components,
  including central processing units (CPUs), memory, and input and
  output devices.
\item
  Describe the functions of the different types of software, mainly the
  operating systems, applications, and utilities.
\item
  Understand fundamental programming concepts, such as algorithms,
  variables, control structures, and data types.
\item
  Explain the symbiotic relationship between hardware and software in
  computing systems.
\end{itemize}

\section{Introduction}\label{introduction-7}

As we study information science, it is essential to establish a
comprehensive understanding of the fundamental technologies inherent to
information systems. This chapter serves as a guide to explore a
computing system, particularly the computer system, with a central focus
on its two primary categories: hardware and software. This foundational
knowledge serves as the prerequisite to discern the intricate functions
that govern digital information.

By exploring the specifics of hardware and software, we pave the way to
identify potential innovations and improvements. As these technological
components form the backbone influencing the landscape of information
science -- without the nuanced understanding of these fundamental
elements, our ability to navigate and contribute meaningfully to the
advancements in digital information would be constrained.

When we talk about computers, our minds naturally conjure up images of
tangible elements such as a monitor, a keyboard, a mouse, and other
electronic components neatly enclosed within a rectangular casing. To
put it simply, this is what we call \textbf{Hardware}. To give it a
proper definition, the hardware refers to the physical components of an
analog or digital computer system or electronic devices. These are the
machinery, circuits, and devices that constitute the computer's physical
structure and enable it to function. While we commonly associate
hardware with personal computers, it actually permeates an extensive
array of computing systems and electronic devices such as mobile
devices, point-of-sale devices, self-service checkout machines,
automated kiosks, gaming consoles, medical devices, and others.

Yet, these components and devices are merely manufactured silicon,
aluminum, or copper. There exists a crucial counterpart to this
physicality: the \textbf{Software}. It is the software of a system that
breathes life into these mechanical structures, transforming it to
dynamic and intelligent tools. Software refers to the intangible set of
instructions, programs, or data that tells the computing system to
execute specific tasks or operations. It comes into the form of code
written in programming languages, applications, operating systems, and
other system utilities. A computing device can only properly function
when both hardware and software work together.

The sophisticated nature of a computer system makes it difficult to
fully understand from the outset. A key to it is to look into the
hierarchical nature inherent in complex systems. This hierarchical
system refers to the arrangement of interrelated subsystems in a
structured hierarchy or top-down layers. Think of it as a set of nested
categories, starting with a general category at the top and becoming
more specific as you go down. This way of organizing is helpful in
defining its design and description. The overarching concepts can be
systematically dissected and streamlined individually, contingent upon
the specific level under consideration. At each level, the focus is
directed towards two essential aspects: the structure and the function.
\textbf{Structure} is defined as the arrangement and interrelationship
of components within the system. It is the spatial configuration or
organizational framework that tells how different parts of a system
interact. \textbf{Function} is defined as the role of each individual
component within the system. It is the operations, activities, or tasks
that the individual component performs (Stallings 2016).

Now that we understand how systems are generally organized, let us dive
into the actual basics. This chapter will cover key concepts, including:

\begin{itemize}
\tightlist
\item
  Hardware

  \begin{itemize}
  \tightlist
  \item
    Internal Hardware
  \item
    Computer Architecture and Organization
  \item
    Input and Output Devices
  \end{itemize}
\item
  Software

  \begin{itemize}
  \tightlist
  \item
    Operating Systems
  \item
    Libraries/Utilities
  \item
    Applications
  \item
    Programming Languages
  \end{itemize}
\end{itemize}

\section{Hardware}\label{hardware}

In general, a computer system can perform four (4) basic functions:

\begin{enumerate}
\def\labelenumi{\arabic{enumi}.}
\tightlist
\item
  \textbf{Data processing} involves the collection of data into the
  system and the conversion of it into functional information for the
  system to utilize.
\item
  \textbf{Data storage} is the organized storing or holding of data
  within the system enabling the system or user to retrieve it when
  needed.
\item
  \textbf{Data movement} is the process of transporting data from one
  location to another, facilitating communication.
\item
  \textbf{Control} manages the flow of data and execution of
  instructions between different components of the system.
\end{enumerate}

To enable the computer to execute these functions seamlessly, it relies
on a set of integral structural components. A traditional computer with
a single processor has four (4) main ones:

\begin{enumerate}
\def\labelenumi{\arabic{enumi}.}
\tightlist
\item
  A \textbf{central processing unit (CPU)}, or \textbf{processor}, which
  controls the operation of a computer and processes the data to execute
  the instructions for the system to function. It is the core element
  that drives the computational capabilities of a computer.
\item
  \textbf{Main memory} refers to the electronic storage where data and
  instructions are stored for processing.
\item
  \textbf{Input and output} devices, collectively referred to as
  \textbf{I/O}, communicate the data or signal between a computer system
  and its external environment.

  \begin{enumerate}
  \def\labelenumii{\roman{enumii})}
  \tightlist
  \item
    \emph{Input} refers to the data received by the computer, while
  \item
    \emph{Output} refers to the data that the computer sends to its
    external devices.
  \end{enumerate}
\item
  \textbf{System interconnection} refers to the physical links that
  connect various components within a computer system, enabling them to
  communicate and work together.

  \begin{enumerate}
  \def\labelenumii{\roman{enumii})}
  \tightlist
  \item
    A \emph{system bus} is a communication system that consists of
    conducting wires with all other components attached to it to
    transmit data within the computer.
  \item
    A \emph{networking infrastructure} is also a type of system
    interconnection that allows multiple systems to communicate with
    each other.
  \end{enumerate}
\end{enumerate}

These components form the internal and external elements of a computer,
with the internal parts referred to as components and the external
hardware components known as peripherals. There may be one or more of
each of these components. If there is more than one processor, then it
is called a multicore computer (Stallings 2016; Weiser 1999).

\begin{enumerate}
\def\labelenumi{\arabic{enumi}.}
\setcounter{enumi}{4}
\tightlist
\item
  A \textbf{core} is added to its structure which consists of all the
  processors inside a single chip.
\item
  Multicore computers also use multiple layers or memory called
  \textbf{cache}. This is a smaller version of the main memory, enabling
  the system to access data faster than storing it in the main memory.
\end{enumerate}

Let's look into probably the most complex component, which is the CPU.
The main structure of the CPU consists of the following units:

\begin{enumerate}
\def\labelenumi{\arabic{enumi}.}
\tightlist
\item
  The \textbf{control unit} manages the overall operations of the CPU,
  processing the data and executing the instructions.
\item
  The \textbf{arithmetic and logic unit (ALU)} performs the actual data
  processing of the system i.e.~the arithmetic (addition, subtraction,
  multiplication, and division) and logical (AND, OR, NOT, XOR)
  operations.
\item
  The \textbf{registers} are what serves as the internal storage of the
  CPU.
\item
  \textbf{CPU interconnection} is the communication between the
  mechanisms within the CPU.
\end{enumerate}

\includegraphics[width=4.52in,height=1.31in]{chap4_hardwaresoftware_files/figure-latex/mermaid-figure-1.png}

Now that we've delved into the intricacies of computer organization,
let's shift our focus to the broader components that constitute it.
Let's examine the contents contained within the confines of the
rectangular casing, such as: 1. The \textbf{motherboard} is a printed
circuit board (PCB) that contains the CPU and all other components,
serving as the main hub for the computer. 2. The \textbf{CPU}, as
aforementioned, is the core of the computer's processes to execute
instructions. 3. To facilitate the storage and preservation of the
memory, the computer has three (3) primary components:

\begin{itemize}
\tightlist
\item
  \emph{Random access memory (RAM)} is essentially the volatile memory
  of the computer. Memory is temporarily stored here for quick access
  and easy manipulation. When the computer powers off, data is cleared.
\item
  \emph{Solid state drive (SSD)} is a non-volatile memory, but it uses
  flash memory to provide quick access to frequently used data. This
  memory is still stored even when the computer is powered off.
\item
  \emph{Hard drive} is also a non-volatile memory that is high-capacity
  for long-term data retention. Files, documents, or applications are
  stored in this physical drive.
\end{itemize}

\begin{enumerate}
\def\labelenumi{\arabic{enumi}.}
\setcounter{enumi}{3}
\tightlist
\item
  Graphic Processing Units, Network Interface Cards, ports, power
  supplies, transistors, and chips, all of which fall under the umbrella
  of internal hardware.
\end{enumerate}

For the components to effectively acquire data to process and present
information, peripheral devices, or external hardware. These are the I/O
devices that serve as conduits for the exchange of data (Awati and
Rosencrance 2021). Common input devices include:

\begin{enumerate}
\def\labelenumi{\arabic{enumi}.}
\tightlist
\item
  A \textbf{keyboard} is used to input text, numbers, special
  characters, and commands. It has a set of keys that represent a
  character, symbol, or function such as navigation (up, down, left,
  right), modifier (shift, ctrl, cmd, alt), or others (media playback,
  brightness, etc)
\item
  A \textbf{mouse} is a pointing device that provides the user to
  navigate the interface and interact with elements with a cursor.
\item
  Microphones, cameras, touchpads, flash drives, and memory cards are
  also examples of input devices.
\end{enumerate}

Common output devices include (Gilster 2001):

\begin{enumerate}
\def\labelenumi{\arabic{enumi}.}
\tightlist
\item
  A \textbf{monitor} is a device akin to a television screen to display
  information graphically generated by the computer system.
\item
  A \textbf{printer} renders digital data into printed material.
\item
  A \textbf{speaker} serves as an audio output device.
\end{enumerate}

Now, let's transition our focus from hardware to the software, exploring
the vital elements that drive the functionality of computer systems
(Stallings 2016).

\section{Software}\label{software}

Software used to be in a rudimentary format, primarily designed for
specific numerical calculations or other simple data processing tasks.
To execute calculations, switches and circuits were manually set.
However, as computers grew smarter and transitioned from a
hardware-centric approach came the provision of the software (Englander
and Wong 2021; Keutzer 1994).

There are two (2) types of software in a computer: 1. \textbf{System
software} facilitates the connection between hardware components and the
application software. It manages the hardware components by executing
tasks such as maintaining or organizing data. It is designed to load
programs, move data to peripheral devices to perform computations, and
fundamentally, provide an interface for interacting with the computer
hardware. It is the infrastructure of the computer. 2.
\textbf{Application software} provides a platform for end user programs
of a computer system. The interface it provides allows users to directly
interact with the computer system through various programs that are
developed to meet the user requirements and preferences. This comes in
applications such as productivity, entertainment, communication, or any
specialized tasks. 3. \textbf{Libraries} are collections of reusable,
pre-written code, functions, and programs that can be used to perform
basic and common tasks without the need for writing the code from
scratch again. 4. \textbf{Utilities} are software programs that perform
tasks related to system management or maintenance. This usually comes in
the form of an interface or scripting language that the user has to
encode to carry out functions such as compressing files, system
diagnostics, or checking information about your device. 5.
\textbf{Middleware} is the intermediary layer between different computer
applications or components. It is a software that allows different
applications to interact with each other, exchanging data and
information, even if they are built in different frameworks. 6.
\textbf{Development Software} are the tools and programs that allow the
development, design, test, and maintenance of software applications.
Typically, it is an environment where code can be written, debugged, and
managed. An example of this would be an Integrated Development
Environments (IDEs). 7. \textbf{Database Management Systems (DBMS)} are
softwares that manage data. It comes with a provision of an interface
where users and applications can interact with databases through CRUD,
or ``create, read, update, and delete'' functions. DBMS is also in
charge of data integrity, security and other data handling methods
(Englander and Wong 2021; Abelson and Sussman 1996).

Figure visualizes how the hardware and software interact with each
other. Generally, end users do not need to concern themselves with the
low-level architecture of the computer as the computer system itself is
perceived as applications that they use. The operating system (OS)
serves to conceal the intricacies of the hardware from the end user. It
provides a convenient interface where applications and computer hardware
interact. Its primary objectives include optimizing the utilization of
the computer resources, and enhancing the user experience of the entire
computer system. Examples of these operating systems include Windows,
macOS, and Linux. In mobile systems, we also have the likes of iOS and
Android (Stallings 2016).

In essence, the OS allows the following functions:

\begin{itemize}
\tightlist
\item
  Instruction set architecture (ISA) is a repository of machine language
  instructions that a computer follows. This serves as the main boundary
  between hardware and software.
\item
  The OS allows the management and creation of programs or utilities. It
  loads the instructions and data into the main memory, initializes I/O
  devices by handling the instructions or control signals for the
  programs, and controlling the access to the entire system to name a
  few.
\end{itemize}

Now that we have a general understanding of what operating systems
typically do, let's delve into how software effectively communicates
with hardware. This communication involves the translation of
instructions expressed in programming languages, which serve as
intermediaries between the human-readable code crafted by programmers
and the machine code executed by the CPU.

\textbf{Programming languages} can be broadly categorized into two
types: \emph{high-level languages}, which are human-readable and
writable, and \emph{low-level languages}, encompassing machine and
assembly languages understood by the computer.

Additionally, there are also programming paradigms that are overarching
styles of how tasks can be structured and executed (Sebesta 2015; Cocca
2022).

Common paradigms include:

\begin{enumerate}
\def\labelenumi{\arabic{enumi}.}
\tightlist
\item
  \emph{Imperative Programming} consists of sets of instructions that
  explicitly dictates the computer what to do. It emphasizes on how the
  program should achieve a result through a step-by-step procedure.
\item
  \textbf{Declarative programming}, on the other hand, does not specify
  a control flow for the logic of a computation. It simply tells the
  program what result we are expecting. \textbf{Functional programming}
  is a declarative type that focuses on mathematical functions,
  immutability, and avoids changing-states. \textbf{Object-Oriented
  Programming (OOP)}, as its name suggests, organizes code into objects
  that are instances of classes. It focuses on several concepts such as
  inheritance, encapsulation, and polymorphism.
\end{enumerate}

Other paradigms are procedural and generic.

Here are some popular programming languages (Sebesta 2015; Cocca 2022):

\begin{itemize}
\tightlist
\item
  \textbf{Machine Language} is a low-level language that comprises
  binary numeric codes that represent computer-executable operations. It
  utilizes binary digits (0s and 1s), known as bits. These bits
  represent both instructions and data, translating the instructions and
  data directly into electronic signals. The computer understands this
  binary representation due to its electronic nature. In computing,
  information processing occurs through electronic circuits
  transitioning in two states: on (1) or off (0), so this binary
  representation in machine language reflects different electrical
  states within the computer's circuitry, indicating which operations
  should be performed, such as addition, subtraction, or data movement
  between memory locations.
\item
  \textbf{Assembly Language} is a low-level, imperative language that
  consists of mnemonic codes to represent operations and introduce the
  naming of memory blocks. For instance, if machine language code
  expresses addition as `01100110 00001010,' the corresponding assembly
  code line could read `ADD R2, R0, R1.' For those delving more deeply
  into computer architecture, mastering assembly language becomes
  crucial.
\item
  \textbf{Python} is a high-level, multi-paradigm (functional, OOP),
  general-purpose language that is distinctly readable and clear. It is
  an interpreted language, meaning instructions can be executed without
  a compiler. Python's code structure relies on indentation to define
  functions, loops, or classes. It is highly versatile and commonly used
  in web development, mathematics, and data analytics.
\item
  \textbf{C++} is another popular high-level, general-purpose,
  multi-paradigm (procedural, OOP) language that is actually an
  extension of another programming language, C. It is often used for
  creating large-scale applications as it provides a low-level access to
  memory. With this, it can actually be a mid-level language as it
  offers high-level abstraction, but also direct manipulation to the
  hardware resources.
\item
  \textbf{Structured Query Language (SQL)} is a high-level, declarative
  language that is used in managing and manipulating relational
  databases. It allows users to specifically detail what data to
  retrieve or change.
\end{itemize}

Now that we have explored the different types of programming languages,
each with its unique characteristics and applications, it becomes
evident that understanding fundamental programming concepts is the
universal foundation in effective software development.

\begin{enumerate}
\def\labelenumi{\arabic{enumi}.}
\item
  \textbf{Algorithms} are step-by-step procedures that solve specific
  problems or perform precise mathematical functions such as sorting,
  sequencing, etc. Algorithms are not the code itself, but rather, it is
  the logic on how a desired output will be produced with the given
  input. Key aspects of algorithms are time execution and data storage
  requirements as these will dictate its efficiency when processed.
  Examples would be sorting algorithms, searching algorithms, and
  recursive algorithms.
\item
  A \textbf{variable} is generally a memory location where you can store
  and retrieve data from. When declaring a variable, you often provide a
  short, but descriptive, name and assign a value it will hold.
\item
  \textbf{Data types} define what type of value a variable holds. It has
  different memory requirements, capacity, and operations that can be
  performed on it. Basic data types include:
\end{enumerate}

\begin{itemize}
\tightlist
\item
  \emph{Integers:} whole numbers such as -2, 0, or 1980.
\item
  \emph{Float:} real numbers with fractional components like 1.0, 2.3,
  or -12.232.
\item
  \emph{Boolean:} true or false, which can also be represented by 1 or
  0, respectively.
\item
  \emph{String:} text or sequence of characters like ``hello'' or
  ``hello\_world123''
\end{itemize}

\begin{enumerate}
\def\labelenumi{\arabic{enumi}.}
\setcounter{enumi}{3}
\tightlist
\item
  \textbf{Control Structure} determine the flow of the program
  execution. These are statements that allow the system to select an
  alternative control flow path or trigger a repeated execution of
  statements. Common control structures include: Selection statements
  which instructs the computer to make a division when given certain
  conditions e.g.~an if-then statement, or if-else statement. Iterative
  statements, or loops, cause a statement to be executed multiple times
  depending on the condition e.g.~for loops.
\end{enumerate}

This chapter provides a foundational glimpse into the complexity and the
symbiotic relationship of hardware and software. As we conclude this
chapter, it should have offered you a broad overview of their
functionalities within a hierarchical framework. Understanding hardware
does not only provide insights into its components but also spark
innovation for enhancing computer infrastructure. Additionally, since
software leverages the capabilities of the hardware to execute its
functions, you can examine which aspects of the hardware do we have to
develop for more robust programs.

At the same time, the limitless potential of software opens avenues for
a myriad of possibilities. Now that you have a foundational
understanding of its functions and structure, you have the ability to
pinpoint areas requiring enhancement. This knowledge becomes
instrumental in optimizing the field of information science by delving
into the very fabric of its infrastructure. Embracing this holistic
perspective positions us to maximize the potential inherent in our
understanding of both hardware and software components.

\section{Assessment}\label{assessment}

\begin{enumerate}
\def\labelenumi{\arabic{enumi}.}
\tightlist
\item
  Which component is considered the core element that drives the
  computational capabilities of a computer?
\end{enumerate}

\begin{enumerate}
\def\labelenumi{\alph{enumi})}
\tightlist
\item
  Main memory
\item
  Input devices
\item
  Processor
\item
  Solid State Drive (SSD)
\end{enumerate}

\begin{enumerate}
\def\labelenumi{\arabic{enumi}.}
\setcounter{enumi}{1}
\tightlist
\item
  True/False: The Arithmetic and Logic Unit (ALU) of the CPU performs
  both arithmetic (e.g., addition, subtraction) and logical (e.g., AND,
  OR) operations.
\item
  Briefly explain the symbiotic relationship between hardware and
  software in computing systems.
\item
  Define the hierarchical system in the context of complex computing
  systems. How does it help in understanding the organization of a
  computer?
\end{enumerate}

\begin{tcolorbox}[enhanced jigsaw, titlerule=0mm, rightrule=.15mm, bottomtitle=1mm, opacityback=0, bottomrule=.15mm, colbacktitle=quarto-callout-tip-color!10!white, opacitybacktitle=0.6, left=2mm, toprule=.15mm, colframe=quarto-callout-tip-color-frame, leftrule=.75mm, toptitle=1mm, colback=white, title=\textcolor{quarto-callout-tip-color}{\faLightbulb}\hspace{0.5em}{Critical Question}, breakable, coltitle=black, arc=.35mm]

Explain the significance of understanding the organization of a
computing system. How does this understanding contribute to advancements
in information science? Provide examples to support your argument.

\end{tcolorbox}

\subsection{Answers}\label{answers}

\begin{enumerate}
\def\labelenumi{\arabic{enumi}.}
\tightlist
\item
  C
\item
  True
\item
  Hardware and software work together in a symbiotic relationship, where
  hardware provides the physical structure and the necessary platform
  for the software to execute specific tasks, such as data processing or
  running applications.
\item
  The hierarchical system breaks down complexity into manageable levels.
  It helps the understand computer system by categorizing elements from
  general to specific, providing a systematic approach to examine both
  structure and function at different levels.
\end{enumerate}

\subsection{Bibliography}\label{bibliography}

\phantomsection\label{refs}
\begin{CSLReferences}{1}{0}
\bibitem[\citeproctext]{ref-abelson1996qc}
Abelson, Harold, and Gerald Jay Sussman. 1996. \emph{Structure and
Interpretation of Computer Programs}. 2nd ed. MIT Electrical Engineering
and Computer Science. London, England: MIT Press.

\bibitem[\citeproctext]{ref-aws_cdn}
Amazon Web Services. n.d. {``What Is a CDN (Content Delivery
Network)?''} n.d. \url{https://aws.amazon.com/what-is/cdn/}.

\bibitem[\citeproctext]{ref-aws_load_balancing}
---------. n.d. {``What Is Load Balancing?''} n.d.
\url{https://aws.amazon.com/what-is/load-balancing/}.

\bibitem[\citeproctext]{ref-amonoo2023stakeholder}
Amonoo Nkrumah, B., W. Qian, A. Kaur, and C. Tilt. 2023. {``Stakeholder
Accountability in the Era of Big Data: An Exploratory Study of Online
Platform Companies.''} \emph{Qualitative Research in Accounting \&
Management} 20 (4): 447--84.
\url{https://doi.org/10.1108/QRAM-03-2022-0042}.

\bibitem[\citeproctext]{ref-annunziato2020experience}
Annunziato, M. 2020. {``The {`Experience Age'} Is Already Here.''} The
Daily Campus. 2020.
\url{https://dailycampus.com/2020/10/09/the-experience-age-is-already-here/}.

\bibitem[\citeproctext]{ref-Awati2021bz}
Awati, Rahul, and Linda Rosencrance. 2021. {``Computer Hardware.''}
\url{https://www.techtarget.com/searchnetworking/definition/hardware};
TechTarget. October 2021.

\bibitem[\citeproctext]{ref-blinova2023corporatesustainability}
Blinova, E., T. Ponomarenko, and S. Tesovskaya. 2023. {``Key Corporate
Sustainability Assessment Methods for Coal Companies.''}
\emph{Sustainability (2071-1050)} 15 (7): 5763.
\url{https://doi.org/10.3390/su15075763}.

\bibitem[\citeproctext]{ref-busuena2023introduction}
Busueña, J., and J. Pomperada. 2023. \emph{Introduction to Information
Technology and Computer Fundamentals}. 2nd ed. Unlimited Books Library
Services; Publishing Inc.

\bibitem[\citeproctext]{ref-clarke2018individual}
Clarke, Blanaid. 2018. {``Individual Accountability in Irish Credit
Institutions---Lessons to Be Learned from the United Kingdom's Senior
Managers' Regime.''} \emph{Common Law World Review}, 35--52.

\bibitem[\citeproctext]{ref-cloudflare_vpc}
Cloudfare. n.d. {``What Is a Virtual Private Cloud (VPC)?''} n.d.
\url{https://www.cloudflare.com/learning/cloud/what-is-a-virtual-private-cloud/}.

\bibitem[\citeproctext]{ref-cocca2022dn}
Cocca, Germán. 2022. {``Programming Paradigms -- Paradigm Examples for
Beginners.''}
\url{https://www.freecodecamp.org/news/an-introduction-to-programming-paradigms/}.
May 2022.

\bibitem[\citeproctext]{ref-djordjevic2019corporate}
Đorđević, D. B., M. Vuković, S. Urošević, N. Štrbac, and A. Vuković.
2019. {``Studying the Corporate Social Responsibility in Apparel and
Textile Industry.''} \emph{Industria Textila} 70 (4): 336--41.
\url{https://doi.org/10.35530/IT.070.04.1572}.

\bibitem[\citeproctext]{ref-englander2021mq}
Englander, Irv, and Wilson Wong. 2021. \emph{The Architecture of
Computer Hardware, Systems Software, and Networking}. 6th ed. Nashville,
TN: John Wiley \& Sons.

\bibitem[\citeproctext]{ref-geeks_for_geeks_iot_challenges}
Geeks, Geeks for. 2023. {``Challenges in Internet of Things (IoT).''}
2023.
\url{https://www.geeksforgeeks.org/challenges-in-internet-of-things-iot/}.

\bibitem[\citeproctext]{ref-gillis2023iot}
Gillis, A. S. 2023. {``Internet of Things (IoT).''} 2023.
\url{https://www.techtarget.com/iotagenda/definition/Internet-of-Things-IoT}.

\bibitem[\citeproctext]{ref-gilster2001pc}
Gilster, Ron. 2001. \emph{{PC} Hardware: A Beginner's Guide}. New York:
Osborne/{McGraw}-Hill. \url{http://site.ebrary.com/id/10015274}.

\bibitem[\citeproctext]{ref-goodman2008information}
Goodman, Seymour, Detmar W. Straub, and Richard Baskerville. 2008.
\emph{Information Security: Policy, Processes, and Practices}.
Routledge.

\bibitem[\citeproctext]{ref-google_cloud_storage}
Google. n.d. {``What Is Cloud Storage?''} n.d.
\url{https://cloud.google.com/learn/what-is-cloud-storage}.

\bibitem[\citeproctext]{ref-hameed2023association}
Hameed, F., M. Alfaraj, and K. Hameed. 2023. {``The Association of Board
Characteristics and Corporate Social Responsibility Disclosure Quality:
Empirical Evidence from Pakistan.''} \emph{Sustainability (2071-1050)}
15 (24): 16849. \url{https://doi.org/10.3390/su152416849}.

\bibitem[\citeproctext]{ref-honigsberg2019individual}
Honigsberg, C. 2019. {``The Case for Individual Audit Partner
Accountability.''} \emph{Vanderbilt Law Review} 72 (6): 1871--1922.

\bibitem[\citeproctext]{ref-ibm_cloud_computing}
IBM. n.d. {``What Is Cloud Computing?''} n.d.
\url{https://www.ibm.com/topics/cloud-computing}.

\bibitem[\citeproctext]{ref-ibm_iot}
---------. n.d. {``What Is the Internet of Things (IoT)?''} n.d.
\url{https://www.ibm.com/topics/internet-of-things}.

\bibitem[\citeproctext]{ref-ivey2023iotapplications}
Ivey, A. 2023. {``7 Real-World IoT Applications and Examples.''} 2023.
\url{https://cointelegraph.com/news/7-iot-applications-and-examples}.

\bibitem[\citeproctext]{ref-jebaraj2023cloudcompanies}
Jebaraj, K. 2023. {``Top 10 Cloud Computing Companies of 2024.''} 2023.
\url{https://www.knowledgehut.com/blog/cloud-computing/top-cloud-computing-companies}.

\bibitem[\citeproctext]{ref-jeursen2022cover}
Jeursen, T. 2022. {``"Cover Your Ass": Individual Accountability, Visual
Documentation, and Everyday Policing in Miami.''} \emph{PoLAR: Political
\& Legal Anthropology Review} 45 (2): 186--200.
\url{https://doi.org/10.1111/plar.12505}.

\bibitem[\citeproctext]{ref-kang2022sustainabletraining}
Kang, Y.-C., H.-S. Hsiao, and J.-Y. Ni. 2022. {``The Role of Sustainable
Training and Reward in Influencing Employee Accountability Perception
and Behavior for Corporate Sustainability.''} \emph{Sustainability
(2071-1050)} 14 (18): 11589--N.PAG.
\url{https://doi.org/10.3390/su141811589}.

\bibitem[\citeproctext]{ref-keutzer1994he}
Keutzer, Kurt. 1994. {``Hardware/Software Co-Simulation.''} In
\emph{Proceedings of the 31st Annual Conference on Design Automation
Conference - {DAC} '94}. New York, New York, USA: ACM Press.

\bibitem[\citeproctext]{ref-lombardi2004information}
Lombardi, Olimpia. 2004. {``What Is Information?''} \emph{Foundations of
Science} 9: 105--34.
\url{https://doi.org/10.1023/B:FODA.0000025034.53313.7c}.

\bibitem[\citeproctext]{ref-lundin2023information}
Lundin, L. L. 2023. \emph{Information Security}. Salem Press
Encyclopedia.

\bibitem[\citeproctext]{ref-marr2023cloudtrends}
Marr, B. 2023. {``The 10 Biggest Cloud Computing Trends in 2024 Everyone
Must Be Ready for Now.''} 2023.
\url{https://www.forbes.com/sites/bernardmarr/2023/10/09/the-10-biggest-cloud-computing-trends-in-2024-everyone-must-be-ready-for-now/?sh=ea1da466d672}.

\bibitem[\citeproctext]{ref-microsoft2023securedata}
Microsoft. 2023. {``11 Best Practices for Securing Data in Cloud
Services.''} 2023.
\url{https://www.microsoft.com/en-us/security/blog/2023/07/05/11-best-practices-for-securing-data-in-cloud-services/}.

\bibitem[\citeproctext]{ref-muts2024iot}
Muts, I. 2024. {``10+ Best IoT Cloud Platforms in 2024.''} 2024.
\url{https://euristiq.com/best-iot-cloud-platforms/}.

\bibitem[\citeproctext]{ref-sebesta2015ek}
Sebesta, Robert W. 2015. \emph{Concepts of Programming Languages}. 11th
ed. Upper Saddle River, NJ: Pearson.

\bibitem[\citeproctext]{ref-stallings2016computer}
Stallings, William. 2016. \emph{Computer Organization and Architecture}.
10th edition. Pearson.

\bibitem[\citeproctext]{ref-sugandhi2023iotfuture}
Sugandhi, A. 2023. {``The Future of IoT: Trends and Predictions for
2024.''} 2023.
\url{https://www.knowledgehut.com/blog/web-development/iot-future}.

\bibitem[\citeproctext]{ref-velazquez2022iot}
Velazquez, R. 2022. {``IoT: The Internet of Things. What Is the Internet
of Things? How Does IoT Work?''} 2022.
\url{https://builtin.com/internet-things}.

\bibitem[\citeproctext]{ref-weiser1999og}
Weiser, Mark. 1999. {``The Computer for the 21 \(^{st}\) Century.''}
\emph{ACM SIGMOBILE Mob. Comput. Commun. Rev.} 3 (3): 3--11.

\bibitem[\citeproctext]{ref-williams2012using}
Williams, B. K., and S. C. Sawyer. 2012. \emph{Using Information
Technology: Introductory Edition}. McGraw-Hill Europe.

\bibitem[\citeproctext]{ref-zhang2023environmental}
Zhang, Y., M. Imeni, and S. A. Edalatpanah. 2023. {``Environmental
Dimension of Corporate Social Responsibility and Earnings Persistence:
An Exploration of the Moderator Roles of Operating Efficiency and
Financing Cost.''} \emph{Sustainability (2071-1050)} 15 (20): 14814.
\url{https://doi.org/10.3390/su152014814}.

\end{CSLReferences}

\bookmarksetup{startatroot}

\chapter{Database Systems and Data
Management}\label{database-systems-and-data-management}

\bookmarksetup{startatroot}

\chapter{Database Systems and Data
Management}\label{database-systems-and-data-management-1}

\emph{By the end of this section, you should be able to:}

\begin{itemize}
\tightlist
\item
  Define Database Systems and Data Management
\item
  Familiarize of how a database system works
\item
  Explain the importance of an effective data management for an
  organization
\end{itemize}

\section{Data and Data Management}\label{data-and-data-management}

Data is present everywhere. Whatever we see, consume, hear, feel, smell
can be considered data. According to Ackoff R.L., (1989) data is
\emph{``as products of observation''} while information is considered
``in descriptions'' (R.L. Ackoff, 1989). A cuneiform tablet for instance
can be considered data and information at the same time. Being a slab of
stone with unintelligible writing is an observation or data that is
subject for further investigation while for paleographers it is
considered already as information because of their capacity to interpret
it in descriptions. Wisdom being as \emph{``the ability to increase
effectiveness''} (R.L. Ackoff, 1989) is how it is important to preserve
these cuneiform as it is equal in preserving knowledge and history.
Bates argues that information ``does not consist of matter and energy
but\ldots{} pattern of organization of matter and energy.''

Data is the entity that is present among our reality and what we can
make sense about. We may choose or ignore data depending if it is a
necessity or not. These are raw facts, figures, or entities that can be
subject for storage for such utility, hence, such organizations collect
a particular set of data that may help them for decisions for their
operations. Part of the operations is what we call data management. Data
management, according to the definition of Oracle Philippines (2024),
\emph{``is the practice of collecting, keeping, and using data securely,
efficiently, and cost-effectively.''} Moreover, \textbf{data management}
is a process in which information is acquired, stored, retrieved, and
utilized within an organization. This involves practices to ensure data
throughout its life cycle. Data management consists of the following
tasks:

\begin{itemize}
\tightlist
\item
  *\textbf{Data capture:} the process of collection of data
\item
  \textbf{Data classification:} the process of classification and
  segregation of data
\item
  \textbf{Data storage:} segregated data is then stored
\item
  \textbf{Data arranging:} data has to be arranged for proper storage
\item
  \textbf{Data retrieval:} data is eventually pulled out every now and
  then whenever needed, hence, there must be indexing for an easier
  retrieval.
\item
  \textbf{Data maintenance:} is the tasks in keeping the data to be
  up-to-date
\item
  \textbf{Data Verification:} the process of checking error in data
  before its storage
\item
  \textbf{Data Coding:} for easier reference, data undergoes coding
\item
  \textbf{Data Editing:} modification of data for presentation
\item
  \textbf{Data transcription:} conversion of data from one form into
  another.
\item
  \textbf{Data transmission:} transmission or transfer of data where it
  will be forwarded to another place for other functions.
\end{itemize}

Through effective data management, high-quality data is secured and this
is beneficial for the organization. Some of which are: (1) Improve
decision making, (2) increase Customer Satisfaction, (3) Increase Sales,
(4) Improve Innovation, (5) Raise Productivity, and (6) Ensure
Compliance. Data being organized and processed will have an additional
value beyond its intrinsic value. This collection of data is called
information. According to\ldots{} the nine characteristics of quality
formation are being 1. Accessible, 2. Accurate, 3. Complete, 4.
Economical, 5. Relevant, 6. Reliable, 7. Secure,. 8. Timely, 9.
Verifiable.

\section{Data Lifecycle:}\label{data-lifecycle}

\subsection{Database Systems}\label{database-systems}

\textbf{A. What is a Database?}

A database is a collection of data which is carefully organized. It
helps an organization achieve its goal by providing information for
managers/ decision makers. It is an important component of information
systems that allows the storage, query data, and update. Database uses
tables for the organization of information. Data with different elements
are then related to each other .

\textbf{B. Database Management System}

Database Management System (DBMS)- a group of programs ``used to access
and manage a database as well as provide an interface between the
database and its users and other application processes''. (Reynolds \&
Stair, 2021)

Use of High-Quality Data and Database Technology - Companies and
organizations take advantage of database technology for analysis and
decision-making. Data being used is then organized and consolidated to
aid the organization in formulating decisions, thus, increasing
productivity , output or profit.

\subsection{Advantages and Disadvantages of
DBMS}\label{advantages-and-disadvantages-of-dbms}

\subsubsection{Advantages}\label{advantages}

\begin{itemize}
\tightlist
\item
  \textbf{Controlling of Redundancy} - prevention of duplication of
  data.
\item
  \textbf{Improved Data Sharing}
\item
  \textbf{Data integrity} - centralized control of data allows
  monitoring the integrity of data in the database.
\item
  \textbf{Security} - having a complete authority allows the Database
  System to ensure access only through its proper channels.
\item
  \textbf{Data Consistency} - involves the elimination of data
  redundancy
\item
  \textbf{Efficient Data Access}
\item
  \textbf{Enforcement of Standards} - since data is centralized, data
  standards such as naming convention and data quality standards can be
  established.
\item
  \textbf{Data Independence}
\item
  \textbf{Reduced Application Development and Maintenance Time}
\end{itemize}

\subsubsection{Disadvantages:}\label{disadvantages}

\begin{itemize}
\tightlist
\item
  \textbf{Complexity} -designers and developers require knowledge about
  software because it is multi-functional.
\item
  \textbf{Demand for large memory} - because of its complex nature,
  large memory is needed.
\item
  \textbf{Centralized system}
\item
  \textbf{DBMS are generalized software.}
\end{itemize}

\section{Hypothetical Situation:}\label{hypothetical-situation}

Consider this scenario, a retail company utilizes a DBMS to have a
better operation and provide better customer experience. For example,
ITC retail faces a ;lot of challenges like inventory management,
tracking of customer purchase, and analysis of sales trends. ITC retail
then implements a DBMS to streamline processes, imr\prove data accuracy,
and gain valuable insights from consumer behavior. In order to make this
possible, ITC retail deployed a relational database system to manage its
inventory, sales, and customer information. Since it has a lot of
branches, each store has a local database that synchronizes with a
central database in real-time. With this method, consistent and updated
information all throughout the stores are being provided for.

\section{Conclusion:}\label{conclusion-1}

Database systems allow the capturing of data for it to be modified,
stored, updated and be subject to analysis and decision-making. Since
data may present us with variables that help organizations to identify
factors that may help them improve operations or formulate strategy. The
integration of DBMS and effective data management has become integral
for modern organizations and firms especially in the midst of the
complexities of the information age. Businesses from various sectors
utilize DBMS to centralize, organize, consolidate data for strategic
decision-making, security, provide better customer service, and a more
efficient operation. Moreover, database and data management does not
only improve internal operations but also foster innovation that enables
them to adapt in an ever-changing market landscape. DBMS will remain as
a tool to unlock the full potential of organizations where information
is an important driver for innovation and success.

\section{Critical Thinking
Questions:}\label{critical-thinking-questions-6}

\begin{enumerate}
\def\labelenumi{\arabic{enumi}.}
\tightlist
\item
  What are the primary considerations for a firm/ organization should a
  database be utilized?
\item
  What could be the repercussions of moving to a database as a service
  environment?
\end{enumerate}

\section{References}\label{references-3}

\phantomsection\label{refs}
\begin{CSLReferences}{1}{0}
\bibitem[\citeproctext]{ref-abelson1996qc}
Abelson, Harold, and Gerald Jay Sussman. 1996. \emph{Structure and
Interpretation of Computer Programs}. 2nd ed. MIT Electrical Engineering
and Computer Science. London, England: MIT Press.

\bibitem[\citeproctext]{ref-aws_cdn}
Amazon Web Services. n.d. {``What Is a CDN (Content Delivery
Network)?''} n.d. \url{https://aws.amazon.com/what-is/cdn/}.

\bibitem[\citeproctext]{ref-aws_load_balancing}
---------. n.d. {``What Is Load Balancing?''} n.d.
\url{https://aws.amazon.com/what-is/load-balancing/}.

\bibitem[\citeproctext]{ref-amonoo2023stakeholder}
Amonoo Nkrumah, B., W. Qian, A. Kaur, and C. Tilt. 2023. {``Stakeholder
Accountability in the Era of Big Data: An Exploratory Study of Online
Platform Companies.''} \emph{Qualitative Research in Accounting \&
Management} 20 (4): 447--84.
\url{https://doi.org/10.1108/QRAM-03-2022-0042}.

\bibitem[\citeproctext]{ref-annunziato2020experience}
Annunziato, M. 2020. {``The {`Experience Age'} Is Already Here.''} The
Daily Campus. 2020.
\url{https://dailycampus.com/2020/10/09/the-experience-age-is-already-here/}.

\bibitem[\citeproctext]{ref-Awati2021bz}
Awati, Rahul, and Linda Rosencrance. 2021. {``Computer Hardware.''}
\url{https://www.techtarget.com/searchnetworking/definition/hardware};
TechTarget. October 2021.

\bibitem[\citeproctext]{ref-blinova2023corporatesustainability}
Blinova, E., T. Ponomarenko, and S. Tesovskaya. 2023. {``Key Corporate
Sustainability Assessment Methods for Coal Companies.''}
\emph{Sustainability (2071-1050)} 15 (7): 5763.
\url{https://doi.org/10.3390/su15075763}.

\bibitem[\citeproctext]{ref-busuena2023introduction}
Busueña, J., and J. Pomperada. 2023. \emph{Introduction to Information
Technology and Computer Fundamentals}. 2nd ed. Unlimited Books Library
Services; Publishing Inc.

\bibitem[\citeproctext]{ref-clarke2018individual}
Clarke, Blanaid. 2018. {``Individual Accountability in Irish Credit
Institutions---Lessons to Be Learned from the United Kingdom's Senior
Managers' Regime.''} \emph{Common Law World Review}, 35--52.

\bibitem[\citeproctext]{ref-cloudflare_vpc}
Cloudfare. n.d. {``What Is a Virtual Private Cloud (VPC)?''} n.d.
\url{https://www.cloudflare.com/learning/cloud/what-is-a-virtual-private-cloud/}.

\bibitem[\citeproctext]{ref-cocca2022dn}
Cocca, Germán. 2022. {``Programming Paradigms -- Paradigm Examples for
Beginners.''}
\url{https://www.freecodecamp.org/news/an-introduction-to-programming-paradigms/}.
May 2022.

\bibitem[\citeproctext]{ref-djordjevic2019corporate}
Đorđević, D. B., M. Vuković, S. Urošević, N. Štrbac, and A. Vuković.
2019. {``Studying the Corporate Social Responsibility in Apparel and
Textile Industry.''} \emph{Industria Textila} 70 (4): 336--41.
\url{https://doi.org/10.35530/IT.070.04.1572}.

\bibitem[\citeproctext]{ref-englander2021mq}
Englander, Irv, and Wilson Wong. 2021. \emph{The Architecture of
Computer Hardware, Systems Software, and Networking}. 6th ed. Nashville,
TN: John Wiley \& Sons.

\bibitem[\citeproctext]{ref-geeks_for_geeks_iot_challenges}
Geeks, Geeks for. 2023. {``Challenges in Internet of Things (IoT).''}
2023.
\url{https://www.geeksforgeeks.org/challenges-in-internet-of-things-iot/}.

\bibitem[\citeproctext]{ref-gillis2023iot}
Gillis, A. S. 2023. {``Internet of Things (IoT).''} 2023.
\url{https://www.techtarget.com/iotagenda/definition/Internet-of-Things-IoT}.

\bibitem[\citeproctext]{ref-gilster2001pc}
Gilster, Ron. 2001. \emph{{PC} Hardware: A Beginner's Guide}. New York:
Osborne/{McGraw}-Hill. \url{http://site.ebrary.com/id/10015274}.

\bibitem[\citeproctext]{ref-goodman2008information}
Goodman, Seymour, Detmar W. Straub, and Richard Baskerville. 2008.
\emph{Information Security: Policy, Processes, and Practices}.
Routledge.

\bibitem[\citeproctext]{ref-google_cloud_storage}
Google. n.d. {``What Is Cloud Storage?''} n.d.
\url{https://cloud.google.com/learn/what-is-cloud-storage}.

\bibitem[\citeproctext]{ref-hameed2023association}
Hameed, F., M. Alfaraj, and K. Hameed. 2023. {``The Association of Board
Characteristics and Corporate Social Responsibility Disclosure Quality:
Empirical Evidence from Pakistan.''} \emph{Sustainability (2071-1050)}
15 (24): 16849. \url{https://doi.org/10.3390/su152416849}.

\bibitem[\citeproctext]{ref-honigsberg2019individual}
Honigsberg, C. 2019. {``The Case for Individual Audit Partner
Accountability.''} \emph{Vanderbilt Law Review} 72 (6): 1871--1922.

\bibitem[\citeproctext]{ref-ibm_cloud_computing}
IBM. n.d. {``What Is Cloud Computing?''} n.d.
\url{https://www.ibm.com/topics/cloud-computing}.

\bibitem[\citeproctext]{ref-ibm_iot}
---------. n.d. {``What Is the Internet of Things (IoT)?''} n.d.
\url{https://www.ibm.com/topics/internet-of-things}.

\bibitem[\citeproctext]{ref-ivey2023iotapplications}
Ivey, A. 2023. {``7 Real-World IoT Applications and Examples.''} 2023.
\url{https://cointelegraph.com/news/7-iot-applications-and-examples}.

\bibitem[\citeproctext]{ref-jebaraj2023cloudcompanies}
Jebaraj, K. 2023. {``Top 10 Cloud Computing Companies of 2024.''} 2023.
\url{https://www.knowledgehut.com/blog/cloud-computing/top-cloud-computing-companies}.

\bibitem[\citeproctext]{ref-jeursen2022cover}
Jeursen, T. 2022. {``"Cover Your Ass": Individual Accountability, Visual
Documentation, and Everyday Policing in Miami.''} \emph{PoLAR: Political
\& Legal Anthropology Review} 45 (2): 186--200.
\url{https://doi.org/10.1111/plar.12505}.

\bibitem[\citeproctext]{ref-kang2022sustainabletraining}
Kang, Y.-C., H.-S. Hsiao, and J.-Y. Ni. 2022. {``The Role of Sustainable
Training and Reward in Influencing Employee Accountability Perception
and Behavior for Corporate Sustainability.''} \emph{Sustainability
(2071-1050)} 14 (18): 11589--N.PAG.
\url{https://doi.org/10.3390/su141811589}.

\bibitem[\citeproctext]{ref-keutzer1994he}
Keutzer, Kurt. 1994. {``Hardware/Software Co-Simulation.''} In
\emph{Proceedings of the 31st Annual Conference on Design Automation
Conference - {DAC} '94}. New York, New York, USA: ACM Press.

\bibitem[\citeproctext]{ref-lombardi2004information}
Lombardi, Olimpia. 2004. {``What Is Information?''} \emph{Foundations of
Science} 9: 105--34.
\url{https://doi.org/10.1023/B:FODA.0000025034.53313.7c}.

\bibitem[\citeproctext]{ref-lundin2023information}
Lundin, L. L. 2023. \emph{Information Security}. Salem Press
Encyclopedia.

\bibitem[\citeproctext]{ref-marr2023cloudtrends}
Marr, B. 2023. {``The 10 Biggest Cloud Computing Trends in 2024 Everyone
Must Be Ready for Now.''} 2023.
\url{https://www.forbes.com/sites/bernardmarr/2023/10/09/the-10-biggest-cloud-computing-trends-in-2024-everyone-must-be-ready-for-now/?sh=ea1da466d672}.

\bibitem[\citeproctext]{ref-microsoft2023securedata}
Microsoft. 2023. {``11 Best Practices for Securing Data in Cloud
Services.''} 2023.
\url{https://www.microsoft.com/en-us/security/blog/2023/07/05/11-best-practices-for-securing-data-in-cloud-services/}.

\bibitem[\citeproctext]{ref-muts2024iot}
Muts, I. 2024. {``10+ Best IoT Cloud Platforms in 2024.''} 2024.
\url{https://euristiq.com/best-iot-cloud-platforms/}.

\bibitem[\citeproctext]{ref-sebesta2015ek}
Sebesta, Robert W. 2015. \emph{Concepts of Programming Languages}. 11th
ed. Upper Saddle River, NJ: Pearson.

\bibitem[\citeproctext]{ref-stallings2016computer}
Stallings, William. 2016. \emph{Computer Organization and Architecture}.
10th edition. Pearson.

\bibitem[\citeproctext]{ref-sugandhi2023iotfuture}
Sugandhi, A. 2023. {``The Future of IoT: Trends and Predictions for
2024.''} 2023.
\url{https://www.knowledgehut.com/blog/web-development/iot-future}.

\bibitem[\citeproctext]{ref-velazquez2022iot}
Velazquez, R. 2022. {``IoT: The Internet of Things. What Is the Internet
of Things? How Does IoT Work?''} 2022.
\url{https://builtin.com/internet-things}.

\bibitem[\citeproctext]{ref-weiser1999og}
Weiser, Mark. 1999. {``The Computer for the 21 \(^{st}\) Century.''}
\emph{ACM SIGMOBILE Mob. Comput. Commun. Rev.} 3 (3): 3--11.

\bibitem[\citeproctext]{ref-williams2012using}
Williams, B. K., and S. C. Sawyer. 2012. \emph{Using Information
Technology: Introductory Edition}. McGraw-Hill Europe.

\bibitem[\citeproctext]{ref-zhang2023environmental}
Zhang, Y., M. Imeni, and S. A. Edalatpanah. 2023. {``Environmental
Dimension of Corporate Social Responsibility and Earnings Persistence:
An Exploration of the Moderator Roles of Operating Efficiency and
Financing Cost.''} \emph{Sustainability (2071-1050)} 15 (20): 14814.
\url{https://doi.org/10.3390/su152014814}.

\end{CSLReferences}

\bookmarksetup{startatroot}

\chapter{Cloud Computing and the Internet of
Things}\label{cloud-computing-and-the-internet-of-things}

\section{Cloud Computing}\label{cloud-computing}

\subsection{Learning Objectives}\label{learning-objectives-4}

By the end of this section, you should be able to:

\begin{itemize}
\item
  Explain the differences between Infrastructure as a Service (IaaS),
  Platform as a Service (PaaS), and Software as a Service (SaaS).
  Provide examples of scenarios where each model would be beneficial.
\item
  Evaluate a hypothetical business scenario and recommend whether a
  public, private, hybrid, or community cloud deployment would be most
  suitable. Justify your choice based on the specific needs of the
  business.
\item
  Identify and describe three key practices for ensuring the security of
  data in the cloud. Discuss how these practices contribute to
  maintaining the integrity and confidentiality of digital assets.
\end{itemize}

Compare and contrast the services offered by two major cloud service *
providers, such as Amazon Web Services (AWS) and Microsoft Azure.
Analyze factors that might influence a business's decision to choose one
over the other.

\begin{itemize}
\tightlist
\item
  Summarize the implications of edge computing, the use of multiple
  cloud services, and quantum computing in the context of cloud
  computing. Discuss potential benefits and challenges associated with
  these emerging trends.
\end{itemize}

\subsection{What is Cloud Computing?}\label{what-is-cloud-computing}

Let's dive into the cool world of cloud computing. It's like renting
super-smart computers and storage space, but instead of having them in
your room, they're on the internet. Neat, huh?

Now, there are three main types of this cloud magic:

\begin{itemize}
\tightlist
\item
  \textbf{Infrastructure as a Service (IaaS):}
\end{itemize}

This one's like getting the basic building blocks of computing --
virtual machines, storage, and networking stuff. It's for those who want
control over the nitty-gritty without dealing with actual physical
hardware.

\begin{itemize}
\tightlist
\item
  \textbf{Platform as a Service (PaaS):}
\end{itemize}

Imagine a platform that takes care of all the hard techy stuff so you
can focus on coding and making awesome apps. That's PaaS for you --
making life easier for the creative minds.

\begin{itemize}
\tightlist
\item
  \textbf{Software as a Service (SaaS):}
\end{itemize}

SaaS is like having apps delivered to you over the internet. No need to
install anything -- just use it straight from the cloud. Think Google
Docs or Netflix -- all online, all the time. Now, these cloud things
aren't one-size-fits-all. We've got different flavors:

\begin{itemize}
\item
  \textbf{Public Cloud:}

  \begin{itemize}
  \tightlist
  \item
    It's like the town square of clouds -- open to everyone. Third-party
    folks manage it, making it budget-friendly and scalable for any
    business.
  \end{itemize}
\item
  \textbf{Private Cloud:}

  \begin{itemize}
  \tightlist
  \item
    This one's like your VIP section -- exclusive to one group. Great
    for those who need extra security and control over their digital
    space.
  \end{itemize}
\item
  \textbf{Hybrid Cloud:}

  \begin{itemize}
  \tightlist
  \item
    A mix of public and private -- get the best of both worlds. Scale up
    in the public cloud and keep your top-secret stuff private. Smart,
    huh?
  \end{itemize}
\item
  \textbf{Community Cloud:}

  \begin{itemize}
  \tightlist
  \item
    Sharing is caring! It's like a cloud hangout for a bunch of
    organizations with similar interests. Everyone pitches in, and
    everyone benefits.
  \end{itemize}
\end{itemize}

In a nutshell, cloud computing is all about flexing your tech muscles
without the hassle. It's like having a digital playground where you can
pick and choose what you need. So, whether you're into tech or just
curious, cloud computing is where the future's at!

(IBM n.d.)

\subsection{Big Cloud Companies:}\label{big-cloud-companies}

Alright, when we say big, we mean BIG. We've got the tech giants
stepping up to the plate:

\begin{itemize}
\item
  \textbf{Amazon (AWS):}

  \begin{itemize}
  \tightlist
  \item
    They practically invented this cloud stuff. Amazon Web Services
    (AWS) is like the OG cloud playground.
  \end{itemize}
\item
  \textbf{Microsoft (Azure):}

  \begin{itemize}
  \tightlist
  \item
    You know Windows, right? Well, Microsoft's Azure is their cloud
    service, and it's a heavyweight in the game.
  \end{itemize}
\item
  \textbf{Google (GCP):}

  \begin{itemize}
  \tightlist
  \item
    Google doesn't just do search engines; they've got the Google Cloud
    Platform (GCP) too. Imagine your favorite Google services, but for
    businesses.
  \end{itemize}
\item
  \textbf{IBM:}

  \begin{itemize}
  \tightlist
  \item
    Yup, the computer legends are in the cloud game too. IBM brings its
    tech expertise to the cloud scene.
  \end{itemize}
\item
  \textbf{Oracle:}

  \begin{itemize}
  \tightlist
  \item
    Not just a database wizard -- Oracle is also a big shot in cloud
    services. They handle some serious data power.
  \end{itemize}
\end{itemize}

These companies are like the rockstars of the cloud world, each bringing
its own vibe to the digital concert.

(Jebaraj 2023)

\subsection{Keeping Cloud Data Safe:}\label{keeping-cloud-data-safe}

Okay, so you've got your digital stuff up in the cloud, but how do you
make sure it's locked down tight? Here's the lowdown:

\begin{itemize}
\item
  \textbf{Encrypting Data:}

  \begin{itemize}
  \tightlist
  \item
    Think of encryption as putting your data in a super-secret code.
    Only those with the secret handshake (or key) can unlock and
    understand it. It's like having your own language for your digital
    diary.
  \end{itemize}
\item
  \textbf{Managing Who Can Access Data (Identity and Access
  Management):}

  \begin{itemize}
  \tightlist
  \item
    Imagine you're throwing a party, and you only want your pals to
    join. Identity and Access Management (IAM) is like the bouncer at
    the digital door. It decides who gets in and who stays out. Your
    data, your rules!
  \end{itemize}
\item
  \textbf{Following Rules and Best Practices for Security:}

  \begin{itemize}
  \tightlist
  \item
    Every digital world has its rules, and cloud security is no
    different. It's not just about keeping things safe; it's about
    following the playbook. Best practices for security are like the
    cheat codes to level up your data protection game.
  \end{itemize}
\end{itemize}

So, whether it's AWS, Azure, GCP, IBM, or Oracle, and no matter what
you're storing in the cloud -- from secret recipes to next-level gaming
ideas -- keeping it safe is the name of the game. Encrypt it, control
who gets the keys, and always play by the rules. That way, your digital
world stays your own personal fortress.

(Microsoft 2023)

\subsection{Storing Data in the Cloud:}\label{storing-data-in-the-cloud}

Alright, so you've got your data in the cloud, but not all data is the
same. It's like sorting your room -- different things, different places:

\begin{itemize}
\item
  \textbf{Objects:}

  \begin{itemize}
  \tightlist
  \item
    Think of these like digital treasure chests. You toss your files,
    images, and data into these containers, making it easy to manage and
    access.
  \end{itemize}
\item
  \textbf{Blocks:}

  \begin{itemize}
  \tightlist
  \item
    Imagine your data as a bunch of building blocks. These blocks stack
    up, creating the foundation for your digital creations. It's like
    building a virtual Lego masterpiece.
  \end{itemize}
\item
  \textbf{Files:}

  \begin{itemize}
  \tightlist
  \item
    This one's like your traditional filing cabinet. You organize your
    documents neatly, making it easy to find that homework assignment or
    cat video you saved.
  \end{itemize}
\end{itemize}

Plus, we've got Content Delivery Networks (CDN) in the mix. They're like
the express delivery service of the digital world, making sure your data
gets to you lightning-fast.

(Google n.d.)

\subsection{Connecting in the Cloud:}\label{connecting-in-the-cloud}

Now, let's talk about making connections in this vast digital landscape:

\begin{itemize}
\item
  \textbf{Virtual Private Cloud (VPC) (Cloudfare n.d.):}

  \begin{itemize}
  \tightlist
  \item
    It's like having your private corner in the digital playground. VPC
    lets you create your own network, keeping your stuff away from the
    digital nosy neighbors.
  \end{itemize}
\item
  \textbf{Load Balancing (Amazon Web Services n.d.):}

  \begin{itemize}
  \tightlist
  \item
    Imagine you're juggling a bunch of balls; you want to make sure none
    drop, right? Load balancing is like the expert juggler, evenly
    distributing the workload among different servers. It keeps
    everything running smoothly.
  \end{itemize}
\item
  \textbf{Content Delivery Networks (CDN) (Amazon Web Services n.d.):}

  \begin{itemize}
  \tightlist
  \item
    CDNs not only deliver data quickly but also help speed up the
    delivery of content, like images and videos. It's like getting your
    favorite snacks delivered to your doorstep ASAP.
  \end{itemize}
\end{itemize}

\subsection{New Trends in Cloud
Computing:}\label{new-trends-in-cloud-computing}

Now, let's check out what's cooking in the cloud kitchen:

\begin{itemize}
\item
  \textbf{Edge Computing:}

  \begin{itemize}
  \tightlist
  \item
    Picture this -- instead of sending all your data to a far-off cloud,
    you process it right where you need it. Edge computing brings the
    power closer to home, making things faster and more efficient.
  \end{itemize}
\item
  \textbf{Using Multiple Cloud Services Together:}

  \begin{itemize}
  \tightlist
  \item
    It's like having a buffet of digital tools. You pick and choose what
    works best for you. Maybe some AWS for storage, Azure for apps --
    mix and match for the ultimate tech combo.
  \end{itemize}
\item
  \textbf{Quantum Computing Exploration:}

  \begin{itemize}
  \tightlist
  \item
    Brace yourselves, tech adventurers! Quantum computing is like
    entering a whole new dimension. It's super advanced and can solve
    problems that regular computers struggle with. We're talking about
    next-level computing power here.
  \end{itemize}
\end{itemize}

So, from organizing your digital treasures to creating your private
digital space and riding the waves of cutting-edge tech, the cloud is
more than just a storage locker -- it's a digital playground full of
endless possibilities.

(Marr 2023)

\section{Internet of Things (IoT):}\label{internet-of-things-iot}

By the end of this section, you should be able to:

\begin{itemize}
\item
  Define the fundamental concepts of the Internet of Things, including
  ``Things,'' Connectivity, Data Processing, and User Interface. Provide
  real-world examples to illustrate these concepts.
\item
  Compare and contrast at least three communication protocols used in
  IoT, such as MQTT, CoAP, and HTTP/HTTPS. Explain the significance of
  these protocols in facilitating effective communication among IoT
  devices.
\item
  Outline three security measures that can be implemented to protect IoT
  devices and networks. Discuss the importance of respecting user
  privacy in the design and deployment of IoT solutions.
\item
  Investigate and present a case study on a specific industry or sector
  leveraging IoT technology. Highlight the practical uses, benefits, and
  potential challenges faced in implementing IoT solutions within that
  context.
\item
  Explore the implications of 5G networks, the integration of AI and
  machine learning in IoT, and the concept of edge computing. Assess how
  these trends may shape the future landscape of IoT and its
  applications.
\end{itemize}

\subsection{What is IoT? (IBM n.d.)}\label{what-is-iot-ibm_iot}

So, imagine your fridge, your watch, even your toaster, all chatting
away on the internet. That's IoT -- connecting everyday things to the
digital dance floor. Here are the main players:

\begin{itemize}
\item
  \textbf{Things:}

  \begin{itemize}
  \tightlist
  \item
    These are your everyday objects -- from your thermostat to your
    coffee maker -- getting a digital upgrade.
  \end{itemize}
\item
  \textbf{Connectivity:}

  \begin{itemize}
  \tightlist
  \item
    This is how these things talk to each other and to the internet.
    It's like the Wi-Fi wizardry making it all possible.
  \end{itemize}
\item
  \textbf{Data Processing:}

  \begin{itemize}
  \tightlist
  \item
    All the info these things generate needs sorting and making sense
    of. That's where the brainy part comes in, processing the data so
    it's useful.
  \end{itemize}
\item
  \textbf{User Interface:}

  \begin{itemize}
  \tightlist
  \item
    And of course, you need a way to interact with all this cool tech.
    The user interface is like the control panel, making it easy for you
    to boss around your smart devices.
  \end{itemize}
\end{itemize}

\subsection{How IoT Works (Velazquez
2022):}\label{how-iot-works-velazquez2022iot}

Alright, now let's peek behind the curtain and see how the IoT magic
happens:

\begin{itemize}
\item
  \textbf{Devices:}

  \begin{itemize}
  \tightlist
  \item
    These are your smart gadgets -- sensors, cameras, you name it --
    collecting data and doing cool stuff.
  \end{itemize}
\item
  \textbf{Connectivity:}

  \begin{itemize}
  \tightlist
  \item
    The devices need to talk to each other and to the digital cloud.
    This is where connectivity comes in, making sure everyone's on the
    same digital wavelength.
  \end{itemize}
\item
  \textbf{Cloud:}

  \begin{itemize}
  \tightlist
  \item
    All the data generated by your smart things goes to the cloud for
    processing and storage. It's like the grand data hub where
    everything comes together.
  \end{itemize}
\item
  \textbf{Applications:}

  \begin{itemize}
  \tightlist
  \item
    Finally, the applications are like the digital maestros,
    orchestrating how your smart devices behave and what they do.
  \end{itemize}
\end{itemize}

\subsection{Talking in IoT (Gillis
2023):}\label{talking-in-iot-gillis2023iot}

Now, let's chat about how these smart devices communicate -- it's like
they've got their own language:

\begin{itemize}
\item
  \textbf{MQTT, CoAP, HTTP/HTTPS, AMQP:}

  \begin{itemize}
  \tightlist
  \item
    These are like the secret codes they use to exchange information.
    Think of them as the languages your gadgets speak to each other.
  \end{itemize}
\item
  \textbf{LoRaWAN, Zigbee:}

  \begin{itemize}
  \tightlist
  \item
    And these are like special dialects for specific devices -- making
    sure your smart thermostat understands your smart doorbell without
    any confusion.
  \end{itemize}
\end{itemize}

\subsection{Keeping IoT Safe (Muts
2024):}\label{keeping-iot-safe-muts2024iot}

We've got all this tech talk going on, but we need to keep it secure:

\begin{itemize}
\item
  \textbf{Secure Devices, Networks, and Data:}

  \begin{itemize}
  \tightlist
  \item
    Locking down your smart devices, making sure the connections are
    bulletproof, and keeping the data safe -- that's the golden rule.
  \end{itemize}
\item
  \textbf{Respecting Privacy:}

  \begin{itemize}
  \tightlist
  \item
    Even in this digital dance, privacy matters. Making sure your smart
    devices aren't oversharing -- that's part of keeping things cool and
    respectful.
  \end{itemize}
\end{itemize}

So, from your toaster to your smartwatch, IoT is like giving everyday
things a digital upgrade and connecting them in a tech-savvy symphony.

\subsection{Platforms for IoT (Muts
2024):}\label{platforms-for-iot-muts2024iot}

So, you've got all these cool IoT ideas, but how do you bring them to
life? Well, these platforms are like the playgrounds where IoT dreams
become reality:

\begin{itemize}
\item
  \textbf{AWS IoT:}

  \begin{itemize}
  \tightlist
  \item
    Amazon's not just about delivering packages; they've got a platform
    dedicated to making your IoT projects soar.
  \end{itemize}
\item
  \textbf{Azure IoT:}

  \begin{itemize}
  \tightlist
  \item
    Microsoft jumps into the IoT game with Azure, providing a space for
    your smart devices to collaborate and shine.
  \end{itemize}
\item
  \textbf{Oracle IoT:}

  \begin{itemize}
  \tightlist
  \item
    has large-scale IoT ecosystems with advanced protective mechanisms
    that differentiate it from other players on the market
  \end{itemize}
\item
  \textbf{Particle IoT:}

  \begin{itemize}
  \tightlist
  \item
    The three main ingredients that make up a platform and make it
    effective are the Particle Photon, responsible for hardware, the
    Integrated Development Environment (IDE) that ensures proper work of
    software, and the Particle Cloud, for the Internet.
  \end{itemize}
\item
  \textbf{IRI Voracity}

  \begin{itemize}
  \tightlist
  \item
    serves as a tool for managing big data. It allows full control and
    multiple manipulations with data that businesses run daily. The main
    effort is focused on the collection, sorting, analyzing of
    information and turning it into a valuable asset
  \end{itemize}
\end{itemize}

\subsection{Real Uses of IoT (Ivey
2023):}\label{real-uses-of-iot-ivey2023iotapplications}

Now that we've got the tools, let's see where people are putting IoT to
work in the real world:

\begin{itemize}
\item
  \textbf{Smart Cities:}

  \begin{itemize}
  \tightlist
  \item
    Imagine a city where traffic lights know when you're running late
    and adjust just for you. That's the magic of smart cities powered by
    IoT.
  \end{itemize}
\item
  \textbf{Industrial IoT (IIoT):}

  \begin{itemize}
  \tightlist
  \item
    Factories becoming smarter, predicting when machines need a break,
    and optimizing production -- that's IIoT at play.
  \end{itemize}
\item
  \textbf{Healthcare IoT:}

  \begin{itemize}
  \tightlist
  \item
    Wearable tech monitoring your health, smart pills sending data to
    your doc -- welcome to the future of healthcare.
  \end{itemize}
\item
  \textbf{Smart Homes:}

  \begin{itemize}
  \tightlist
  \item
    From smart thermostats that know when you're chilly to fridges that
    make grocery lists for you -- IoT is turning homes into tech havens.
  \end{itemize}
\item
  \textbf{Agriculture IoT:}

  \begin{itemize}
  \tightlist
  \item
    Farms aren't left out! IoT helps farmers monitor crops, control
    irrigation, and even track the health of their livestock.
  \end{itemize}
\end{itemize}

\subsection{Challenges with IoT (Geeks
2023):}\label{challenges-with-iot-geeks_for_geeks_iot_challenges}

Of course, with great tech comes great challenges:

\begin{itemize}
\item
  \textbf{Security and Privacy Issues:}

  \begin{itemize}
  \tightlist
  \item
    Keeping your smart fridge from sharing your late-night snack habits
    -- that's the challenge. Privacy and security are always on the
    front lines.
  \end{itemize}
\item
  \textbf{Making Different Devices Work Together:}

  \begin{itemize}
  \tightlist
  \item
    It's like getting a team of superheroes to cooperate -- making sure
    your smartwatch talks nicely to your smart home speaker.
  \end{itemize}
\item
  \textbf{Handling a Lot of Data:}

  \begin{itemize}
  \tightlist
  \item
    With all these smart devices churning out data, you need big brains
    (and servers) to handle the load.
  \end{itemize}
\end{itemize}

\subsection{What's Next for IoT? (Sugandhi
2023):}\label{whats-next-for-iot-sugandhi2023iotfuture}

Hold onto your tech hats; the future's looking even more exciting:

\begin{itemize}
\item
  \textbf{5G Networks for Faster Connections:}

  \begin{itemize}
  \tightlist
  \item
    Imagine everything happening in the blink of an eye -- that's what
    5G brings to the IoT party, supercharging our connections.
  \end{itemize}
\item
  \textbf{Using AI and Machine Learning in IoT:}

  \begin{itemize}
  \tightlist
  \item
    Your devices get smarter with AI and machine learning -- your smart
    home might just predict your pizza cravings before you do.
  \end{itemize}
\item
  \textbf{Future of IOT in Different Fields:}

  \begin{itemize}
  \tightlist
  \item
    With the ability to connect devices, sensors \& equipment to the
    internet, IoT is enabling a new era of data-driven decision making,
    automation \& optimization
  \end{itemize}
\item
  \textbf{Processing Data Closer to Where It's Created (Edge
  Computing):}

  \begin{itemize}
  \tightlist
  \item
    No more sending all that data to the cloud. Edge computing brings
    the processing power closer to home, making things faster and more
    efficient.
  \end{itemize}
\end{itemize}

So, whether you're dreaming up the next big IoT solution or just excited
to see where the tech wave takes us, the Internet of Things is turning
the ordinary into the extraordinary.

\section{Critical Questions}\label{critical-questions}

\subsection{Cloud Computing}\label{cloud-computing-1}

\begin{enumerate}
\def\labelenumi{\arabic{enumi}.}
\item
  How can organizations ensure the security of their data in the cloud,
  especially considering the different deployment models such as public,
  private, and hybrid?
\item
  For businesses considering a move to the cloud, what factors should
  they weigh in deciding between Infrastructure as a Service (IaaS),
  Platform as a Service (PaaS), and Software as a Service (SaaS)?
\item
  In the context of cloud storage and data handling, what are the key
  considerations for establishing effective data governance policies and
  ensuring compliance with regulations? When it comes to storing stuff
  in the cloud, what kind of rules should companies set up?
\item
  As businesses increasingly adopt multiple cloud services, how can they
  address challenges related to interoperability and data consistency
  across different cloud platforms? With everyone using different cloud
  services, how do they make sure all these play nice together?
\item
  Considering emerging trends like edge computing, quantum computing
  exploration, and the use of artificial intelligence in the cloud, how
  might these advancements shape the future landscape of cloud
  computing, and what challenges might arise? How's it shaking up the
  tech world, and what's the deal with AI teaming up with the cloud?
\end{enumerate}

\subsection{Internet of Things}\label{internet-of-things}

\begin{enumerate}
\def\labelenumi{\arabic{enumi}.}
\item
  In examining the integration of everyday objects into the digital
  sphere through IoT, how does this transformative process occur,
  enabling mundane items to communicate and collaborate over the
  internet? How do they make everyday stuff talk in the Internet of
  Things?
\item
  If one seeks to comprehend the operational intricacies of binding
  devices, cloud infrastructure, and applications in the realm of IoT,
  what fundamental principles govern their collective functionality?
\end{enumerate}

Delving into the communication protocols of IoT devices, what
significance do languages like MQTT, CoAP, and others hold in
facilitating seamless inter-device communication within the Internet of
Things? 1. Considering the proliferation of smart devices, what
protocols and measures are in place to safeguard the integrity and
privacy of IoT ecosystems? How can users ensure the security of their
connected devices? With all these smart gadgets around, how do we make
sure they stay secure and respect our privacy?

\begin{enumerate}
\def\labelenumi{\arabic{enumi}.}
\tightlist
\item
  Exploring real-world applications, how does IoT contribute to the
  advancement of society, specifically in areas such as Smart Homes,
  Industrial IoT, and Agriculture IoT? What tangible benefits arise from
  the integration of IoT technologies in these domains?
\end{enumerate}

\section{Assessment Section}\label{assessment-section}

\subsection{Cloud Computing:}\label{cloud-computing-2}

\begin{itemize}
\tightlist
\item
  Think of an app that you use daily. How was cloud computing
  incorporated or utilized in its operations? Suggest improvements for
  the app.
\end{itemize}

\subsection{Internet of Things (IoT):}\label{internet-of-things-iot-1}

\begin{itemize}
\tightlist
\item
  If you were to make something become a smart device, what would it be?
  What functions will it do? How can society benefit from that
  innovation?
\end{itemize}

\section{References}\label{references-4}

\phantomsection\label{refs}
\begin{CSLReferences}{1}{0}
\bibitem[\citeproctext]{ref-abelson1996qc}
Abelson, Harold, and Gerald Jay Sussman. 1996. \emph{Structure and
Interpretation of Computer Programs}. 2nd ed. MIT Electrical Engineering
and Computer Science. London, England: MIT Press.

\bibitem[\citeproctext]{ref-aws_cdn}
Amazon Web Services. n.d. {``What Is a CDN (Content Delivery
Network)?''} n.d. \url{https://aws.amazon.com/what-is/cdn/}.

\bibitem[\citeproctext]{ref-aws_load_balancing}
---------. n.d. {``What Is Load Balancing?''} n.d.
\url{https://aws.amazon.com/what-is/load-balancing/}.

\bibitem[\citeproctext]{ref-amonoo2023stakeholder}
Amonoo Nkrumah, B., W. Qian, A. Kaur, and C. Tilt. 2023. {``Stakeholder
Accountability in the Era of Big Data: An Exploratory Study of Online
Platform Companies.''} \emph{Qualitative Research in Accounting \&
Management} 20 (4): 447--84.
\url{https://doi.org/10.1108/QRAM-03-2022-0042}.

\bibitem[\citeproctext]{ref-annunziato2020experience}
Annunziato, M. 2020. {``The {`Experience Age'} Is Already Here.''} The
Daily Campus. 2020.
\url{https://dailycampus.com/2020/10/09/the-experience-age-is-already-here/}.

\bibitem[\citeproctext]{ref-Awati2021bz}
Awati, Rahul, and Linda Rosencrance. 2021. {``Computer Hardware.''}
\url{https://www.techtarget.com/searchnetworking/definition/hardware};
TechTarget. October 2021.

\bibitem[\citeproctext]{ref-blinova2023corporatesustainability}
Blinova, E., T. Ponomarenko, and S. Tesovskaya. 2023. {``Key Corporate
Sustainability Assessment Methods for Coal Companies.''}
\emph{Sustainability (2071-1050)} 15 (7): 5763.
\url{https://doi.org/10.3390/su15075763}.

\bibitem[\citeproctext]{ref-busuena2023introduction}
Busueña, J., and J. Pomperada. 2023. \emph{Introduction to Information
Technology and Computer Fundamentals}. 2nd ed. Unlimited Books Library
Services; Publishing Inc.

\bibitem[\citeproctext]{ref-clarke2018individual}
Clarke, Blanaid. 2018. {``Individual Accountability in Irish Credit
Institutions---Lessons to Be Learned from the United Kingdom's Senior
Managers' Regime.''} \emph{Common Law World Review}, 35--52.

\bibitem[\citeproctext]{ref-cloudflare_vpc}
Cloudfare. n.d. {``What Is a Virtual Private Cloud (VPC)?''} n.d.
\url{https://www.cloudflare.com/learning/cloud/what-is-a-virtual-private-cloud/}.

\bibitem[\citeproctext]{ref-cocca2022dn}
Cocca, Germán. 2022. {``Programming Paradigms -- Paradigm Examples for
Beginners.''}
\url{https://www.freecodecamp.org/news/an-introduction-to-programming-paradigms/}.
May 2022.

\bibitem[\citeproctext]{ref-djordjevic2019corporate}
Đorđević, D. B., M. Vuković, S. Urošević, N. Štrbac, and A. Vuković.
2019. {``Studying the Corporate Social Responsibility in Apparel and
Textile Industry.''} \emph{Industria Textila} 70 (4): 336--41.
\url{https://doi.org/10.35530/IT.070.04.1572}.

\bibitem[\citeproctext]{ref-englander2021mq}
Englander, Irv, and Wilson Wong. 2021. \emph{The Architecture of
Computer Hardware, Systems Software, and Networking}. 6th ed. Nashville,
TN: John Wiley \& Sons.

\bibitem[\citeproctext]{ref-geeks_for_geeks_iot_challenges}
Geeks, Geeks for. 2023. {``Challenges in Internet of Things (IoT).''}
2023.
\url{https://www.geeksforgeeks.org/challenges-in-internet-of-things-iot/}.

\bibitem[\citeproctext]{ref-gillis2023iot}
Gillis, A. S. 2023. {``Internet of Things (IoT).''} 2023.
\url{https://www.techtarget.com/iotagenda/definition/Internet-of-Things-IoT}.

\bibitem[\citeproctext]{ref-gilster2001pc}
Gilster, Ron. 2001. \emph{{PC} Hardware: A Beginner's Guide}. New York:
Osborne/{McGraw}-Hill. \url{http://site.ebrary.com/id/10015274}.

\bibitem[\citeproctext]{ref-goodman2008information}
Goodman, Seymour, Detmar W. Straub, and Richard Baskerville. 2008.
\emph{Information Security: Policy, Processes, and Practices}.
Routledge.

\bibitem[\citeproctext]{ref-google_cloud_storage}
Google. n.d. {``What Is Cloud Storage?''} n.d.
\url{https://cloud.google.com/learn/what-is-cloud-storage}.

\bibitem[\citeproctext]{ref-hameed2023association}
Hameed, F., M. Alfaraj, and K. Hameed. 2023. {``The Association of Board
Characteristics and Corporate Social Responsibility Disclosure Quality:
Empirical Evidence from Pakistan.''} \emph{Sustainability (2071-1050)}
15 (24): 16849. \url{https://doi.org/10.3390/su152416849}.

\bibitem[\citeproctext]{ref-honigsberg2019individual}
Honigsberg, C. 2019. {``The Case for Individual Audit Partner
Accountability.''} \emph{Vanderbilt Law Review} 72 (6): 1871--1922.

\bibitem[\citeproctext]{ref-ibm_cloud_computing}
IBM. n.d. {``What Is Cloud Computing?''} n.d.
\url{https://www.ibm.com/topics/cloud-computing}.

\bibitem[\citeproctext]{ref-ibm_iot}
---------. n.d. {``What Is the Internet of Things (IoT)?''} n.d.
\url{https://www.ibm.com/topics/internet-of-things}.

\bibitem[\citeproctext]{ref-ivey2023iotapplications}
Ivey, A. 2023. {``7 Real-World IoT Applications and Examples.''} 2023.
\url{https://cointelegraph.com/news/7-iot-applications-and-examples}.

\bibitem[\citeproctext]{ref-jebaraj2023cloudcompanies}
Jebaraj, K. 2023. {``Top 10 Cloud Computing Companies of 2024.''} 2023.
\url{https://www.knowledgehut.com/blog/cloud-computing/top-cloud-computing-companies}.

\bibitem[\citeproctext]{ref-jeursen2022cover}
Jeursen, T. 2022. {``"Cover Your Ass": Individual Accountability, Visual
Documentation, and Everyday Policing in Miami.''} \emph{PoLAR: Political
\& Legal Anthropology Review} 45 (2): 186--200.
\url{https://doi.org/10.1111/plar.12505}.

\bibitem[\citeproctext]{ref-kang2022sustainabletraining}
Kang, Y.-C., H.-S. Hsiao, and J.-Y. Ni. 2022. {``The Role of Sustainable
Training and Reward in Influencing Employee Accountability Perception
and Behavior for Corporate Sustainability.''} \emph{Sustainability
(2071-1050)} 14 (18): 11589--N.PAG.
\url{https://doi.org/10.3390/su141811589}.

\bibitem[\citeproctext]{ref-keutzer1994he}
Keutzer, Kurt. 1994. {``Hardware/Software Co-Simulation.''} In
\emph{Proceedings of the 31st Annual Conference on Design Automation
Conference - {DAC} '94}. New York, New York, USA: ACM Press.

\bibitem[\citeproctext]{ref-lombardi2004information}
Lombardi, Olimpia. 2004. {``What Is Information?''} \emph{Foundations of
Science} 9: 105--34.
\url{https://doi.org/10.1023/B:FODA.0000025034.53313.7c}.

\bibitem[\citeproctext]{ref-lundin2023information}
Lundin, L. L. 2023. \emph{Information Security}. Salem Press
Encyclopedia.

\bibitem[\citeproctext]{ref-marr2023cloudtrends}
Marr, B. 2023. {``The 10 Biggest Cloud Computing Trends in 2024 Everyone
Must Be Ready for Now.''} 2023.
\url{https://www.forbes.com/sites/bernardmarr/2023/10/09/the-10-biggest-cloud-computing-trends-in-2024-everyone-must-be-ready-for-now/?sh=ea1da466d672}.

\bibitem[\citeproctext]{ref-microsoft2023securedata}
Microsoft. 2023. {``11 Best Practices for Securing Data in Cloud
Services.''} 2023.
\url{https://www.microsoft.com/en-us/security/blog/2023/07/05/11-best-practices-for-securing-data-in-cloud-services/}.

\bibitem[\citeproctext]{ref-muts2024iot}
Muts, I. 2024. {``10+ Best IoT Cloud Platforms in 2024.''} 2024.
\url{https://euristiq.com/best-iot-cloud-platforms/}.

\bibitem[\citeproctext]{ref-sebesta2015ek}
Sebesta, Robert W. 2015. \emph{Concepts of Programming Languages}. 11th
ed. Upper Saddle River, NJ: Pearson.

\bibitem[\citeproctext]{ref-stallings2016computer}
Stallings, William. 2016. \emph{Computer Organization and Architecture}.
10th edition. Pearson.

\bibitem[\citeproctext]{ref-sugandhi2023iotfuture}
Sugandhi, A. 2023. {``The Future of IoT: Trends and Predictions for
2024.''} 2023.
\url{https://www.knowledgehut.com/blog/web-development/iot-future}.

\bibitem[\citeproctext]{ref-velazquez2022iot}
Velazquez, R. 2022. {``IoT: The Internet of Things. What Is the Internet
of Things? How Does IoT Work?''} 2022.
\url{https://builtin.com/internet-things}.

\bibitem[\citeproctext]{ref-weiser1999og}
Weiser, Mark. 1999. {``The Computer for the 21 \(^{st}\) Century.''}
\emph{ACM SIGMOBILE Mob. Comput. Commun. Rev.} 3 (3): 3--11.

\bibitem[\citeproctext]{ref-williams2012using}
Williams, B. K., and S. C. Sawyer. 2012. \emph{Using Information
Technology: Introductory Edition}. McGraw-Hill Europe.

\bibitem[\citeproctext]{ref-zhang2023environmental}
Zhang, Y., M. Imeni, and S. A. Edalatpanah. 2023. {``Environmental
Dimension of Corporate Social Responsibility and Earnings Persistence:
An Exploration of the Moderator Roles of Operating Efficiency and
Financing Cost.''} \emph{Sustainability (2071-1050)} 15 (20): 14814.
\url{https://doi.org/10.3390/su152014814}.

\end{CSLReferences}

\section{Keywords}\label{keywords-1}

\subsection{Cloud Computing:}\label{cloud-computing-3}

\begin{itemize}
\tightlist
\item
  Cloud Computing, Virtual Machines, Infrastructure as a Service (IaaS),
  Platform as a Service (PaaS), Software as a Service (SaaS), Public
  Cloud, Private Cloud, Hybrid Cloud, Community Cloud, Encryption,
  Identity and Access Management (IAM), Best Practices for Security,
  Amazon (AWS), Microsoft (Azure), Google (GCP), IBM, Oracle, Encrypting
  Data, Identity and Access Management (IAM), Security, Best Practices
  for Security, Objects, Blocks, Files, Content Delivery Networks (CDN),
  Virtual Private Cloud (VPC), Load Balancing, Content Delivery Networks
  (CDN), Edge Computing, Using Multiple Cloud Services Together, Quantum
  Computing
\end{itemize}

\subsection{Internet of Things (IoT):}\label{internet-of-things-iot-2}

\begin{itemize}
\tightlist
\item
  IoT, Connectivity, Things, Data Processing, User Interface, Devices,
  Connectivity, Cloud, Applications, MQTT, CoAP, HTTP/HTTPS, AMQP,
  LoRaWAN, Zigbee, Secure Devices, Networks, Data, Privacy, AWS IoT,
  Azure IoT, Google Cloud IoT, IBM Watson IoT, Smart Cities, Industrial
  IoT (IIoT), Healthcare IoT, Smart Homes, Agriculture IoT, Security and
  Privacy Issues, Interoperability, Handling Data, 5G Networks, AI,
  Machine Learning, Edge Computing
\end{itemize}

\bookmarksetup{startatroot}

\chapter*{References}\label{references-5}
\addcontentsline{toc}{chapter}{References}

\markboth{References}{References}

\phantomsection\label{refs}
\begin{CSLReferences}{1}{0}
\bibitem[\citeproctext]{ref-abelson1996qc}
Abelson, Harold, and Gerald Jay Sussman. 1996. \emph{Structure and
Interpretation of Computer Programs}. 2nd ed. MIT Electrical Engineering
and Computer Science. London, England: MIT Press.

\bibitem[\citeproctext]{ref-aws_cdn}
Amazon Web Services. n.d. {``What Is a CDN (Content Delivery
Network)?''} n.d. \url{https://aws.amazon.com/what-is/cdn/}.

\bibitem[\citeproctext]{ref-aws_load_balancing}
---------. n.d. {``What Is Load Balancing?''} n.d.
\url{https://aws.amazon.com/what-is/load-balancing/}.

\bibitem[\citeproctext]{ref-amonoo2023stakeholder}
Amonoo Nkrumah, B., W. Qian, A. Kaur, and C. Tilt. 2023. {``Stakeholder
Accountability in the Era of Big Data: An Exploratory Study of Online
Platform Companies.''} \emph{Qualitative Research in Accounting \&
Management} 20 (4): 447--84.
\url{https://doi.org/10.1108/QRAM-03-2022-0042}.

\bibitem[\citeproctext]{ref-annunziato2020experience}
Annunziato, M. 2020. {``The {`Experience Age'} Is Already Here.''} The
Daily Campus. 2020.
\url{https://dailycampus.com/2020/10/09/the-experience-age-is-already-here/}.

\bibitem[\citeproctext]{ref-Awati2021bz}
Awati, Rahul, and Linda Rosencrance. 2021. {``Computer Hardware.''}
\url{https://www.techtarget.com/searchnetworking/definition/hardware};
TechTarget. October 2021.

\bibitem[\citeproctext]{ref-blinova2023corporatesustainability}
Blinova, E., T. Ponomarenko, and S. Tesovskaya. 2023. {``Key Corporate
Sustainability Assessment Methods for Coal Companies.''}
\emph{Sustainability (2071-1050)} 15 (7): 5763.
\url{https://doi.org/10.3390/su15075763}.

\bibitem[\citeproctext]{ref-busuena2023introduction}
Busueña, J., and J. Pomperada. 2023. \emph{Introduction to Information
Technology and Computer Fundamentals}. 2nd ed. Unlimited Books Library
Services; Publishing Inc.

\bibitem[\citeproctext]{ref-clarke2018individual}
Clarke, Blanaid. 2018. {``Individual Accountability in Irish Credit
Institutions---Lessons to Be Learned from the United Kingdom's Senior
Managers' Regime.''} \emph{Common Law World Review}, 35--52.

\bibitem[\citeproctext]{ref-cloudflare_vpc}
Cloudfare. n.d. {``What Is a Virtual Private Cloud (VPC)?''} n.d.
\url{https://www.cloudflare.com/learning/cloud/what-is-a-virtual-private-cloud/}.

\bibitem[\citeproctext]{ref-cocca2022dn}
Cocca, Germán. 2022. {``Programming Paradigms -- Paradigm Examples for
Beginners.''}
\url{https://www.freecodecamp.org/news/an-introduction-to-programming-paradigms/}.
May 2022.

\bibitem[\citeproctext]{ref-djordjevic2019corporate}
Đorđević, D. B., M. Vuković, S. Urošević, N. Štrbac, and A. Vuković.
2019. {``Studying the Corporate Social Responsibility in Apparel and
Textile Industry.''} \emph{Industria Textila} 70 (4): 336--41.
\url{https://doi.org/10.35530/IT.070.04.1572}.

\bibitem[\citeproctext]{ref-englander2021mq}
Englander, Irv, and Wilson Wong. 2021. \emph{The Architecture of
Computer Hardware, Systems Software, and Networking}. 6th ed. Nashville,
TN: John Wiley \& Sons.

\bibitem[\citeproctext]{ref-geeks_for_geeks_iot_challenges}
Geeks, Geeks for. 2023. {``Challenges in Internet of Things (IoT).''}
2023.
\url{https://www.geeksforgeeks.org/challenges-in-internet-of-things-iot/}.

\bibitem[\citeproctext]{ref-gillis2023iot}
Gillis, A. S. 2023. {``Internet of Things (IoT).''} 2023.
\url{https://www.techtarget.com/iotagenda/definition/Internet-of-Things-IoT}.

\bibitem[\citeproctext]{ref-gilster2001pc}
Gilster, Ron. 2001. \emph{{PC} Hardware: A Beginner's Guide}. New York:
Osborne/{McGraw}-Hill. \url{http://site.ebrary.com/id/10015274}.

\bibitem[\citeproctext]{ref-goodman2008information}
Goodman, Seymour, Detmar W. Straub, and Richard Baskerville. 2008.
\emph{Information Security: Policy, Processes, and Practices}.
Routledge.

\bibitem[\citeproctext]{ref-google_cloud_storage}
Google. n.d. {``What Is Cloud Storage?''} n.d.
\url{https://cloud.google.com/learn/what-is-cloud-storage}.

\bibitem[\citeproctext]{ref-hameed2023association}
Hameed, F., M. Alfaraj, and K. Hameed. 2023. {``The Association of Board
Characteristics and Corporate Social Responsibility Disclosure Quality:
Empirical Evidence from Pakistan.''} \emph{Sustainability (2071-1050)}
15 (24): 16849. \url{https://doi.org/10.3390/su152416849}.

\bibitem[\citeproctext]{ref-honigsberg2019individual}
Honigsberg, C. 2019. {``The Case for Individual Audit Partner
Accountability.''} \emph{Vanderbilt Law Review} 72 (6): 1871--1922.

\bibitem[\citeproctext]{ref-ibm_cloud_computing}
IBM. n.d. {``What Is Cloud Computing?''} n.d.
\url{https://www.ibm.com/topics/cloud-computing}.

\bibitem[\citeproctext]{ref-ibm_iot}
---------. n.d. {``What Is the Internet of Things (IoT)?''} n.d.
\url{https://www.ibm.com/topics/internet-of-things}.

\bibitem[\citeproctext]{ref-ivey2023iotapplications}
Ivey, A. 2023. {``7 Real-World IoT Applications and Examples.''} 2023.
\url{https://cointelegraph.com/news/7-iot-applications-and-examples}.

\bibitem[\citeproctext]{ref-jebaraj2023cloudcompanies}
Jebaraj, K. 2023. {``Top 10 Cloud Computing Companies of 2024.''} 2023.
\url{https://www.knowledgehut.com/blog/cloud-computing/top-cloud-computing-companies}.

\bibitem[\citeproctext]{ref-jeursen2022cover}
Jeursen, T. 2022. {``"Cover Your Ass": Individual Accountability, Visual
Documentation, and Everyday Policing in Miami.''} \emph{PoLAR: Political
\& Legal Anthropology Review} 45 (2): 186--200.
\url{https://doi.org/10.1111/plar.12505}.

\bibitem[\citeproctext]{ref-kang2022sustainabletraining}
Kang, Y.-C., H.-S. Hsiao, and J.-Y. Ni. 2022. {``The Role of Sustainable
Training and Reward in Influencing Employee Accountability Perception
and Behavior for Corporate Sustainability.''} \emph{Sustainability
(2071-1050)} 14 (18): 11589--N.PAG.
\url{https://doi.org/10.3390/su141811589}.

\bibitem[\citeproctext]{ref-keutzer1994he}
Keutzer, Kurt. 1994. {``Hardware/Software Co-Simulation.''} In
\emph{Proceedings of the 31st Annual Conference on Design Automation
Conference - {DAC} '94}. New York, New York, USA: ACM Press.

\bibitem[\citeproctext]{ref-lombardi2004information}
Lombardi, Olimpia. 2004. {``What Is Information?''} \emph{Foundations of
Science} 9: 105--34.
\url{https://doi.org/10.1023/B:FODA.0000025034.53313.7c}.

\bibitem[\citeproctext]{ref-lundin2023information}
Lundin, L. L. 2023. \emph{Information Security}. Salem Press
Encyclopedia.

\bibitem[\citeproctext]{ref-marr2023cloudtrends}
Marr, B. 2023. {``The 10 Biggest Cloud Computing Trends in 2024 Everyone
Must Be Ready for Now.''} 2023.
\url{https://www.forbes.com/sites/bernardmarr/2023/10/09/the-10-biggest-cloud-computing-trends-in-2024-everyone-must-be-ready-for-now/?sh=ea1da466d672}.

\bibitem[\citeproctext]{ref-microsoft2023securedata}
Microsoft. 2023. {``11 Best Practices for Securing Data in Cloud
Services.''} 2023.
\url{https://www.microsoft.com/en-us/security/blog/2023/07/05/11-best-practices-for-securing-data-in-cloud-services/}.

\bibitem[\citeproctext]{ref-muts2024iot}
Muts, I. 2024. {``10+ Best IoT Cloud Platforms in 2024.''} 2024.
\url{https://euristiq.com/best-iot-cloud-platforms/}.

\bibitem[\citeproctext]{ref-sebesta2015ek}
Sebesta, Robert W. 2015. \emph{Concepts of Programming Languages}. 11th
ed. Upper Saddle River, NJ: Pearson.

\bibitem[\citeproctext]{ref-stallings2016computer}
Stallings, William. 2016. \emph{Computer Organization and Architecture}.
10th edition. Pearson.

\bibitem[\citeproctext]{ref-sugandhi2023iotfuture}
Sugandhi, A. 2023. {``The Future of IoT: Trends and Predictions for
2024.''} 2023.
\url{https://www.knowledgehut.com/blog/web-development/iot-future}.

\bibitem[\citeproctext]{ref-velazquez2022iot}
Velazquez, R. 2022. {``IoT: The Internet of Things. What Is the Internet
of Things? How Does IoT Work?''} 2022.
\url{https://builtin.com/internet-things}.

\bibitem[\citeproctext]{ref-weiser1999og}
Weiser, Mark. 1999. {``The Computer for the 21 \(^{st}\) Century.''}
\emph{ACM SIGMOBILE Mob. Comput. Commun. Rev.} 3 (3): 3--11.

\bibitem[\citeproctext]{ref-williams2012using}
Williams, B. K., and S. C. Sawyer. 2012. \emph{Using Information
Technology: Introductory Edition}. McGraw-Hill Europe.

\bibitem[\citeproctext]{ref-zhang2023environmental}
Zhang, Y., M. Imeni, and S. A. Edalatpanah. 2023. {``Environmental
Dimension of Corporate Social Responsibility and Earnings Persistence:
An Exploration of the Moderator Roles of Operating Efficiency and
Financing Cost.''} \emph{Sustainability (2071-1050)} 15 (20): 14814.
\url{https://doi.org/10.3390/su152014814}.

\end{CSLReferences}



\end{document}
